%!TEX TS-program = pdflatex
% dissertation.tex -- main dissertation file
%
% Wisconsin dissertation template
% Copyright (c) 2008-2009 William C. Benton.  All rights reserved.
%
% This program can redistributed and/or modified under the terms
% of the LaTeX Project Public License Distributed from CTAN
% archives in directory macros/latex/base/lppl.txt; either
% version 1 of the License, or (at your option) any later version.
%
% This program includes other software that is licensed under the
% terms of the LPPL and the Perl Artistic License; see README for details.
%
% You, the user, still hold the copyright to any document you produce
% with this software (like your dissertation).
%
\PassOptionsToPackage{dvipsnames}{xcolor} % prevent an option clash
% Comment to insert real images
\PassOptionsToPackage{draft}{graphicx}

%%% You'll want ``oneside'' for the deposit version, but probably not for any versions that don't need to meet the UW requirements
\documentclass[12pt,oneside,letterpaper]{memoir}

% preamble.tex -- packages to include
%
% Wisconsin dissertation template
% Copyright (c) 2008 William C. Benton.  All rights reserved.
%
% This program can redistributed and/or modified under the terms
% of the LaTeX Project Public License Distributed from CTAN
% archives in directory macros/latex/base/lppl.txt; either
% version 1 of the License, or (at your option) any later version.
%
% This program includes other software that is licensed under the
% terms of the LPPL and the Perl Artistic License; see README for details.
%
% You, the user, still hold the copyright to any document you produce
% with this software (like your dissertation).

%% You should use natbib
\IfFileExists{natbib.sty}{%
\usepackage[authoryear]{natbib}%
}{}

\usepackage[margin=1in]{geometry}
\ifpdf
\usepackage[pdftex]{graphicx}
\else
\usepackage{graphicx}
\fi
\usepackage{pdflscape}
\usepackage{rotating}
%% You probably need appendix, if you want appendices
\IfFileExists{appendix.sty}{%
\usepackage{appendix}%
}{}


%% the spacing in memoir is weird, you'll need to use this
\DisemulatePackage{setspace}
%% \usepackage[onehalfspacing]{setspace}
\usepackage[doublespacing]{setspace}

%% List setup; the ``hanglist`` environment will allow you to have
%% nicely-typeset enumerated lists (i.e. with the numbers hanging in
%% the margins).  You need at least version 2.1 of enumitem.sty.  If
%% you don't have enumitem installed at all, hanglist will just be an
%% alias for enumerate.
\IfFileExists{enumitem.sty}{%
\usepackage[loadonly]{enumitem}[2007/06/30]%
\newlist{hanglist}{enumerate}{1}% 
\setlist[hanglist]{label=\arabic*.}%
\setlist[hanglist,1]{leftmargin=0pt}%
}{%
\newenvironment{hanglist}{\begin{enumerate}}{\end{enumerate}}%
}


%% Avoid ugly "Type 3" fonts
\usepackage{lmodern}
\usepackage[LY1]{fontenc}
\usepackage[LY1]{eulervm}
%% Comment out any of these that you don't want
\usepackage{amssymb}
\usepackage{amsmath}
\usepackage{amsthm}
\usepackage{bm}
%\usepackage{theorem}
\usepackage{hyperref}
\usepackage[dvipsnames]{xcolor}
\usepackage{wrapfig}
\usepackage[nonumberlist,xindy,toc]{glossaries}

\usepackage{algorithm}
\usepackage[noend]{algpseudocode}
\usepackage{color}
\usepackage{framed}
\definecolor{shadecolor}{gray}{0.9}

\algnewcommand{\algorithmicreshape}{\textbf{reshape}}%
\algnewcommand{\Reshape}{\algorithmicreshape~}%

\algnewcommand{\algorithmicrecompute}{\textbf{compute}}%
\algnewcommand{\Compute}{\algorithmicrecompute~}%

\algnewcommand{\algorithmicreset}{\textbf{set}}%
\algnewcommand{\Set}{\algorithmicreset~}%

\renewcommand{\algorithmicrequire}{\textbf{Input:}}
\renewcommand{\algorithmicensure}{\textbf{Output:}}

\usepackage{multirow}
\usepackage{array}
\usepackage{booktabs}
\usepackage{comment}
\usepackage{forest}
\usepackage{float}

\newcommand{\tr}{\ensuremath{\mbox{tr}}}
\newcommand{\str}{\ensuremath{\mbox{\small tr}}}
\newcommand{\bvec}[1]{\hbox{\boldmath $#1$}}
\newcolumntype{C}[1]{>{\centering\let\newline\\\arraybackslash\hspace{0pt}}m{#1}}


\definecolor{light-gray}{gray}{.9}
\definecolor{dark-gray}{gray}{0.60}

\newcommand{\reducedstrut}{\vrule width 0pt height .9\ht\strutbox depth .9\dp\strutbox\relax}
\newcommand{\chighlight}[1]{%
  \begingroup
  \setlength{\fboxsep}{.5pt}%  
  \colorbox{light-gray}{\reducedstrut#1\/}%
  \endgroup
}

\usepackage{tikz}
\usetikzlibrary{decorations.pathreplacing,calc}
\newcommand{\tikzmark}[1]{\tikz[overlay,remember picture] \node (#1) {};}

\newcommand*{\AddNote}[4]{%
    \begin{tikzpicture}[overlay, remember picture]
        \draw [decoration={brace,amplitude=0.5em},decorate,ultra thick,red]
            ($(#3)!(#1.north)!($(#3)-(0,1)$)$) --  
            ($(#3)!(#2.south)!($(#3)-(0,1)$)$)
                node [align=center, text width=2.5cm, pos=0.5, anchor=west] {#4};
    \end{tikzpicture}
}%


\IfFileExists{mathpartir.sty}{%
\usepackage{mathpartir}%
}{}

%%%%% LISTINGS package and setup
\IfFileExists{listings.sty}{%
\usepackage{listings}%
}{}



%% Get rid of ugly borders around PDF hyperlinks (e.g. for cross-references, bib entries, or URLs)
\hypersetup{pdfborder = 0 0 0}

%% You want microtype.
\IfFileExists{microtype.sty}{%
\usepackage[protrusion=true,expansion=true]{microtype}%
}{}

%\pagestyle{thesisdraft}

\usepackage{afterpage}

% Surround parts of graphics with box
\usepackage{boxedminipage}

%% booktabs (thx to Nate Rosenblum for bringing this beautiful package
%% to my attention)
\IfFileExists{booktabs.sty}{%
\usepackage{booktabs}%
}{}

% This is now the recommended way for checking for PDFLaTeX:
\usepackage{ifpdf}



%% Substitute your favorite serif and sans fonts here....
%\IfFileExists{tgpagella.sty}{%
% TeX Gyre pagella, like Palatino
%\usepackage{tgpagella}%
%}{}

\usepackage{ifthen}

\providecommand\pdfforprinting{true}

\ifthenelse{\equal{\pdfforprinting}{false}}{%
\newenvironment{myrotatedtable}{%
  \newgeometry{margin=1cm}%
  \landscape%
  \table[p]%
}{%
  \endtable%
  \endlandscape%
  \restoregeometry%
}%
}{%
\newenvironment{myrotatedtable}{%
    \begin{sidewaystable}%
}{%
    \end{sidewaystable}
}%
}%

\usepackage{makeidx}
\makeindex
\makeglossaries

{\theoremstyle{plain}
\newtheorem{thm}{Theorem}[chapter]
\newtheorem{cor}[thm]{Corollary}
\newtheorem{define}[thm]{Definition}
\newtheorem{exmpl}[thm]{Example}
}
{\theoremstyle{remark}
\newtheorem{rmk}[thm]{Remark}
}

\newtheoremstyle{customsty1}
{3pt}%
{3pt}%
{}% --- body font
{}% --- indent amount
{\bfseries}% --- Theorem head font
{:}% --- Punctuation after head
{.5em}% --- space after head
{}% --- theorem head spec (can be left empty, meaning 'normal')

% Define 'newtheorems' that use ``customsty1''
{\theoremstyle{customsty1} 
}

% Some helper commands used in original source documents
\newcommand{\newtext}[1]{#1}
\newcommand{\oldtext}[1]{}
\newcommand{\newmath}[1]{\ensuremath{#1}}
\newcommand{\oldmath}[1]{}
\newcommand{\addedfragile}[1]{#1}
\renewcommand{\changed}[2]{\newtext{#1}\oldtext{#2}}




%%% NB: the ``deposit'' chapter- and page- styles should conform to UW
%%% requirements.  If you are producing a pretty version of your
%%% dissertation for web use later, you will certainly want to make
%%% your own chapter and page styles.

\makechapterstyle{deposit}{%
  \renewcommand{\chapterheadstart}{}
  \renewcommand{\printchaptername}{}
  \renewcommand{\chapternamenum}{}
  \renewcommand{\printchapternum}{\parbox{2em}{\MakeLowercase{\Large\scshape\thechapter{}}} }
  \renewcommand{\afterchapternum}{}
  \renewcommand{\printchaptertitle}[1]{%
  \raggedright\Large\scshape\MakeLowercase{##1}}
  \renewcommand{\afterchaptertitle}{%
  \vskip\onelineskip \hrule\vskip\onelineskip}
}

\makepagestyle{deposit}
 
\makeatletter
 
\renewcommand{\chaptermark}[1]{\markboth{#1}{}}
\renewcommand{\sectionmark}[1]{\markboth{#1}{}}
 
\makeevenfoot{deposit}{}{}{}
\makeoddfoot{deposit}{}{}{}
\makeevenhead{deposit}{\thepage}{}{}
\makeoddhead{deposit}{}{}{\thepage}
\makeatother

%%% set up page numbering for chapter pages to satisfy UW requirements
%%% NB: You will want to delete until the ``SNIP'' mark if you are
%%% making a ``nice'' copy
\copypagestyle{chapter}{plain}
\makeoddfoot{chapter}{}{}{}
\makeevenhead{chapter}{\thepage}{}{}
\makeoddhead{chapter}{}{}{\thepage}
%%% SNIP

%%% bib nonsense
\makeatletter
\newenvironment{wb-bib}[1]{%
  \chapter*{references}
\ifnobibintoc\else 
\phantomsection 
\addcontentsline{toc}{chapter}{References} 
\fi 
\prebibhook
  \begin{bibitemlist}{#1}}{\end{bibitemlist}\postbibhook}

\AtBeginDocument{%
  \@ifpackageloaded{natbib}{% natbib is loaded
    \addtodef{\endthebibliography}{}{\vskip-\lastskip\postbibhook}
    \@ifpackagewith{natbib}{sectionbib}{% with sectionbib option
      \renewcommand{\bibsection}{\@memb@bsec}}%
      {\renewcommand{\bibsection}{\@memb@bchap}}}%
  {}
  \@ifpackagewith{chapterbib}{sectionbib}{%
    \renewcommand{\sectionbib}[2]{}
    \renewcommand{\bibsection}{\@memb@bsec}}{}
}
\makeatother


\renewcommand{\sc}{\scshape}

%%% Local Variables:
%%% mode: latex
%%% TeX-master: "../document"
%%% End:

\input{includes/defs}
\input{includes/thesisdefs}
\newacronym[%
description={A Practical Visual Interactive System is a dynamic
virtual environment where by user input produces changes in state for
the purpose of supporting some task.}]
{pvis}{PVIS}{practical visual interactive system}

\newacronym[%
description={A Head Mounted Display is a virtual reality display
  device that fits over a user's head, providing them with an isolated
  and personal immersive experience into virtual environments.}]
{hmd}{HMD}{head mounted display}

\newglossaryentry{tacit}{%
  description={Tacit Knowledge is knowledge which is difficult to
    communicate to others via direct methods, such as speaking or
    writing. Examples might include how to play an instrument or playing
    a sport.},
  name={tacit knowledge}
}

\newglossaryentry{psymotor}{%
  description={Psychomotor skills are skills that require extensive coordination between the physical body and mental processes. Examples include playing an instrument or physically manipulating surgical tools during surgery.},
  name={psychomotor skill}
  }

\newglossaryentry{deformablesolid}{%
description={A deformable solid is a volumetric region whose shape reacts to both internal mechanics and external stimuli},
name={deformable solid}
}

\newglossaryentry{hyperelastic}{%
description={A hyperelastic material is a material whose strain energy does not depend on the history of past deformations the material has undergone. While idealized and not found in nature, these materials are often described as rubber-like, as rubber and similar real materials are very hyperelastic in their behavior under low stress loads.},
name={hyperelastic}
}

\newglossaryentry{plasticity}{%
description={A plastic material is a material which incorporates previous deformation history when computing the current deformation strain energy. This gives rise to persistent effects like creasing and shape-holding after bends. Common materials with plasticity effects include most metals, paper, and some plastics.},
name={plasticity}
}

\newacronym[%
description={The Finite Element Method is a type of computational analysis of materials which utilizes a number of discrete regions or elements to describe a larger region of material. By using the relatively simple behaviors of each element, a vastly more complex of the whole material can be derived.}]
{fem}{FEM}{Finite Element Method}
%\usepackage[noend]{algcompatible}
\usepackage{algorithm}
\usepackage[noend]{algpseudocode}
\usepackage{tikz}
\usetikzlibrary{calc}
\algnewcommand\algorithmicgoto{\textbf{goto: }}
\algnewcommand\Goto{\algorithmicgoto}
\algrenewcommand\algorithmicindent{0.5em}%
\algnewcommand{\LineComment}[1]{\State {\color{NavyBlue} \(\triangleright\) \textbf{#1}}}

\usepackage{caption}% http://ctan.org/pkg/caption
\captionsetup[ruled]{labelsep=colon}
\makeatletter
\@addtoreset{algorithm}{chapter}% algorithm counter resets every chapter
\makeatother
\renewcommand{\thealgorithm}{\thechapter.\arabic{algorithm}}% Algorithm # is <chapter>.<algorithm>


\makeatletter
\newcommand{\StatexIndent}[1][3]{%
  \setlength\@tempdima{\algorithmicindent}%
  \Statex\hskip\dimexpr#1\@tempdima\relax}
\algdef{S}[IF]{IfNoThen}[1]{\algorithmicif\ #1}%
\makeatother

\makeatletter
\tikzset{%
  remember picture with id/.style={%
    remember picture,
    overlay,
    save picture id=#1,
  },
  save picture id/.code={%
    \edef\pgf@temp{#1}%
    \immediate\write\pgfutil@auxout{%
      \noexpand\savepointas{\pgf@temp}{\pgfpictureid}}%
  },
  if picture id/.code args={#1#2#3}{%
    \@ifundefined{save@pt@#1}{%
      \pgfkeysalso{#3}%
    }{
      \pgfkeysalso{#2}%
    }
  }
}

\def\savepointas#1#2{%
  \expandafter\gdef\csname save@pt@#1\endcsname{#2}%
}

\def\tmk@labeldef#1,#2\@nil{%
  \def\tmk@label{#1}%
  \def\tmk@def{#2}%
}

\tikzdeclarecoordinatesystem{pic}{%
  \pgfutil@in@,{#1}%
  \ifpgfutil@in@%
    \tmk@labeldef#1\@nil
  \else
    \tmk@labeldef#1,(0pt,0pt)\@nil
  \fi
  \@ifundefined{save@pt@\tmk@label}{%
    \tikz@scan@one@point\pgfutil@firstofone\tmk@def
  }{%
  \pgfsys@getposition{\csname save@pt@\tmk@label\endcsname}\save@orig@pic%
  \pgfsys@getposition{\pgfpictureid}\save@this@pic%
  \pgf@process{\pgfpointorigin\save@this@pic}%
  \pgf@xa=\pgf@x
  \pgf@ya=\pgf@y
  \pgf@process{\pgfpointorigin\save@orig@pic}%
  \advance\pgf@x by -\pgf@xa
  \advance\pgf@y by -\pgf@ya
  }%
}


% end of Andrew's code

\newlength\AlgIndent
\setlength\AlgIndent{0pt}
% main command to draw the colored background
\newcounter{mymark}
\newcommand\ColorLine{%
  \stepcounter{mymark}%
  \tikz[remember picture with id=mark-\themymark,overlay] {;}%
  \begin{tikzpicture}[remember picture,overlay]%
    \filldraw[YellowGreen]%
   let \p1=(pic cs:mark-\themymark),
   \p2=(pic cs:mark-\themymark) in 
   ([xshift=-\ALG@thistlm,yshift=-0.7ex]0,\y1)  rectangle ++(\linewidth+.1em,\baselineskip);
  \end{tikzpicture}%
}%
\makeatother

% colored loops and declarations
\makeatletter
\algnewcommand\CREQUIRE{\item[\setlength\AlgIndent{1.6em}\ColorLine\algorithmicrequire]}%
\algnewcommand\CENSURE{\item[\setlength\AlgIndent{1.6em}\ColorLine\algorithmicensure]}%
\algnewcommand\CSTATE{\State\ColorLine}%
\algnewcommand\CSTATEx{\Statex\ColorLine}%
\algnewcommand\CCOMMENT{\Comment\ColorLine}%

\algdef{SE}[WHILE]{CWHILE}{CENDWHILE}%
   [2][default]{\ColorLine\algorithmicwhile\ #2\ \algorithmicdo #1}%
   {\ColorLine\algorithmicend\ \algorithmicwhile}%
\algdef{SE}[FOR]{CFOR}{CENDFOR}%
   [2][default]{\ColorLine\algorithmicfor\ #2\ \algorithmicdo #1}%
   {\ColorLine\algorithmicend\ \algorithmicfor}%
\algdef{SE}[FOR]{CFORALL}{CENDFORALL}%
   [2][default]{\ColorLine\algorithmicforall\ #2\ \algorithmicdo #1}%
   {\vspace{-\baselineskip}%\algorithmicend\ \algorithmicfor
}%
\algdef{SE}[FOR]{FORALL}{ENDFORALL}%
   [2][default]{\algorithmicforall\ #2\ \algorithmicdo #1}%
   {\algorithmicend\ \algorithmicfor}%
\algdef{SE}[LOOP]{CLOOP}{CENDLOOP}%
   [1][default]{\ColorLine\algorithmicloop #1}%
   {\ColorLine\algorithmicend\ \algorithmicloop}%
\algdef{SE}[REPEAT]{CREPEAT}{CUNTIL}%
   [1][default]{\ColorLine\algorithmicrepeat #1}%
   [1]{\ColorLine\algorithmicuntil\ #1}%
\algdef{SE}[IF]{CIF}{CENDIF}%
   [2][default]{\ColorLine\algorithmicif\ #2\ \algorithmicthen #1}%
   {\vspace{-\baselineskip}%\algorithmicend\ \algorithmicif
}%
\algdef{C}[IF]{CIF}{CELSIF}%
   [2][default]{\ColorLine\algorithmicelse\ \algorithmicif\ #2\ \algorithmicthen #1}%
\algdef{Ce}[ELSE]{CIF}{CELSE}{CENDIF}%
   [1][default]{\ColorLine\algorithmicelse #1}%



\makeatother

%%% Local Variables: 
%%% mode: latex
%%% TeX-master: "nonmanifold_levelsets"
%%% End: 


\svnidlong{$LastChangedBy$}{$LastChangedRevision$}{$LastChangedDate$}{$HeadURL: http://freevariable.com/dissertation/branches/diss-template/dissertation.tex $} 

\clearpage\pagenumbering{roman}  % This makes the page numbers Roman (i, ii, etc)

\title{Techniques for Single System Integration of Elastic Simulation Features}
\author{Nathan M. Mitchell}
\department{Computer Sciences}

\date{2016}

\begin{document}

%%% Uncomment the following if your .bib contains references that you will not 
%%% explicitly cite, but that should be in the final bibliography:
% \nocite{*}

\ifpdf
\DeclareGraphicsExtensions{.pdf, .jpg, .tif}
\else
\DeclareGraphicsExtensions{.eps, .jpg}
\fi

\maketitle

%% Add \part declarations if you want, but it's not necessary
%\part{Preliminaries}

\svnidlong{$LastChangedBy$}{$LastChangedRevision$}{$LastChangedDate$}{$HeadURL: http://freevariable.com/dissertation/branches/diss-template/frontmatter/frontmatter.tex $}
\vcinfo{}

%%% SOME OF THIS CODE IS ADAPTED FROM THE VENERABLE withesis.cls

% COPYRIGHT PAGE
%  - To include a copyright page use \copyrightpage
\copyrightpage

% DEDICATION
\begin{dedication}
  \emph{For my grandfather,\\ whose lifelong passion for teaching others\\ inspired me to
    pursue my own goals and education.}\\
  \vspace*{3cm}
  \includegraphics[width=4cm]{frontmatter/dahlia.png}
\end{dedication}

%% BEGIN PAGESTYLE

%%% You can pick a pagestyle if you want; see the memoir class
%%% documentation for more info.  The default ``deposit'' option meets
%%% the UW thesis typesetting requirements but is probably
%%% unsatisfactory for making a version of your dissertation that
%%% won't be deposited to the graduate school (e.g. for web or a nice
%%% printed copy)

\chapterstyle{deposit}
\pagestyle{deposit}


% ACKNOWLEDGMENTS
\begin{acks}

I would like to thank all the people who have
supported me and my work over the years. In
particular, I would like to thank my family who
have shown me unwavering support and my friends
who have kept me sane. Also, I would like to thank
Ben Recht, Ronald Fedkiw, Matthew Cong for their
advice and discussions on several of the
contributions made in this
dissertation. Additionally, I wish to extend my
thanks to Dr. Timothy King, the medical residents
of the Plastic \& Reconstructive Surgery program,
and the School of Medicine and Public Health at
the University of Wisconsin-Madison for their
involvement and support with the virtual plastic
surgery platform, as well as acknowledge Aaron
Oliker and BioDigital Systems for supplying the
models used in this project.

This work was generously supported by the following grants
and funding sources: NSF grants IIS-1253598,
IIS-1407282, CCF-1423064, CCF-1533885,
CNS-1218432, CCF-1438992.


%%% Local Variables: 
%%% mode: latex
%%% TeX-master: "../dissertation"
%%% End: 

\end{acks}

% CONTENTS, TABLES, FIGURES
\renewcommand{\printtoctitle}[1]{\chapter*{#1}}
\renewcommand{\printloftitle}[1]{\chapter*{#1}}
\renewcommand{\printlottitle}[1]{\chapter*{#1}}

\renewcommand{\tocmark}{}
\renewcommand{\lofmark}{}
\renewcommand{\lotmark}{}

\renewcommand{\tocheadstart}{}
\renewcommand{\lofheadstart}{}
\renewcommand{\lotheadstart}{}

\renewcommand{\aftertoctitle}{}
\renewcommand{\afterloftitle}{}
\renewcommand{\afterlottitle}{}

\renewcommand{\cftchapterfont}{\normalfont} 
\renewcommand{\cftsectionfont}{\itshape} 
\renewcommand{\cftchapterpagefont}{\normalfont} 
\renewcommand{\cftchapterpresnum}{\bfseries} 
%\renewcommand{\cftchapterleader}{\cftparfillskip} 
%\renewcommand{\cftsectionleader}{\cftparfillskip} 
%\renewcommand{\cftchapterafterpnum}{\cftparfillskip} 
%\renewcommand{\cftsectionafterpnum}{\cftparfillskip} 

% \captionnamefont{\small\sffamily} 
% \captiontitlefont{\small\sffamily} 

\renewcommand{\cftdot}{}

% \renewcommand{\contentsname}{contents}
% \renewcommand{\listfigurename}{list of figures}
% \renewcommand{\listtablename}{list of tables}

\addtocontents{toc}{~\hfill\textbf{Page}\par}
\tableofcontents

\clearpage
\listoftables
\addtocontents{lot}{~\hfill\textbf{Page}\par}

\clearpage
\listoffigures
\addtocontents{lof}{~\hfill\textbf{Page}\par}

\clearpage
\listofalgorithms
\addtocontents{loa}{~\hfill\textbf{Page}\par}
\addcontentsline{toc}{chapter}{List of Algorithms}


\clearpage
% NOMENCLATURE
% \begin{conventions}
% % \begin{description}
% % \item{\makebox[0.75in][l]{term}
% %        \parbox[t]{5in}{definition\\}}
% % \end{description}
% \input{conventions}
% \end{conventions}

%% The UW graduate school no longer wants a UMI abstract page
%% Should you need one for some reason, uncomment the following
%% lines.  Thanks to Matt Fredrikson for reporting this!

% \advisorname{Gottlob Frege}
% \advisortitle{Professor}
% \begin{umiabstract}
%  
Techniques for simulating the behavior of elastic objects has matured
considerably over the last several decades, tackling diverse problems
from non-linear models for incompressibility to accurate
self-collisions. Alongside these contributions, advances in parallel
hardware design and algorithms have made simulation more efficient and
affordable than ever before. However, prior research has commonly made
the choice to sacrifice one aspect of simulation for another,
resulting in a fragmented landscape of solutions. For complex,
real-world tasks, such as virtual surgery, a holistic approach is
desirable, where complex behavior, performance, and ease of modeling
are supported equally. This dissertation aims to provide that in the
form of several interconnected investigations, each of which
contribute a piece of an unified solution. First, it will be
demonstrated how advanced non-linear materials can be combined with
lattice deformers yield simulations with behavioral richness and a
high potential for parallelism. This potential will be exploited to
show how a hybrid solver approach based on large macroblocks can
accelerate the convergence of these deformers. Extending the lattice
concept with non-manifold topology allows for efficient processing of
self-collisions and topology change. Finally, a case study on virtual
plastic surgery will demonstrate a real-world problem space where
these ideas can be combined to build an expressive authoring tool,
allowing surgeons to record procedures digitally for future reference
or education.


%%% Local Variables:
%%% mode: latex
%%% TeX-master: "../dissertation"
%%% End:

% \end{umiabstract}

\begin{abstract}
  
Techniques for simulating the behavior of elastic objects has matured
considerably over the last several decades, tackling diverse problems
from non-linear models for incompressibility to accurate
self-collisions. Alongside these contributions, advances in parallel
hardware design and algorithms have made simulation more efficient and
affordable than ever before. However, prior research has commonly made
the choice to sacrifice one aspect of simulation for another,
resulting in a fragmented landscape of solutions. For complex,
real-world tasks, such as virtual surgery, a holistic approach is
desirable, where complex behavior, performance, and ease of modeling
are supported equally. This dissertation aims to provide that in the
form of several interconnected investigations, each of which
contribute a piece of an unified solution. First, it will be
demonstrated how advanced non-linear materials can be combined with
lattice deformers yield simulations with behavioral richness and a
high potential for parallelism. This potential will be exploited to
show how a hybrid solver approach based on large macroblocks can
accelerate the convergence of these deformers. Extending the lattice
concept with non-manifold topology allows for efficient processing of
self-collisions and topology change. Finally, a case study on virtual
plastic surgery will demonstrate a real-world problem space where
these ideas can be combined to build an expressive authoring tool,
allowing surgeons to record procedures digitally for future reference
or education.


%%% Local Variables:
%%% mode: latex
%%% TeX-master: "../dissertation"
%%% End:

\end{abstract}

\clearpage\pagenumbering{arabic}

%%% END STUFF TAKEN FROM WITHESIS EXAMPLE FILE

%%% Local Variables:
%%% mode: latex
%%% TeX-master: "../dissertation"
%%% End:


\chapter{Introduction}

Mankind has been seeking better methods to capture the human body and
its pathologies for the purpose of healing since the earliest days of
modern surgery. From the anatomical drawings of De Vinci, to more
modern practices of constructing realistic simulacra, doctors, and
their students, have been chasing tools that allow them to practice
their skills before operating on real patients. Existing research
shows the benefits of engaging in these practice
sessions\cite{GallaRCHFMSS:2005}. Practiced surgeons make fewer
mistakes and can use preparation sessions to plan new approaches
safely.

This general philosophy, which can be summed up with the classic
proverb of ``measure twice, cut once'', is practiced by many high risk
professions. From flight school to driving simulators, computer
constructed virtual environments have become an integral part of
training people. The reasoning is three-fold: computer simulations are
relatively low cost and are able to be reset quickly, novel parameters
and situations can be introduced more easily than physical
environments, and a trainee's progress can be easily recorded for
later review. With these advantages over purely physical training
environments, why are surgeons still using aids such as diagrams,
physical mannequins, and cadavers?

The short answer is that they often lack any better
alternatives. Performing surgery is a complex task involving a
combination of dexterous and cognitive, often spatial reasoning,
skills\cite{GallaRCHFMSS:2005}. Tools that support all of these areas
are difficult to get right, and most attempts to build technological
aids have focused on subsets of the skills required. Historically,
these have been the dexterous skills, which many authors have tried to
solve with a variety of haptic simulation
techniques\cite{MendoL:2003,LindbT:2007}. While these surgical
simulation philosophies are useful, and have been used in commercial
products\cite{SUSAC:2002--2014}, they don't really meet the need of
training cognitive skills. This need varies across surgical
specialties - reconstructive plastic surgery, which is the focus of this document,
requires the surgeon to have internalized geometrical intuitions in
order to manipulate tissue in the three dimensional space of the human
body. It is this lack that has kept traditional, less technological
aides as the core of many plastic surgery training programs.

In contrast to internal surgery, plastic surgery suffers from the
practical reality that the results of any operation will be visible to
others. This fact puts added pressure on surgeons to not only complete
their tasks correctly, but also in an aesthetically pleasing way. It
follows then that a simulator for plastic surgery operations must
provide an environment for surgeons to freely practice design, as well
as correctly display the outcomes.

This dissertation aims to support the following statement: The creation
of simulators for craniofacial reconstructive plastic surgery is now
technically feasible. The craniofacial region, consisting of the scalp
and face, has been chosen for since many operations can be performed
by manipulating localized skin flaps. In pursuit of this goal, this document will
describe how existing standard practices for simulating elastic
materials can be combined in a holistic fashion, all with an eye
towards performance and practical usability. In the process,
limitations with the current approaches will be explored and, in some
cases, alternative techniques will be proposed to solve issues
stemming from the unique challenges found in plastic surgery simulators.


\chapter{Motivation}

In order to properly understand this dissertation, it is important to
place it the proper context. While the driving topic of this document
is simulation in the plastic surgery domain, the work presented here
in this dissertation derives from more fundamental motivations. The
goal of this chapter is to present a more complete motivation for the
design decisions and technical contributions presented throughout this
document.  In pursuit of this goal, this chapter will describe an
overarching framework for Practical Visual Interactive
Systems. Throughout, both high level design goals and challenges will
be discussed, along with specific examples which relate to plastic surgery.


\section{Practical Visual Interactive Systems}


At a high level, the work completed in support of this thesis falls under the description
of a \Gls{pvis}. A Practical Visual Interactive System \textit{is a dynamic
virtual environment where by user input produces changes in state for
the purpose of supporting some task}. These systems are actually
everywhere in our modern society, though we do not often think
of them in these terms. Examples include video games, virtual avatar
systems, training systems, and more. This section's goal is to define
what a \gls{pvis} is and is not. This deconstruction will be followed
by a examination of the challenges that exist when developing a
\gls{pvis} along with specific examples from plastic surgery
domain. Finally, a justification will be made, which will show that
\gls{pvis} development is warranted, despite the complexities involved.

\subsection{PVIS Deconstruction}

The core of a \gls{pvis} is a virtual world, or environment. The size
and scope of the environment is less important than the fact that the
environment \textit{is not real}. When discussing a \gls{pvis}, we are
talking about artificially constructed settings, often completely
described by a computer program and displayed via some
device. However, a \gls{pvis} is not completely disconnected from
reality. In order to support a user in a particular task, a \gls{pvis}
is often designed to mimic real life situations.

Returning to the definition above, is it appropriate to
declare any virtual environment a \gls{pvis}? Not exactly - a careful
reading of the proposed definition reveals that a \gls{pvis} is
structured according to a series of important properties, or \textit{aspects}.

\begin{description}
\item[Geometry] Most of the forms of \glspl{pvis} that people interact
  with today are displayed as 3D worlds. Full of rich and detailed 3D
  models, this environment type often is preferred when the goal is to
  recreate some aspect of actual reality. However, a \gls{pvis}'s
  environment should not be restricted to one in three dimensions. A
  designer should choose the environment's representation according to
  the utility they wish to extract from a \gls{pvis}. In addition to
  representation, it is also important to consider a \gls{pvis}'s
  interface. But whether a \gls{pvis} is designed around a standard
  desktop computer monitor, or a new virtual reality \gls{hmd}, it is
  important to prevent the choice of technology from overshadowing
  other aspects of the \gls{pvis}. The particular interface should be
  chosen with care in order to augment, not detract from, the utility of
  the system for its users.

\item[Reactivity] A \gls{pvis} should hold state and allow its
  users to change this state meaningfully via interactions with the system. This
  aspect critically enables many other useful properties of a
  \gls{pvis}. First, it brings an important property of determinism
  to the system. Instead of static or random reactions, users should
  expect their actions will produce understandable (yet not
  necessarily simple!) consequences. Second, this property defines the rules of
  the \gls{pvis}, which are important both to the designer and the
  user. The designer needs to understand the rules in order to make
  informed choices and compromises in development (more on this
  later). The user of the system needs the rules, even implicitly
  stated, to gain intuitions and learn how to use the system. If these
  rules faithfully replicate real world behaviors, the user may be
  able to translate knowledge learned from the \gls{pvis} into the
  physical world.
  
\item[Dynamism] Often conflated with the aspect of reactivity, dynamism
  brings in a temporal component to the \gls{pvis}. This moves the \gls{pvis} away
  from a what otherwise might be a simple state machine, whose states
  can be traversed in any order, to a system with a history. Whether
  this history property is permanent, where user input creates
  indelible changes to the virtual environment, or flexible, giving a
  user a timeline of consequences to move through at will, depends on the
  ultimate purpose of the \gls{pvis}. 

\item[Interactivity] While the aspects of reactivity and dynamism work
  together to create complex systems of internal state for a
  \gls{pvis}, they don't cover \textit{how} a user interacts with the
  system. While a movie could be construed as a very simple
  \gls{pvis}, whose state is the current frame and only dynamic
  behavior is to advance the frame, typically interacting with a
  \gls{pvis} is more semantically rich than a set of actions like
  play, pause and stop. More complex interactions can be found in
  advanced \glspl{pvis}, such as video games and training
  simulators. Here, interactions are designed to replicate the look
  and feel of real controls, or exist to immerse a user into the world
  by allowing them to embody a virtual character through movement or
  speech. These advanced modes serve to both increase the realism of
  the system and to give the user a more comfortable or familiar set
  of mental tools to think about the system with.

\item[Purpose] A \gls{pvis} without a purpose is not a \gls{pvis}. The
  \gls{pvis}'s purpose helps define what features are important and
  clarifies what metrics should be used to measure its ultimate
  success. All of the other aspects of a \gls{pvis} must work together
  to support this purpose.  However, the reasons one might want a
  \gls{pvis} are wide and varied. Video games have already been
  mentioned, whose purposes might be to entertain or tell a story. But
  \glspl{pvis} come in many forms, for many purposes. A virtual
  meeting place, where participants exist as avatars in a virtual
  environment is a \gls{pvis}, and the purpose is to facilitate
  communication between users. Professional disciplines also employ
  \glspl{pvis} for a variety of purposes, including training via
  simulators and prototyping new products via computer aided design
  tools.
      
\end{description}

\subsection{Case Study: Medical Simulation}

Now that we have a firmer idea of what a \acrlong{pvis} consists of,
let us explore a real world problem domain: plastic surgery. By
understanding this complex domain, we will be able to answer the
question that constantly follows any attempt to build a complex
system, such as a \gls{pvis}: Why go through the effort?  Perhaps
restated in concrete terms, what could warrant the thousands of
man-hours or years of research required to build such a piece of
software? Ultimately, the answer is found in the utility of the
end-result. The ability of the final artifact, enhanced by the first
four aspects, to produce practical value for its users is the reward.

However, the purpose of a \gls{pvis} is more than just a set of goal
posts to arrive at. For researchers and software engineers, a
practical purpose guides their design decisions and serves as
evaluation metrics for the final product: What is the real-world
usefulness? How well does the system accomplish the task it was designed for?
What are the fine grained aspects of the task? What features need to
implemented (or perhaps more interestingly, not implemented)? Who are
its users? All of these questions are answered by an external guide
and are not addressed by pure software design principles.

For the work described in subsequent chapters in this document, the
domain of plastic surgery will act as this external
guide. This document's goal is to explore systems and techniques which
support the cognitive challenges faced by reconstructive plastic
surgeons. Due to the complexity of this surgical specialty, the scope
of this work is limited to the more tractable sub-problem of local flap
techniques and craniofacial anatomical regions. A later section will
discuss the technical details and specific tasks required for this
domain, but for now let us look more closely at how the five aspects
of a \gls{pvis} intersect with this domain.

\paragraph{Geometry} Surgery in general, plastic surgery in
particular, are areas in medicine which depend strongly on a
practitioner's intuitive spatial reasoning processes. At a high level,
the task of a plastic surgeon is to manipulate the geometry of the
human body into a new configuration. The reasons for this style of
intervention are numerous and range from cosmetic procedures,
post-operation repairs, to fixing congenital
deformities. A surgeon must understand the geometry of their patient
in order to produce aesthetically pleasing outcomes. Since these
operations are often highly visible, failures or mistakes can be
extremely costly in an emotional and social sense for the patient. Any
\gls{pvis} designed for this domain must take these concerns seriously
and present a compelling visual representation.

The work presented in this document achieves this goal by describing a
tool for visually authoring plastic surgery operations in a three
dimensional virtual environment. However, it is worth taking time to consider
some potential alternatives. For instance, a two dimensional sketching
interface could be considered, drawing on the rich history of surgical
instructional diagrams that exist in the field. Certainly such
approaches have been used for decades in surgical textbooks, but they
fall short of giving the viewer a comprehensive perspective of the
inherent three dimensional nature of human anatomy. Alternatively, one
might consider using video of real operations, annotated with
information describing what is happening. While this idea certainly
leaves no question about the three dimensionality of the problem, it
does so by sacrificing the clarity a rendered computer model can
provide and the ability to alter the geometry easily.

A note should also be made on display technology. Here the word
``display'' should be treated loosely and should be taken to mean any
physical channel which conveys information from the \gls{pvis} to the
user. Commonly, this is restricted to visual interfaces, such as
computer monitors or \glspl{hmd}. However, it could also refer to
force feedback devices, where a user receives tactile responses from
their actions in the system. Ultimately, the proper device choice for
a surgical \gls{pvis} depends on its intended purpose. Many commercial
tools for training surgeons on laparoscopic equipment use a tightly
coupled visual and tactile feedback system. Here the goal is to
replicate operating room conditions exactly, hopefully instilling into
physicians the \gls{tacit} required\cite{TACITKNOWLEDGE}.

\paragraph{Reactivity \& Dynamism} During their education, a surgeon
builds internal intuitions about how the various tissues of the human
body. Primarily, they are interested in how the tissues, such as skin,
react to external forces: pulling and pushing. These biomaterials
behave complex and often non-intuitive fashions, which must be
internalized into a surgeon's mental model of an operation. When
considering a virtual representation of an operation, maintaining
these properties for a user that is expecting them from real-life
experience can be important. Incorporating these behaviors into a
\gls{pvis} typically falls under the aspects of reactivity and
dynamism. This document will show how these properties can be
generated by simulating elastic materials, but this is not the only
approach that could be taken.

To avoid the computational expense of physics simulation, one might
consider traditional animation. Here, a trained animator might work
closely with a surgical domain expert to develop three dimensional
animations. These animations would move in the correct fashion,
stretching and bulging where appropriate. However, they would behave
in a fixed fashion, perhaps embodying a series of finite states. The
primary benefits of this approach are flexibility and speed. The
creative team behind the animations can do what they want,
unrestricted by anything except their imagination and knowledge of
real-world behaviors. Additionally, once created, the animations can
be played back at high speed requiring, compared to simulation, few
computational resources. Unfortunately, this requires sacrificing the
ability to make dynamic changes and requires significant time
commitment from the artists and surgeons.

While traditional animation techniques can capture the reactive
qualities of human tissue, physical simulation better captures the
dynamic aspects, especially when reactivity and dynamism are mixed. An
example to consider are the effects of blood loss on tissue. Undamaged
human tissue is composed primarily of water, a material which is
largely imcompressible. Accordingly, human tissue can also be thought
of as effectively incompressible - until a surgeon cuts into it. As
blood leaves through the open wound, the water content of surrounding
tissue drops resulting in the tissue becoming more and more
compressible. This a \textit{dynamic} property of the
tissue. Depending on how quickly a surgeon works, the subsequent
manipulations may be on tissue that is more or less compressible
changing the way they \textit{react}. Physical simulation excels at
capturing this complex interplay between reactive behavior dynamic
properties, which can only be roughly approximated by prescribed animations.

\paragraph{Interactivity} Surgeons require impressive hand-eye
coordination; they carefully trained over many years on how to
interact with human tissue, both manually and via surgical
instruments. A substantial challenge, in developing a \gls{pvis}, is
how to map these complex modes of interaction to a computer system
while retaining usefulness. There are many approaches than can be
taken to reconcile this challenge, depending on the ultimate goals of
the application. If the goal is to provide the user with realistic
physical sensations, haptic feedback interfaces can be used. Many
existing surgical simulation tools in the industry and in academic
circles have used this technique. [CITE some examples] Alternatively,
it might be more important to replicate the exact environment a
surgeon might encounter. Often this utility is desired in cases where
the surgeon is learning unfamiliar equipment or needs training in
exotic interfaces, such as laparoscopic systems. [CITE simbiotix]


\paragraph{Utility} The final \gls{pvis} aspect to discuss in regards
to the plastic surgery domain is that of utility. This topic has been
touched upon in the previous sections on Geometry, Reactivity,
Dynamism, and Interactivity, but here the goal to explore in more
depth how a \gls{pvis} can be used to support the plastic surgery
profession. Surgical skills targeted by computer-based training
solutions have been classified \cite{GallaRCHFMSS:2005} in two major
categories. \emph{\Glspl{psymotor}} refer to the dexterous use of
the surgeon's hands to manipulate instruments in the course of an
operation. In plastic surgery, psychomotor training involves
mechanical aspects of surgical tasks, such as the ``feel''
of tissue being cut or the nuances of manipulating a scalpel to enact
a curved incision. For example, training for laparoscopic procedures
requires a clinician to be familiar with the tactile response of
pushing and pulling on organs and to practice coordination
skills required for suturing and cauterization.  A number of
computer-based solutions focus on psychomotor training
\cite{MendoL:2003,DeKLS:2005,KimCDS:2007,LindbT:2007}.

In contrast to psychomotor training, \emph{cognitive skills} and
training are largely mental rather than dexterous exercises. For
example, in the procedure shown in Figure \ref{fig:gridiron-teaser},
the surgeon needs to contemplate how to best repair a large square
skin defect (i.e. area of excised tissue) by making auxiliary
incisions that create properly shaped ``puzzle pieces'' which can be
sutured together without creating excessive stress. Chentanez et al.\!
\shortcite{ChentARCHGSO:2009} described a cognitive training system
for steerable needle insertion, where the mental challenge lies in
planning a sequence of actions involving needle flexion and torsion,
in order to achieve a desired insertion trajectory.

Focusing on one type of skill training or another affects the design
decisions of a proposed \gls{pvis}. For instance, building a
\gls{pvis} for psychomotor training likely places a greater emphasis
on the interactivity aspects of the system, potentially requiring
haptic feedback mechanisms to provide tactile sensations. Likewise, a
cognitive training aid might require more involved reactivity and
dynamism, such as a more accurate representation of tissue
behaviors. While it might be tempting to say that all aspects, both
psychomotor and cognitive, should be supported equally, it is
important to remember that a \gls{pvis} is not reality. These systems
are fundamentally an approximation of the real world and are subject
to compromises. The next section will explore some of these challenges
in more detail. 

\subsection{PVIS Challenges}

Designing a Practical Visual Interactive System comes with a wide
array of challenges, which is not especially surprising when
considering how many components go into their makeup. In this section,
these challenges will be reviewed, hopefully to give better context
for the design decisions that were made for the rest the work
presented in this document.

\subsubsection{No One Size Fits All?}

During the process of software design, the goal of re-useability is
often discussed. Sometimes designers are referring to reusing
techniques or the software modules, but the theory is more reusable
artifacts are generally more useful. Designing a \gls{pvis}, however,
imposes some interesting roadblocks for the principle of
re-usability. The first issue that often comes up is that the
requirements for a \gls{pvis}, while they can appear similar on the
surface (e.g. display a 3D environment, respond to user input, etc.),
are often implemented with specific optimizations due to the tight
restrictions placed on such systems (e.g. near real-time performance,
extremely complex environments, etc.). Developers faced with these
issues can easily fall into the trap of \textit{blind
  optimization}. This is a form of anti-pattern, where they optimize
the implementation, often quite expertly, for some aspect but without
considering the rest of the system, or later re-usability It is
important to distinguish this from \textit{premature optimization},
where the developer spends time optimizing an implementation before
knowing if such effort is required. Blind optimization is performed
under local justification. It is only with a broader context that it
can be determined to be a wise course of action. Premature
optimization may end up being wasted effort at best, detrimental at
worst.

Designing general purpose \gls{pvis} platforms is not impossible. Game
developers, over a long developmental history, have created many
excellent general platforms for game development, referred to as game
engines. But increasingly, these platforms, and the developers who
design them, are becoming a field in their own right.  In the past, a
game developer might have done everything from writing low level
graphics code to higher level game logic. In contrast, modern games
are often written by developers who know little about the low level
optimizations required to reach the fidelity and performance expected
by current audiences. Instead, these skills are expressed as game
engines - highly tuned, carefully optimized systems which are not a
game per se, but act as a solid foundation for games written on top of
them. They provide \textit{services}: rendering, resource management,
network support, user interface toolkits, and much more. In essence,
this divide is not dissimilar to that of applications and operating
systems.

The work completed in this document generally follows this
philosophy. As will be described in later sections, the systems
presented in this work adhere to two general principles.

\begin{enumerate}
\item \textbf{Avoid Uncalled for Optimization} During implementation,
  it was important to avoid optimizing too soon. In many cases, there
  were open questions (many still remain!) that premature optimization
  could have prevented a better understanding. Only when it was clear
  that optimization was needed, were further steps taken, and taken in
  complete understanding of their consequences on the rest of the
  system.

\item \textbf{Enforce Clean Separation} As will be discussed in more
  detail later, the domain specific motivation, plastic surgery, were
  separated as much as possible from the underlying enabling
  technologies. This allowed a coupled, but functionally isolated,
  system design, where the components needed to build a plastic
  surgery simulation were isolated from the components needed to build
  a high performance finite element simulator.
  
\end{enumerate}


\subsubsection{Realism and Believability}

For the visual and virtual world aspect of a \gls{pvis}, one primary
question that arises is one of  believability or realism. These
concepts are often conflated, though they really should be considered
separately. I view realism as a more direct measure of how much a user
of the system sees what is being presented as an accurate simulacra of
a real object or environment. In contrast, believability is the
measure of how much a user trusts, or is willing to accept, the
information being presented. Lets use modern special effects in
movies, for an example. On one hand, special effects can be used to
introduce elements that have direct real world counterparts, which 
are either too expensive or impractical to use. Examples might include
simulating a full ocean, using a green screen to replace environments,
or full simulation of real people. If done well, these techniques
increase realism, by convincing the audience that the elements being
fabricated really exist. On the other hand, special effects can
introduce clearly fantastical elements: mythical creatures, exotic
environments, or magical effects. These are elements that even the
least critical audience member would not hesitate to say are fake. But
the goal here is not to trick them into believing something is real,
but to convince them that its \textit{plausible}. That the fire breathing
dragon, if it were to really exist, would look just like it does on
the screen. This is the believability aspect at work. Of course, there
is no reason that realism and believability can not work
together. Examples include live action heroic characters being
replaced by animated versions in order to have them perform impossible
stunts. Here the result must be have realism (it looks like the real
person), but also be believable (the impossible stunt looks like it
could have been pulled off).

So what does this mean for \gls{pvis} designs in general? And for medical
simulation specifically? The primary issue with a \gls{pvis} is that of user
input. While a movie can be scripted and refined until realism and
believability is exactly where the actors and artists want it to be, a
\gls{pvis} must produce similar levels of fidelity when faced with arbitrary
user actions. Dealing with this challenge often requires domain
specific knowledge in order to limit the potential space of user
interactions. For example, in a surgery simulation, the user can't do
absolutely anything to a section of simulated tissue. They are forced,
by the context of the situation, to interact with it using the
simulated tools provided: scalpels, sutures, etc... This allows a
surgery \gls{pvis} to be engineered to properly handle all the potential
outcomes of this restricted interface, to better produce realistic and
believable results.

\subsubsection{Design Conflicts}

A major problem facing the construction of any \gls{pvis} is that of design
goal conflicts. By this I am referring to the all too common situation
where supporting one feature or aspect of a system, such as
performance, comes into direct conflict with implementing another
feature. Continuing with the performance example, suppose we wanted to
impose a strict requirement on visual updates to the user. By doing
so, we have restricted our updates with an upper bound on the maximum
amount of work they can complete at any one time due to time
restrictions. This choice may bias further choices towards the use of
other techniques, not because of any technical merit, but of
complexity.

Many of these conflicting goals exist, some well known in general
software design circles. In the context of physical simulation, there
are several conflicts we need to be especially aware of.


\paragraph{Reactivity Vs. Accuracy}
I touched on general reactivity before when talking about what defines
a \gls{pvis}. For simulation, we can use a more precise definition where the
reactivity of a system refers to the time between a user applies some
change to a simulated system (a force impulse, a constraint change,
etc...) and when the user sees the result of the action. This cause
and effect timing is the reactivity of the system. The smaller this
time, the more reactive the simulation \textit{feels}. We can see this
when comparing the simulation to real materials. For real objects, the
reactivity is effectively infinite, as the time between cause and
effect is extremely close to zero.

Of course, real materials have an advantage that simulated materials
do not. Because they are composed of individual atoms, real objects
effectively act as a perfect finite element simulation, where the
elements are almost infinitely small and operate completely in
parallel to each other. Computer simulated materials are much coarser
in their resolution and, despite great advances in parallel
processing, do not come close to that naturally available in real
materials. Thus, as we increase a simulated object's resolution in
order to capture more and more detail, or use more complex elements
that capture more interesting macroscopic effects, the overall
reactivity of the simulation decreases as more effort is spent
resolving each user action.

The challenge is to find the appropriate balance between the desired
reactivity of a simulation and the accuracy of the simulation. A major
focus of my work has been to explore how both of these aspects can be
increased simultaneously, both by exploiting underutilized parallelism
opportunities and by looking at novel data structures to extract
additional effective resolution without significantly doing so.
  
\paragraph{Domain Utility Vs. Generality}

Another two aspects that often find themselves in conflict for
physical simulation systems are the concepts of domain utility and
generality. Lets look at domain utility first, as its the more
straightforward of the two. In the simplest terms, domain utility
refers to making design choices in a system that primary serve the
specific task, or domain, that it is currently being built for. This
may refer to choosing or discarding certain features, deciding what
API best suits the current task, or making optimization along critical
paths for the client application. All of these choices can be
reasonable, even correct, as long as you never intend to reuse the
system for any other purpose.

Generality, on the other hand, asks what is the commonality of
different tasks and guides design choices along this route. A general
design should be flexible to different and changing requirements. Such
a system typically eschews APIs built for specific tasks and instead
tries to distill out the fundamental building blocks that any
potential client may need from the system. The difficulties with this
philosophy are two-fold. First, it isn't always obvious what the
fundamental interfaces are, partially because designers by necessity
must look at past applications to define them and are ignorant about
future ones. But secondly, general designs often cannot make
simplifying assumptions that domain specific knowledge provides and
leaves them with overly complex code that tries to optimize for every
use case or simpler code which doesn't optimize anything.

So why would anyone design a general system? On the surface, they seem
harder to build effectively and often don't result in well optimized
solutions, impacting other aspects such as reactivity. The short
answer is flexibility. For a well studied domain, where every last
detail is known and accounted for, a specialized system is probably
the best choice. But in order to answer challenging research
questions, tools and problems often have to change quickly and in
unexpected ways as researchers adapt to new findings and explore new
directions. Medical simulation is very much one of these areas, where
new questions are constantly arising and old preconceptions are
abandoned. As such, I have made considerable effort to building
generalized simulation systems, and attempting to identify which areas
are ready for optimization and which are not.

 
\subsection{Why Develop a PVIS?}

At this point, a concerned reader may be asking why one should go
through all the trouble of developing these types of systems. In the
previous sections, I've described a complex series of requirements for
\gls{pvis} construction, each challenging in isolation, let alone
combined. On top of these technical features, I've also laid out
various higher level issues, such as problems with generality, user
acceptance, and correctness. Despite all of these problems, I maintain
that developing \gls{pvis} style platforms is not only doable, but
ultimately desirable. In order to understand my position, let me
describe the three avenunes by which a \gls{pvis} creates value:
As a Catalyst, By Filling a Need, and Intrinsically.  

\subsubsection{Catalyst for Advances}

As I've covered before, a reasoning about a \gls{pvis} is a complex task,
implementing one is more so. In going about this process however, we
have the potential to learn a lot. Anytime we have to adapt a \gls{pvis} to
a new domain or integrate new functionality, we will ultimately
generate questions and hopefully new answers. In this way, \gls{pvis}
implementations are a generator for new research. Whether it is
answering questions about rendering, human-computer interaction,
systems, software design, optimization, or in the case of this thesis,
physical simulation, a \gls{pvis} acts as a fertile soil within which we, as
researchers, are able to experiment in many areas. But more than
simply providing a platform to test isolated ideas, a \gls{pvis} is by its
nature integrated. Any change affects and is affected by everything
else, forcing researchers to take in and understand the big picture
around their work and where it fits into the whole.

\subsubsection{Utility Gap}

The second reason that building a \gls{pvis} is often worthwhile is to fill
a need. As I stated, a \gls{pvis} is, at its core, software with a
purpose. In some cases, such as game development, many implementations
of a \gls{pvis} have already been created, making the bar much higher as to
the need for another one. But in other areas, such as medical
simulation, the gaps in functionality coverage are more severe. To use
the \gls{pvis} described by this document as an example, there have been
many projects developing systems for laparoscopic organ surgery, for
instance, but few systems for performing simulated plastic surgery,
let alone a fully featured \gls{pvis} with a rich set of
capabilities. Filling this gap with the utility provided by a \gls{pvis} is
then extremely valuable, as without it practitioners are left behind in
a world where their colleagues are more and more enjoying the benefits
of modern computing technology.

\subsubsection{Intrinsic Value}

At a basic level, a \gls{pvis} itself is valuable. Even if we ignore the
value of a \gls{pvis} in fulfilling its specified purpose, or the additional
research that can be spawned as a result of its construction, building
the \gls{pvis} is beneficial to its developers and the greater
community. For the former, implementing a \gls{pvis} requires time,
dedication, and skill - but no one enters and leaves such a project
unchanged. Simply being a developer on a \gls{pvis} helps a developer become
a better software engineer, simply through the long hours of practice
they will spend on it. Beyond individual developers, building a \gls{pvis}
is important to the community at large for the simple reason that it
demonstrates that such a project can be done. Like all large pieces of
software, sometimes the most important idea they can convey is that such a project is
even feasible at all.

\section{Software Design}

To start, let us examine the first facet, Software Design. Whole books have
been written about this topic, many of which are full of valuable and
insightful ideas. Its not my goal to claim any particular innovation
or new vision for software design. Instead, I merely wish to lay clear
how software design has served as a guiding focus during my
research. All too commonly, research software is written quickly and,
while not nessessarily shoddily, perhaps without careful attention to
choices made that will later effect reusablity, maintainability, or
debugging. While these issues affect any major software project, my
experience has shown that the stresses associated with research often
exhasbertate them. Strict deadlines and a constant push for new ideas
can easily result in poor design choices. For these reasons, I have
tried to instill good software design ideas into my work.

What exactly is good software design? And more revelently, what is
good design in the context of simulation software? I'll be touching on
these questions more fully when I talk about the specific systems
I implemented over the course of the research described in this
document. In short, however, I think the primary goal of good design
for simulation software should be to support the concept of
Extensiblity Flexibility. What do I mean by that? In plain english, I
mean that simulation software should be designed such that is it easy
to add new features or use the software for new purposes without
requiring large amounts of rewriting. This is especially important for
research code, as required features are often unknown, or even
unknowable when development begins on a project. Being able to
continously reuse existing proven code as much as possible between projects
allows researchers to develop quicker, hopefully with less bugs.



\chapter{Engineering Deconstruction}

Given the design challenge of creating a \acrlong{pvis} for plastic
surgery training, the next question is by what methods can best
accomplish this task. Commerically, there exist examples of surgical
simulation products based on both procedural animations and physical
simulations. As discussed in the previous chapter, a detailed
procedural animation can meet many of the goals for a surgical
training \gls{pvis}. The choice of these methods can by provide realistic geometry,
respond to user inputs by advancing down pre-scripted event chains,
and serve as a functional replacement for traditional illustrated
procedure guides. However, they fall short when assisting in cognitive training
tasks.

In these scenarios, the surgeon undergoing training is expected to be
developing a mental intuition for how operations work at a fundamental
level and be capable of adapting this knowledge to new
situations. These requirements go beyond a simple recall of the
``correct'' approach and instead requires surgeons to understand the
behaviors of tissue at a intuitive level. Traditionally, this level of
understanding is gained through years of experience working with live
patients, or animal models. Through this process, surgeons experiment
with designs (with experienced mentors guiding them) and observe the
results in order to better understand what works and what does not. It
is the support of this intellecual freedom, the ability to deviate
from diagrams in textbooks, that is important for a plastic surgery
\gls{pvis} to capture. To address these concerns, we turn away from
procedural animations to physical simulation. Physical simulation can
help address this problem by providing correct, real-time responses to
a user's actions, regardless of what actions they take.

The remaining chapters in this document will discuss technical
contributions that will help in building a plastic surgery training
\gls{pvis} with physical simulation, but the goal of this chapter is
to deconstruct the basic engineering challenges found in this
space. This will help frame the work presented in subsequent chapters,
which can otherwise feel somewhat disconnected from the topics of
plastic surgery or \gls{pvis} design in general.

% PVIS, Virtual Surgery, Simulation

% How do we get to Simulation from PVIS/Virtual Surgery? Why not just do
% animations? Well, they are not a PVIS according to the definition
% defined earlier - No real connection between
% reactivity/dynamism/interactivity.

% Instead, we are going to consider building our virtual surgery pvis
% around a core of functionality consisting of elastic body
% simulation. Why? Allows us to represent the virtual tissue with
% realistic geometry. Maintains a state representing the behavior of
% real tissue. Can change over time imitating real time-varying
% responces of real tissue. Can appropriately react to user input
% through conceptual physical inputs. Can be built with accurate models
% to help support real world surgical tasks.

% Lets break this down then

\section{The Goal}

To start, let us bring the topic of discussion down from the higher
level thoery discussed in the the previous chapter and provide a
concrete goal for what we want to provide. At the core, our proposed
surgical \gls{pvis} can be described as a system which provides three
dimensional models of tissue a reactive mechnical response to previously
unscripted user input. Let us break this down further and look at each
of these parts in more detail. 

First, what does it mean to have a three dimensional tissue model with
mechanical responces?  Starting from the real world, it can be said
that tissues are fundamentally collections of cells, or more
basically, atoms. Considering the collection as a whole, these small
elements give rise to the tissue's shape and form. We will return to
this idea of small elements later in the chapter, but for now let us
reason about these materials as if they were continuous entities. We
do this for the convenience of reasoning about them in a mathematical
sense, for ultimately all of the complex physical behavoirs we expect
from soft, squishy materials can be described by equations for stress
and strain.

To begin, we can consider an abstraction: a model of
tissue which consists of a closed boundary in three dimensional
coordinate system and the region which it encloses. We will be
refering to this region of space as the object's domain, represented
by the symbol $\Omega$. Later, we will talk about how to represent
this domain discretely on a computer, but for now simply treat it as a
prescribed region.

The next challange is to describe what we mean by mechanical
response. Mechanical response of soft bodies, such as human tissue,
can be described as a series of relationships between shape, forces,
and energy. We will be using these relationships to convert between
one aspect and another - in other words, produce new shapes from
considering the forces applied to an object. To explain this idea, let
us first consider a hypothetical elastic object, such as a block of
rubber. If one was to squeeze the block from two opposite sides, it
would deform, or change shape. And if the block is released, the shape
returns to the original configuration. What is happening in this
situtation is the physical manifestation of the relationship between
the applied forces, the block's internal energy, and its shape. In
particular, we are primarily interested in a class of materials known
as \gls{hyperelastic} materials. These materials are idealized as they
do not consider the object's prior deformation history when we compute
their current state and depend only on the current shape of the
object. The alternative would be some degree of
\gls{plasticity}. Plasticity effects are incredibly common in real
life, from the persistent wrinkles in uncrumpled paper to the fact
that metal holds its shape after being bent. Despite this, we can
avoid introducing the complexity of these effects by noting that human
tissue, which exhibits plastic effects over long periods of time, is
effectively hyperelastic during the time frame of a typical surgical
operation. However, none of this really describes how these materials
work at a mathematical level. To do so, we return to our high school
physics lectures.

\section{Equations for Deformation}

If we recall Newton's Third Law of Motion, external forces applied to
objects are met with an equal and opposite force. But if we look at
the object wholistically, we would see that these opposing forces
don't just exist on the surface, but are distributed interally
throughout the object's material. These internal forces are referred
to as mechanical stress. Stress is defined to be force applied over an
cross-sectional area, or volume for three dimensional objects. If we
were to look at a single point of an object, we could define the
internal stress of that point by a second dimensional tensor, which
describes what force vectors are felt by that point in three
orthoginal directions. Ultimately, this internal stress field causes
the object to deform, or change shape, resulting in strain - the
mechanical term for a related tensor value which is defined as the
ratio of changed length (or volume) compared to the orginal
length. The strain tensor, in combination with the specific material
being deformed, represents a quantity of potential energy. This energy
represents the amount of work required to deform the object.  Given
these basic mechanical deformation concepts, we can now develop a
theoretical framework for simulating deformable objects.

The first part of that process is describing our representation of
state. To do so, we return to the object's shape. The deformation of
an object is the change in shape between one configuration and
another. In order to keep track of the deformation, we can record it
relative to a object's reference configuration. This is an arbitrarily
choosen shape in the coordinate system of the object from which all
other deformations are measured. Mathematically, we can define this as
a mapping between the domain of the object and $\mathbf{R}^3$:
$$\phi: \Omega \rightarrow \mathbf{R}^3$$. Under this regime, we
consider the domain of $phi$ to be points in $\Omega$, often refered
to as material locations as they can conceptually be considered
infentesimal blobs of deformable material. We simultously locate and
identify them by a spatial vector: $\vec{X} = (X, Y, Z)$. The points
in $R^3$ which make up the image of $\phi$ are the cooresponding
deformed locations $x$. It is worth noting that the reverse mapping
could have been considered, i.e the mapping that projects from a shape
back to the reference configuration. This type of relationship is
not unusual in computer graphics, for instance via mesh
parameterization for texture mapping. Unfortunately, this relationship
is a little harder to deal with, espically if we wish to handle
situations where the mapping is not one-to-one, such as when the
deformation causes the object to overlap with itself. For the
techniques presented in this document, we will be using a reference
configuration that is non-self penetrating and a forward deformation
map to avoid these problems.

Now that we can define what it means for an object to be deformed, we
can return to the problem of the stress, strain, and energy
relationships. To do so, we'll be using the concept known as a
deformation map. As mentioned previously, the strain an object has
under deformation is spatially varying tensor field which describes
the ratio of stretch or compression. In order to access this information,
we can take the derivative of the deformation map $\phi$:

\begin{equation}
  \label{equ:deformationgradient}
  \mathbf{F}(\vec{X}) = \frac{\partial \phi(
      \vec{X} )}{ \vec{X} }
\end{equation}

We call this function $F$ the deformation gradient and it is almost,
but not quite equivient to the strain. The reason is that the object's
strain should be invarient to rigid body transformations,
i.e. translations and rotations. However, it is still reasonable to
use this value as a basis for computing the cooresponding strain
energy, the potential energy captured by the deformation. To do so, we
use another relationship called a strain measure. The strain measure
attempts to provide a quantiative measure of the amount of
deformation. To put it another way, the strain measure acts as a
filter - it enhances aspects of the deformation gradient that the
object should respond more strongly to (e.g. sheer) while reducing the
effect of those it should be invarient to (e.g. rotation). There exist
several popular strain measures which vary in computational complexity
and in accuracy. A common and easy strain measure is the the small
strain measure $\epsilon$, which has the following form:

\begin{equation}
  \label{equ:smallstrain}
  \epsilon = \frac 1 2 (\mathbf{F} + \mathbf{F}^T) - \mathbf{I}
\end{equation}

While it has nice properties, by acting as a linear
relationship between an object's deformation and its strain energy, the small
strain maeasure suffers from not being invarient to rotations,
making it typically unsuitable for large deformations. In constrast,
the Green strain $\mathbf{E}$ is more robust under such conditions,
but gives up the linearity property as a result:

\begin{equation}
  \label{equ:greenstrain}
  \mathbf{E} = \frac 1 2 (\mathbf{F}^T\mathbf{F}-\mathbf{I})
\end{equation}
  
Once we have choosen the strain measure for our material, we can use
it to derive the material's energy density function
$\Psi(\mathbf{F})$. This is a measure of the potential energy due to
deformation per unit volume of the material. If we integrate this
function over the entire domain $\Omega$ of the object, we can compute
the total energy contained in the object as a result of the
deformation. The energy density function is critical in describing a
material, as all other equations describing its state can be derived
from it. Similar to the strain measure, there are several popular
energy density functions which can describe generic deformable
materials. Linear elasticity, Equation \ref{equ:linearelasticity},
which derives from the small strain tensor, is easy and uncomplicated
to compute due to its linear relationship to the deformation
gradient. However, since it is based on the small strain tensor, it is
not rotationally invarient and can exhibit phyiscally inaccurate
behaviors under large deformations.

\begin{equation}
  \label{equ:linearelasticity}
  \Psi(\mathbf{F}) = \mu \epsilon : \epsilon \frac \lambda 2 \text{tr}^2(\epsilon)
\end{equation}

However, just as with the small strain tensor and the Green strain
tensor, there exist more accurate, if non-linear, energy density
functions. If we simply replace the strain tensor used in the Linear
elasticity model with the Green strain tensor, we get what is known as
the St. Venant-Kirchhoff elasticity function (Equation
\ref{equ:stvenantkirchhoff}).
\begin{equation}
  \label{equ:stvenantkirchhoff}
  \Psi(\mathbf{F}) = \mu \mathbf{E} : \mathbf{E} \frac \lambda 2
  \text{tr}^2(\mathbf E)
\end{equation}
Unfortunately, this description of elastic energy is highly nonlinear
and has the unfortunate side effect of reducing resistence to
compression once the material has been compressed past a certain
threshhold. In the film production industry, another nonlinear energy
function is often used instead, called Corotated elasticity. This
model is based off a new strain measure created from a polar
decomposition of the deformation gradient,
\begin{gather*}
\mathbf F = \mathbf R\mathbf S\\
\epsilon_c = \mathbf S - \mathbf I
\end{gather*}

resulting in a energy density function which maintains the rotational
invarience properties ideal for large deformations, but remains
similar enough to a linear relationship to be more predictable under deformation. 
\begin{equation}
  \label{equ:corotatedelasticity}
   \Psi(\mathbf{F}) = \mu \epsilon_c : \epsilon_c \frac \lambda 2
   \text{tr}^2(\epsilon_c) = \mu \lVert S - I \rVert^2_F \frac \lambda 2
   \text{tr}^2(S-I)
\end{equation}

From the energy density expression, it becomes a relatively simple
matter to derive the energy for the entire deformed domain $\Omega$ by simply
integrating the density:

\begin{equation}
  \label{equ:systemenergy}
  E(\phi) = \int_\Omega \Psi( \mathbf F ) \,d\vec{X}
\end{equation}

Once we have a description of the material's energy, we can derive a
secondary expression for the stress. Remember, similar to strain, the
stress tensor is a evaluation of the force a material experiences at
an infintesimal volumetric region. In this document, we will be mostly
concerned with one type of stress tensor, the 1st Piola-Kirchhoff
stress tensor $\mathbf{P}$, which is a second order, $3 \times 3$
tensor. Fortunately, for the class of materials we are interested in
here, hyperelastic materials, the defination of the stress tensor is
straightforward:
\begin{equation}
  \label{equ:piolakirchhoffstresstensor}
  \mathbf{P}(\mathbf F) = \frac{ \partial \Psi(\mathbf F)}{\partial
    \mathbf F}
\end{equation}

Finally, we can compute the force density (force per unit volume) at a
material point, by taking the divergence of $\mathbf P$:

\begin{equation}
  \label{equ:forcedensity}
  \vec{\mathit f}(\vec X) =
  \text{\textbf{div}}_{\vec X}\mathbf P (\vec X)
\end{equation}

\section{Solving for Mechanical Response}

With these equations in hand, we can now reason about how to produce
realistic mechanical responses from materials - the question we opened
this chapter with. At this point, it is time to also return to the
initial description of our material as a composition of small
elements. Of course, in our case, the small elements are not the same
as cells in human tissue, but they do represent finite quanta of
material, thus giving rise to the name of the approach we will be
using throughout this document: \gls{fem}. We now move away from the
continous world we were previously occuping into a discrete one more
suitable for computation on computers.

In our new discrete world, we must first translate our former
continous domain $\Omega$ into a data structure representable on a
computer. Here we have a number of choices to pick from. \gls{fem}
literature has described many approaches for descrization over the
years, ranging from volumetric meshes to particle based
approaches. These designs can be evaluated over several categories:
regularity, conformity, and ease of use. Regularity refers to the
extent that the technique uses repeating data structures or provides
implicit internal relationships that can be predicted, which is often
extremely benificial for performance. Conformity refers to how well
the data structure represents the object's boundary shape. Finally,
ease of use is a catch all term that includes any quality which makes
the data structure easy or troublesome to include in higher level
pipelines. In this document, the approach we will be using is a mesh
discretization. In particular, we will be using a design referred to
as an embedding lattice. This data structure is extremely regular and
generally easy to construct, though it can suffer from a lack of
conformity. 

With this representation, our material will be sampled at cartesian
nodes of a regular hexadedral lattice and our elements of computation
will be its cells. We refer to these nodes as degrees of freedom, as
they represent the discrete points where deformation can occur. To
deal with the issue of conformity, we can utilize a technique known as
lattice embedding - where the discrete vertices of our object's
surface, typically represented by another mesh, are embedded in cells
of the lattice via a weighted interpolation of its nodes. Later in
this document, additional techniques will be presented to further
increase the accuracy of the lattice's surface conformity. Moreover,
since the terms presented in regards to embedding can be confusing, we
will stick to the following terminology:

\begin{description}
\item[lattice] - The simulation lattice is the discrete representation
  of our simulation domain and is composed of cells of material known
  as elements. Through the simulation lattice, we compute strains,
  stresses, and ultimately forces.
\item[node] - A node is a topological point in our simulation lattice,
  also called a degree of freedom, or DOF. It records values of
  deformation and the resulting forces.
\item[cell] - A cell is the basic computational unit of our simulation
  lattice. It is represented as a regular hexadedral volume and
  connects the eight nodes at its corners.
\item[mesh] - Our material surface is represented by a mesh. This mesh
  is conforming, i.e it attempts to match the shape of the continuous
  domain as closely as possible. The deformed material surface is the
  desired visual result of our deformation simulation.
\item[vertex] - A vertex is a topological point in our material
  mesh. While it is deformed by the simulation lattice, its position
  in space is wholely determined by its embedding relationship,
  typically bilinear, to its parent cell.
\end{description}

Using this simulation lattice as a discrete approximation of our
continuous domain, we can now translate the concepts from the previous
section. Instead of material points, we have nodes. Each node is
identified by an identifier. This identifier can be as simple as an
integer, e.g. Node 52, but since we are using a regular lattice, we
will continue with the previous spatial labeling, although in this case we
will now use integer coordinates. Our former concept of a reference
configuration will remain, by assigning each node a reference position
in space. Defining our deformation map $\phi$ then becomes as
simple as:
\begin{equation}
  \label{equ:discretedeformationmap}
  \phi(\vec{X}:\mathbf x) = \sum_i x_i \mathcal N_i(\vec{X})
\end{equation}

Where we define $\mathbf x = {x_1, x_2, x_3, \ldots, x_n}$ to be our
deformation state, which includes the displacement of every node in
our lattice from its reference location. In this expression
$\mathcal N_i$ is the \textit{shape function} for each node in the
cell. This shape function corresponds to the interpolation method we
will employ, example trilinear or tricubic interpolation. Using this
new description of $\phi$, we can derive a continuous defintion of the
deformation map for any point $\vec{X}$ inside the cell just as
before:

\begin{equation}
  \label{equ:discretedeformationgradient}
  \mathbf F(\vec{X}:\mathbf x) = \frac{ \partial \phi(\vec{X}:\mathbf x)
  }{\partial \vec{X}} + \mathbf I
\end{equation}

In this expression for the deformation gradient, we include an
identity term which compensates for the fact that our deformation
state $\mathbf x$ is a record of relative displacements and not
absolute deformed positions.  From this expression, we can sum the
energy contributions from each discrete cell $e$. This results in an
energy expression for the discrete state of the object:

\begin{equation}
  \label{equ:discreteenergy}
  E(\mathbf x) = \sum_e E^e(\mathbf x) = \sum_e \int_{\Omega_{e}}\Psi^e( \mathbf F(\vec{X}:\mathbf x) ) \,d \vec{X}
\end{equation}

From these expressions, we can derive an expression for forces at each
node by simply taking the derivative of the systems energy at each
cell and summing its contribution onto the node:

\begin{equation}
  \label{equ:discreteforces}
  \vec{f}_i(\mathbf x) = \sum_{\mathcal N_i} \vec{f}_i^e(\mathbf x)
  \text{, where } \vec{f}_i^e = \frac{\partial E^e(\mathbf x)}{\partial \mathbf x}
\end{equation}

Using this relationship between nodal displacements and nodal forces,
we can finally describe the process for computing deformations.

For the purposes of this discussion, we'll be limiting our focus to
quasistatic deformation simulation sequences. Quastistatic deformations are the
result of the system being solved for internal equilibrium while we
change external positioning constraints. Generally, this idea boils
down to solving for:

$$
\mathbf f(\mathbf x) = 0
$$
Where
$\mathbf f = {\vec{f}_1, \vec{f}_2, \vec{f}_3, \ldots, \vec{f}_n}$ is
our force state, similar to $\mathbf x$ as our deformation state.
Since the relationship between forces and deformation is generally
non-linear, unless we are using a linear elasticity model for small
deformation situations, we can use a standard Newton-Rhapson iteration
approach to solve for $\mathbf x$ such that $\mathbf f(\mathbf x) =
0$. At each $k$ steps of this iteration, we need to compute a linear
approximation of our force-deformation relationship around our current
solution:

\begin{equation}
  \mathbf f(\mathbf x_k + d\mathbf x) \approx \mathbf f(\mathbf x_k) +
  \frac{\partial \mathbf{f}}{\partial \mathbf x}\biggm|_k\,d\mathbf x
\end{equation}

In this expression, we compute the force derivative around the state
at iteration $k$. Since we are interested in having our update
$d\mathbf x$ make our forces zero, we can derive the Newton-Rhapson
update expression:

\begin{equation}
  \label{equ:newtonstep}
  \underbrace{- \left(\frac{\partial \mathbf{f}}{\partial \mathbf
      x}\biggm|_k\right)}_{\mathbf K(\mathbf x_k)}
\,\underbrace{\vphantom{- \left(\frac{\partial \mathbf{f}}{\partial \mathbf
      x}\biggm|_k\right)}d\mathbf x}_{d\mathbf x_k}
  = \underbrace{\vphantom{- \left(\frac{\partial \mathbf{f}}{\partial \mathbf
      x}\biggm|_k\right)}\mathbf f(\mathbf x_k)}_{\mathbf f_k}
\end{equation}

We refer to the matrix $\mathbf K$ as the \textit{stiffness
  matrix}. In order to arrive at a solved equibrium configuration, for
each iteration $k$ our task becomes the following steps:

\begin{enumerate}
  \item Update the stiffness matrix $\mathbf K$ for the current
    configuration $\mathbf x_k$.
  \item Compute forces $\mathbf f_k$ corresponding to $\mathbf x_k$
    for our right hand side of the update expression. \textit{Check to
      see if we can terminate.}
  \item Solve $\mathbf K$ for a displacement update $d\mathbf x$.
  \item Update our current state $\mathbf x_{k+1} = \mathbf x_k +
    d\mathbf x$
  \end{enumerate}

  At the end of this chapeter, we'll look at the emergent challenges
  from this design, but for now it is important to note that the
  solution to the stiffness matrix inverse is the major computational
  expense during this process. To see why, we should take a brief look
  at its construction. At a high level, the stiffness matrix is a
  relationship between forces and displacements, i.e. how much force
  is generated due to an infestismal displacement of any node of our
  lattice. When reasoning about this idea, remember that the
  simulation lattice is composed of hexaderal elements - every
  interior node is connected to 27 neighboring nodes. These are the
  neighbors which form the one-ring of adjecent cells. Moreover, in
  three dimensions each relationship to a neighbor takes the form of a
  $3 \times 3$ matrix, relating the force vector of the node to the
  displacement vector of each neighbor. For a hypothetical problem of
  a $128^3$ domain, this means our total non-zero element count in our
  matrix is approximately 500 million entries, or just under two
  gigabtyes of storage space if using a floating point
  representation\footnote{This is not including any additional storage
    cost overheads for representing the sparse matrix itself. This
    number is simply the space required to store only the non-zero
    coeffiecents alone.}. Practically, this means that the solution to
  this matrix is non-trivial, at least if we want to accomplish it
  quickly.

  \section{User Interaction}

  Until this point, the discussion of materials and mechanical
  responses have been avoiding the last component of the initial
  problem statement: unscripted user interactions. This is a complex
  topic and it is best to approach it after having a basic
  understanding of how simulated materials are solved in general. In
  the real world, we interact with objects intuititaivly - we push and
  pull things, glue them in place, or join them via connectors. For
  simulated objects, we can perform many of the same conceptual tasks
  by mapping their high level descriptions into the fundemental forces
  behind them. By doing so, we introduce a new wrinkle into
  our previous equations: external forces and constraints.

  Let us briefly look at a high level overview of the various types of
  user interactions and constraints we will want for our problem. For
  plastic surgery, we commonly are performing a limited set of
  conceptual actions: Pulling on tissue, clamping it down, sutureing
  it together, or cutting it. We'll talk about the last action later,
  as it's a bit more complicated. The first three actions are fairly
  straightforward however.

  \textbf{Position constraints} Arguably the easiest constraint type,
  positional constraints can be thought of as holding a part of an
  object fixed in space with infinetely strong glue. Mathematically
  speaking, these constraints are refered to as \textit{dirichlet}, or
  boundary conditions which must be met at all times during the
  simulation. Numerically, dirichlet conditions are applied by
  disallowing any displacement to occur on such constrained nodes. For
  virtual surgery, these constraints could serve to pin the edges of
  tissue down, or act as a transition between the simulated and
  non-simulated regions, preventing separation. 

  \textbf{Force constraints} The second form of constraint are force
  constraints. These constraints are direct applications of force
  targeted at particular points of the simulated object. In terms of
  the equations we've looked at before, force constraints modify both
  the stiffness matrix $K$ and the righthand side $\mathbf f$ in
  Equation \ref{equ:newtonstep}. In this document, we will be treating
  these types of constraints as zero-rest length springs, which obey
  the simple one dimensional form of Hooke's law:
  \begin{equation}
    \label{equ:hookeslaw}
    F = -k\lVert x_a - x_b\rVert_2
  \end{equation}
  In this linear relationship, the restorative forces (forces acting
  in a direction to return the spring to its resting length) $F$ are proportinal to the
  amount the spring has been stretched, represented by the distance
  between its two end points $x_a$ and $x_b$, all of which is
  modulated by a spring constant $k$.

  While this relationship is simple on the surface, there does exist a
  small complication: Where are its endpoints? Remember, the only
  entities in our problem that hold forces are the nodes of the
  simulation lattice. While we could enforce the restriction that
  spring endpoints always coinside with these nodes, it would be more
  useful to be able to place these force constraints arbitrarily. To
  do so, we must embed the end points into the lattice. This is
  similar to how we embed the vertices of our material mesh into the
  lattice - interpolating nodal displacements. However, in this case,
  we also must perform the opposite action once we have computed a
  force - distributing this force back onto the nodes of the cell
  where the spring was attached.

  With this minor alteration, we can now represent two types of force
  constraints: Single sided and dual sided. Single sided constraints
  refer to spring constraints which have one end point embedded in the
  simulation lattice, while the second is positioned manually in
  space, acting as a dirichlet constraint. This allows a user to pull
  regions of the simulated object towards positions in space. By
  embedded both end points, we can build dual sided constraints. As
  expected, these constraints act to pull two disparat regions of
  material closer together. In the context of virtual surgery, this
  could be used to represent a suture or other similar mechanical linkage.
  
  \subsection{Topology Change}

  The last major form of user interaction with the simulated tissue
  object is via topology change, i.e. cutting. Apart from physically
  moving tissue around or joining it with sutures, this interaction
  represents the surgeon's ability to incise tissue with a scalpel or
  other cutting implement. As such, supporting topology change is very
  important as otherwise most procedures would be impossible to
  perform.

  As such, it is worth considering the ways that topology change in
  simulated objects can be done. Initially, it might be tempting to
  say that topology change needs to mimic the actual physics of a
  scalpel cutting tissue. In other words, track the physical forces at
  the tip of the blade, the strain limit of the tissue, and causing
  the material of the simulated object to separate as the blade
  traverses the surface. This approach is appealing due to the
  naturalness of the effect - as the blade moves, tissue is cleanly
  cut and slides away from it on either side, just as one might expect
  in a real life procedure.

  Unfortunately, this form of \textit{online} cutting, where the
  topology change and deformation solution are fully coupled, is very
  complicated. It requires careful maintainance of multiple data
  structures and the stiffness matrix, all while ensuring the problem
  remains robust and avoids spurious forces. It is also not, strictly
  speaking, nessessary. If we were approaching the problem from a
  psychomotor perspective, where we were interested in training the
  user on how it would feel to cut tissue, this type of cutting would
  be required to correctly inform haptic feedback devices. However, as
  stated in the previous chapter, we are mostly interested in the
  cognitive aspects of plastic surgery. Under this regime, online
  cutting is less important than providing users a clear interface in
  which to plan and enact cuts with precision. Further, while it is
  true that more advanced procedures require cutting tissue which has
  been pulled back and exposed, there do exist an important class of
  so-called local flap procedures which can be cut all at once. These
  procedures are important, as they are serve as core building blocks
  for more complex procedures, often being composed together in
  non-trivial ways. It will be this class of procedures we will be
  focusing on for supporting topology change.

  Instead of supporting online cutting, we instead will design our
  platform to handle cuts performed at discrete time epochs. In fact,
  we will provide an additional simplification constraint: All cuts
  will occur to the material's reference configuration, before any
  deformation occurs. While this restriction prevents us from
  attempting the more complex surgical procedures, like a cleft lip
  repair or operations on highly volumetric regions like the human
  breast, it easily supports local flap style operations. Yet, even this
  highly restricted form of topology change has important challanges
  to overcome. The first challange is what to we mean by topology
  change in the context of a simulation lattice? If we had replaced
  our lattice with a simulation mesh, perhaps a conforming
  tetrahederal mesh, topological change could have meant disconnecting
  tetrahedra from each other, or even remeshing the object to conform
  to the cut. However, a lattice typically doesn't have that kind of
  flexibilty, as its topology is implicit for performance reasons. The
  best approximation to the tetrahedral case would be to simply remove
  entire cells, leading to extremely coarse, axis-aligned cuts. In
  Chapter \ref{Gridiron}, we'll show how these problems can be
  overcome with the introduction of a hybrid implicit-explicit
  topology design.

  \subsection{Collision}

  A final topic related to user interaction, and to some degree
  topology change, is collision. Collision refers to the ability of
  the simulated object to correctly respond to impact with another
  object or itself. Collision is a particularly tricky property to
  support in elastic simulation, espeically self collision. Collision
  can be broken into two distinct steps: collision detection and
  collision response, or handling. Collision detection is the aptly
  named process of determining whether or not collision has occured at
  any particular point in time. For volumetric objects, which possess
  distinct inside and outside regions, collision detection typically
  takes the form of testing for surface penetration. The challange is
  how to perform this test efficiently, since it needs to run at least
  once per timestep. For rigid body collisions, i.e. collisons between
  the soft body and an external rigid object, level sets have been
  employed quite successfully for collision detection purposes with
  complexities in the order $O(1)$ for any point of
  interest. For self collision scenarios, 
  
  \section{Engineering Challanges}












% domain can described by a number of
% data structures. The classic approach would be to describe the region
% as a mesh - a collection of discrete points connected by a defined
% topology. However, other options such as a level sets or density point
% clouds are also possible. For the purposes of this work, we will be
% focusing on the first option - meshes.
% Again, we can work our way down from the real world version
% to a more mathematical description and finally to a discrete version
% for the purposes of computation.














% or an energy density
% function $\Psi(\textbf{F})$- which captures the amount of potential energy present at an
% infintesimal volumtric region as a result of the
% deformation. Ultimately, the energy density function is our first hook
% into the precise material properties of our deformable object. By
% choosing the energy density function carefully, we can make the object
% respond more strongly to certain types of deformation over others. For
% instance, a popular measure is the Green Strain:

% $$ $$


% The
% goals of this function is to accurately capture all of the strain
% energy, while disregarding components of the deformation gradient
% which result from translations or rotations. 



% We can use this deformation map to generate a continuous deformation
% gradient, defined as
% $$\mathbf F(\vec X): \frac{\partial \phi
%   (\vec{X})}{\partial \vec{X}}$$
% This gradient is the 










% For the physical simulation of the human body, or parts of it, we can
% represent its form as a deformable solid. While it is true that human
% tissue is composed primarily of water, the collection of tissues that
% make up human skin, fat, and muscle can be treated as a solid, yet
% deformable, mass. What then is a deformable solid, from a computer
% simulation perspective? A \gls{deformablesolid} is a volumetric
% region whose shape reacts to both internal mechanics and external
% stimulus events. Right away, this definition opens new questions. What
% do we mean by shape? And how does it react? What are the internal
% mechanics and how are they different from external stimuli? In this
% section, these questions will be looked at in more detail.

% \subsection{State representation}

% The first issue to content with is the fundamental nature of the
% deformable object in the context of the simulation system. Unlike
% reality, where such objects are composed of collections of atoms
% arranged into more complex molecules which in turn give rise to
% macroscopic behavior, simulated objects exist as we, the developers,
% decide. Our decisions in this space determine the ultimate potential
% behaviors of the simulated objects. The main issue facing simulation
% development is that, unlike reality, computer systems are
% fundementally an approximation of the extreme complexity found in the
% real world. As such, we are forced to make trade offs when deciding
% what features our simulations will be able to support. Many of these
% trade offs can be rooted in the basic question of state
% representation.


% The first property to consider is that of continuous and discrete
% representation. Often, real materials are said to be continuous in
% nature\footnote{This is of course not true. As mentioned before, real
%   materials are composed of discrete, if extremely numerous,
%   collections of atoms. But we can generally ignore this fact when looking at large scale
%   (relatively speaking), bulk behaviors.}, and when we  mathematically
% reason about their behavior, we typically do so using continuous
% equations. However, despite the advantages this approach provides,
% such as easier reasoning about differential behavior, it is quite
% tricky to represent continuous mathematical models on a computer. This
% is not an unheard of problem by any means - IEEE floating point
% representation attempts to solve the exact issue of representing
% continuous numbers as discrete collections of bits manageable by
% modern computer architectures. Likewise, it is important to be able to
% transition between the continuous equations representing material
% behavior and the discrete versions of them which can be represented
% by computer hardware.

% But what are these material equations? Again, understanding this
% aspect requires us to continue examining the question of state
% representation. In this case, the major question is that of simulated
% properties. First, let us consider the property of shape. While not
% always easy to define for all things, such as gases or particulate
% matter, for the types of objects we are interested in simulating for
% virtual surgery, we can reasonably say that they possess a defined
% boundary in three dimensional space. That is to say, there exists a
% closed two dimensional manifold which separates the inside, or
% material part, from the outside, the exterior part, of the simulated
% object. Again though, we face the problem of representation - do we
% define our boundary as a continuous or discrete entity? Certainly for
% simple shapes (e.g. spheres, polyhedra), we can describe the boundary
% through anaylical expressions. But for more complex shapes, such as
% the organic forms of the human body, these anaylical expressions
% become extremely complex or impossible to define. And this is before
% we face the problem of how to represent this anaylical expression on a
% inherently discrete device such as a computer. To deal with this
% issue, we turn to alternative discrete representations.

% Our options for discrete boundary representations are numerous. A
% common approach is to use a mesh representation, where the boundary is
% defined by a collection of discrete points in space, called vertices,
% connected together via poliganal faces. While common, this approach
% suffers from being only $C0$ continuous. Alternatively, spline patches
% instead of polginal faces can be used to achieve smoother and more
% accurate representations of a surface. Unfortunately, regardless of
% the method choosen here, meshes are ultimately surface representations
% and do not intrinsically indicate which side is to be considered
% inside and which is outside. While conventions involving winding
% patterns can help with the determination, other representations, such
% as volumetric meshes or level sets can supply this information
% unambigously. Volumetric meshes consist of vertices, but unlike
% surface meshes, they are joined together by polyhedra instead of
% polygons. This provides a distinct collection of ``solids'' which
% clearly delinate the material. Level sets, on the otherhand, are a
% representation of an object as a sampled collection of distances away
% from the its surface. This collection allows for easy determination of
% the objects interior and provides other advantages, such as easy
% determination of shortest path to the surface.

% In addition to the shape of an object, there are other mechanical
% properties we might want to store, such as velocity, mass, or
% acceleration. Exactly which properties depend on what behavior we want
% to simulate and how we wish to simulate it. For deformable objects, we
% can typically break this down into three different behaviors, each
% defined by particular aspects in their continuous equations. Before we
% can talk about these behaviors, it is important to first describe the
% basic nature of the continous material equations that govern the
% behavior of our simulatated objects. As was mentioned at the beginning
% of this section, the behavior of our simulated objects is a
% relationship, or function, between applied external forces and the
% mechanical state of the object, where the state could be position,
% velocity or other mechanical properties. The precise definition of
% this function varies, from simple linear relationships to more complex
% non-linear ones. However, we can also categorize these equations into
% three broad categories, as mentioned, based on another property,
% time. The first, static simulations, treat time as an invarient. That
% is to say, static problems assume that the application of external
% forces never change and do not include time dependent effects, such as
% interia and velocity. Dynamic problems, in contrast, directly include
% time as a variable, allowing for simulatuted material to properly
% exhibit the properties of velocity and accleration. The third form,
% quasi-static problems, are a interesting mix between the first
% two. Quasi-static problems use the same equations as static problems,
% but allow the external forces to be parameterized by time, with the
% assumption that the problem is fully solved for each change in the
% time parameter.

% \subsection{Reference Configuration}

\chapter{Related Work}

Faced with the challenge of providing an interactive virtual
environment for authoring plastic surgery simulations, the field of
computer graphics research has generated many potential solutions and
techniques to solve this problem. Procedural techniques
\cite{JoshiMDGS:2007,WangP:2002,KavanCZO:2008} offer real-time
performance for certain animation tasks but lack the physical accuracy
needed in surgical simulations. Consequently, some research ventures
into surgical simulation turned to elastic deformation models
\cite{TerzoPBF:1987} that responded more realistically to scenarios of
probing and cutting \cite{BroC:1996,MendoL:2003,NienhS:2001}. However,
these early works were limited in their scope and effectiveness due to
computational cost, geometric constraints and oversimplified material
models. In this section we outline prior contributions that help
address these limitations, and review a number of existing surgical
simulation systems.

\section{Simulation of Elastic Materials}

Simulation of elastic deformable models is ubiquitous in computer
graphics and remains a vibrant area of research. Algorithmic
techniques for deformable body simulation, pioneered by Terzopoulos et
al \shortcite{TerzoPBF:1987} have attained a significant level of
maturity, leading to broad adoption in visual effects, games, virtual
environments and biomechanics applications. However, numerous
theoretical and technical challenges remain. Research efforts often
emphasize improved computational performance for cost-conscious
interactive applications. Simulation of complex materials and concerns
about accuracy and fidelity, especially in biomechanics applications,
place additional strain on simulation techniques. Finally, ease of use
and deployment in production environments is an important trait that
scholarly research work needs to be sensitive to. Our paper proposes a
grid-based simulation technique with a number of original components
that enhance performance and parallelism, natively accommodate complex
materials (including skin, flesh and muscles) while offering the
simple and familiar front-end of a lattice deformer for easy
integration into an animation pipeline.

Lattice-based volumetric deformers are popular components in both
physics-based and procedural animation techniques. In the case of
physics-based simulation, one of their key advantages is that they
avoid having to construct a simulation-ready conforming volume mesh,
which is a delicate preprocessing task often requiring supervision and
fine-tuning. Another crucial benefit is that the regularity of such
data structures enables aggressive performance optimizations as
vividly demonstrated by shape matching techniques
\cite{RiverJ:2007}. Cartesian lattices have also been leveraged to
accelerate performance in physics-based approaches, albeit
predominantly for simple models such as linear or corotated elasticity
\cite{MuellTG:2004,GeorgW:2008,McAdaZSETTS:2011}. Prior graphics work,
however, has not demonstrated such aggressive performance gains from
lattice-based discretizations when highly nonlinear, anisotropic or
incompressible materials are involved. In part, this is attributed to
the fact that simulation of complex materials commands an increased
level of attention to issues of robust convergence and reliable
treatment of incompressibility. Mature solutions to these concerns
have predominantly been demonstrated in the context of specific
discretizations (e.g. explicit tetrahedral meshes) where regularity of
data structures, compactness of memory footprint and
parallelization/vectorization potential were not inherently
emphasized. Furthermore, as applications requiring the use of complex
materials are also likely to emphasize geometric accuracy, they often
opt for conforming mesh discretizations due to their superior
performance in capturing intricate boundary features, even if their
computational cost is higher.

The use of non-conforming meshes has valuable benefits for fracture
modeling, but embedded techniques have also been used in their own
right for reasons of simplicity and performance. Mueller et al.\!
\shortcite{MuellTG:2004} employed an embedding scheme to perform
volumetric simulation of objects described by their boundary mesh. The
regularity of lattice embeddings has also been exploited in shape
matching techniques \cite{RiverJ:2007} to achieve significant
performance accelerations. Embedding has been combined with
homogenization \cite{NesmePF:2006,KhareMOD:2009} to resolve
sub-element variation of material parameters, optionally with the use
of non-manifold embedding lattices to support objects with a branching
structure \cite{NesmeKJF:2009}.  Jerabkova et al.\!
\shortcite{JerabBBFA:2010} employed a method similar to our own, using
a finer voxel grid to capture material topology to be embedded in a
coarser, non-manifold voxel grid. Finally, Zhao and Barbi\v{c}
\shortcite{ZhaoB:2013} demonstrated the use of multiple voxel grid
domains to segment a model hierarchically, which they used to simulate
plants at interactive rates. In addition to classical FEM approaches,
some authors have achieved success with more exotic
variations. Extended FEM (XFEM) formulations have also been explored
\cite{JerabK:2009}, where discontinuities are introduced into the
element's shape functions, to model cutting. In a similar vein,
Kaufmann et al.\!  \shortcite{KaufmMBG:2009} used discontinuous
Galerkin FEM formulations. Others have dispensed with mesh based
discretizations completely, preferring meshless methods
\cite{DeB:2000} which were also used for surgical simulation
\cite{DeKLS:2005}.

\section{Anatomical Modeling}
Approaches based on the Finite Element Method (FEM) have been
particularly popular in the medical simulation community
\cite{MarchADC:2008} where the need for biologically accurate
materials is more pronounced. In one of the earliest uses of advanced
materials in computer animation, Chen and Zeltzer
\shortcite{ChenZ:1992} focused on anatomical structures such as
muscles. FEM techniques were further leveraged in the animation
literature for the discretization of linear elasticity for fracture
modeling in a small-strain regime \cite{OBriH:1999}. Highly nonlinear
materials such as active musculature \cite{TeranBHF:2003} exposed
challenges in robustness and numerical stability of FEM
discretizations. Invertible FEM \cite{IrvinTF:2004} improved
simulation robustness in scenarios involving extreme compression,
while modified Newton methods \cite{TeranSIF:2005} reduced the cost of
implicit schemes with large time steps. Several of these algorithms
have been incorporated in open-source modeling and simulation packages
\cite{SinSB:2013}. Solutions have also been proposed for material
behaviors such as incompressibility \cite{IrvinSF:2007} and
viscoelasticity \cite{GokteBO:2004,WojtaT:2008}, both of which can be
found in typical biomaterials.  Recent results in coupled
Lagrangian-Eulerian simulation of solids have also facilitated the
inclusion of intricate contact and collision handling in biomechanical
modeling tasks \cite{SuedaKP:2008,LiSNP:2013,FanLP:2014}.

\section{Topology Change}

A number of techniques have targetted topology change during
simulation, due to cutting or fracture. Early work \cite{TerzoF:1988b}
resorted to breaking connectivity of elements when stress limits were
exceeded. Later methods \cite{NienhS:2001} split tetrahedra near cut
boundaries and then used vertex snapping to more accurately
approximate the cut. Local remeshing was also employed to simulate
cracks in brittle materials \cite{OBriH:1999}. An issue with such
subdivision schemes is the possible creation of poorly conditioned
elements, which prompted a number of authors to pursue embedded
simulation schemes \cite{MolinBF:2004,TeranSBNLF:2005}. These
techniques use non-conforming meshes with elements which are only
partially covered by material, in lieu of conforming
remeshing. Embedded simulation can provide a great degree of
flexibility in cutting and fracture scenarios \cite{SifakDF:2007},
although cutting meshes along arbitrary surfaces requires delicate
book-keeping and careful handling of degeneracies.


\section{Surgical Simulation}
While much of the previously discussed work is geared towards general
elastic body simulation in computer graphics, many relevant results
originated in surgery-specific work. Pieper et al.\!
\shortcite{PiepeLR:1995} demonstrated a very early surgical simulation
platform for facial procedures, using FEM elastic shells.  Many
surgical simulation projects focus on the mechanical manipulation of
organs and other soft internal objects
\cite{NienhS:2001,KimCDS:2007}. Even expensive commercial simulators
like the Lap Mentor and GI Mentor primarily focus on pushing and
cutting simulated internal organs
\cite{SUSAC:2002--2014b,SUSAC:2002--2014}. These types of simulations
are so common that several open source frameworks have been built to
specifically support further development
\cite{AllarCFBPDDGo:2007,CavusGT:2006}. These provide easy access to
common components like haptic feedback and APIs to connect multiple
simulated components. Certain surgical simulation systems are tailored
to specific skills, including interactive simulations of needle
insertion \cite{ChentARCHGSO:2009}.

\section{Performance Optimization}
Improving simulation rates is a common challenge for many interactive
modeling tasks, and even more so for accuracy-conscious applications
such as virtual surgery. Attempts to improve performance have either
relied on new data structures, faster solvers, or aggressive use of
parallelization.  The Boundary Element Method \cite{JamesP:1999} has
been used to achieve interactive deformation rates for objects
manipulated via their surface.  Hermann et al.\!
\shortcite{HermaRF:2009} analyzed data flow in their simulations to
inform a parallel scheduler for multicore systems. To avoid write
hazards during parallel code execution, Kim et al.\!
\shortcite{KimP:2011} proposed a system of computation phases with
coalesced memory writes, which allowed them to parallelize force
computation. Related efforts by Courtecuisse and Allard
\shortcite{CourtA:2009}, developed a parallel version of the
Gauss-Seidel algorithm that can run on GPUs. Regular discretizations
have also been coupled with multigrid solvers to facilitate GPU
accelerations for elastic skinning techniques \cite{McAdaST:2010}.
However, in spite of the efficiency of multigrid schemes, adapting
them to the presence of incisions or other intricate topological
features can be a nontrivial proposition.

The need for efficient, ideally interactive simulation of deformable
bodies has been catered to by several procedural techniques
\cite{JoshiMDGS:2007,KavanCZO:2008,VaillBGCRWGP:2013}, although when
fidelity and realism is the objective, physics-based methods are
typically employed \cite{TerzoPBF:1987}. The Finite Element Method has
been very popular in this aspect, and various authors have
successfully used it to animate a diverse spectrum of behaviors
\cite{OBriH:1999,TeranBHF:2003,IrvinTF:2004}.
 
Grid-based, embedded elastic models
\cite{MuellTG:2004,NesmePF:2006,McAdaZSETTS:2011,PatteMS:2012,MitchCS:2015}
have been very popular due to their inherent potential for performance
optimizations, and can also be used with shape-matching approaches
\cite{RiverJ:2007}. They form the foundation for a class of highly
efficient, multigrid-based numerical solution techniques
\cite{ZhuSTB:2010,GeorgW:2008,DickGW:2011}.

Authors have sought to accelerate simulation performance via a number
of avenues, including the use of optimized direct solvers
\cite{SinSB:2013} and delayed updates to factorization approaches
\cite{HechtLSO:2012}. Others have sought to leverage the Boundary
Element Method \cite{JamesP:1999} to approach real-time deformation
and similar formulations that abstract away interior degrees of
freedom to accelerate collision processing \cite{GaoMS:2014}. Our
method has significant ties to these approaches, as well as the
general class of Schur complement methods \cite{QuartV:1999}. In our
present work, we leverage such a formulation to aggregate local
neighborhoods of simulation elements into composite elements that
interface with the simulation system exclusively via their boundary.

\section{Collision Handling}

Level set methods were first introduced by Osher and
Sethian~\shortcite{OsherS:1988} for tracking moving interfaces in the
context of Hamilton-Jacobi equations.  Subsequently, Adalsteinsson and
Sethian~\shortcite{AdalsS:1994} proposed substantial runtime savings
by restricting computations to a thin band of active voxels near the
interface.  Sethian~\shortcite{Sethi:1998} proposed fast marching
methods for monotonically advancing fronts as well as for redistancing
the level set using values seeded only on the narrow band.  Besides
fast computation, a number of methods have also been proposed for
efficiently storing level sets including octrees~\cite{LosasGF:2004},
RLE representations~\cite{HoustNBNM:2006,IrvinGLF:2006,ChentM:2011},
the VDB data structure~\cite{Muset:2013} which evolved from Dynamic
Tubular Grids~\cite{NielsM:2006} and the DB+Grid data
structure~\cite{Muset:2011}, and the virtual-memory based SPGrid data
structure~\cite{SetalABS:2014}.

Methods have been proposed for computing implicit representations of
non-manifold surfaces~\cite{BloomF:1995,YuanYW:2012}. Similar ideas
were used for simulating bubbles~\cite{ZhengYP:2006} and multiphase
fluids~\cite{LosasSSF:2006}. Our work diverges from these approaches
as we enhance the expressive capability of a \emph{single} level set
by embedding signed distance values on an explicit mesh. Our work is
related to the practice of embedding high-resolution geometry in
regular meshes, a concept that was first leveraged by M\"{u}ller et
al.~\shortcite{MulleTG:2004} for deformable body simulations and
fracture.  In addition to hexahedral embeddings, methods such as the
virtual node algorithm~\cite{MolinBF:2004} have been used to create
non-manifold tetrahedral lattices that correspond to thin topological
features in the embedding geometry. Virtual node concepts are also
similar to XFEM methods which were used for crack
modeling~\cite{MoeesDB:1999} and for cutting and fracturing thin
shells~\cite{KaufmMBGG:2009}. This principle has continued to evolve
with many of the topological limitations in prior approaches being
raised by Sifakis et al.~\shortcite{SifakDF:2007} and has been
successfully used in production tools as well~\cite{HellrSSST:2009}.

Our non-manifold level set approach is inspired by these methods, but
it needs to be made cognizant of further topological limitations that
the signed distance field imposes on our representation (see
Section~\ref{sec:non-manifold-level-sets}). Notably, when dealing with
collisions near thin features, all of the aforementioned approaches
employed detection and response techniques based on surface
meshes~\cite{BridsFA:2002} that rely on the availability of good
surface meshes, are computationally expensive, presume collision-free
history or use impulses which makes implicit integration
challenging. To accelerate collision detection and response while
allowing for implicit integration, methods have been proposed using
implicit surface representations~\cite{McAdaZSETTS:2011} which work
even in near-interactive settings, but require enough level set
resolution to avoid any non-manifold features altogether. Recently,
image-based techniques~\cite{FaureBAF:2008,WangFP:2012} have been
proposed which provide an interesting alternative.  Finally, implicit
surfaces have also been recently used in real-time skinning
applications~\cite{VaillBGCRWGP:2013,VaillGBWC:2014}.




\bibliographystyle{unsrt}
\bibliography{references4}

%% Now the glossary
% \glossarystyle{altlistgroup}
%\glsaddall
\printglossaries

%% Want an index?  Neither did I.
%\printindex

\end{document}

