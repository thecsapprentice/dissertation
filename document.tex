%!TEX TS-program = pdflatex
% dissertation.tex -- main dissertation file
%
% Wisconsin dissertation template
% Copyright (c) 2008-2009 William C. Benton.  All rights reserved.
%
% This program can redistributed and/or modified under the terms
% of the LaTeX Project Public License Distributed from CTAN
% archives in directory macros/latex/base/lppl.txt; either
% version 1 of the License, or (at your option) any later version.
%
% This program includes other software that is licensed under the
% terms of the LPPL and the Perl Artistic License; see README for details.
%
% You, the user, still hold the copyright to any document you produce
% with this software (like your dissertation).
%
\PassOptionsToPackage{dvipsnames}{xcolor} % prevent an option clash
% Comment to insert real images
\PassOptionsToPackage{draft}{graphicx}

%%% You'll want ``oneside'' for the deposit version, but probably not for any versions that don't need to meet the UW requirements
\documentclass[12pt,oneside,letterpaper]{memoir}

% preamble.tex -- packages to include
%
% Wisconsin dissertation template
% Copyright (c) 2008 William C. Benton.  All rights reserved.
%
% This program can redistributed and/or modified under the terms
% of the LaTeX Project Public License Distributed from CTAN
% archives in directory macros/latex/base/lppl.txt; either
% version 1 of the License, or (at your option) any later version.
%
% This program includes other software that is licensed under the
% terms of the LPPL and the Perl Artistic License; see README for details.
%
% You, the user, still hold the copyright to any document you produce
% with this software (like your dissertation).

%% You should use natbib
\IfFileExists{natbib.sty}{%
\usepackage[authoryear]{natbib}%
}{}

\usepackage[margin=1in]{geometry}
\ifpdf
\usepackage[pdftex]{graphicx}
\else
\usepackage{graphicx}
\fi
\usepackage{pdflscape}
\usepackage{rotating}
%% You probably need appendix, if you want appendices
\IfFileExists{appendix.sty}{%
\usepackage{appendix}%
}{}


%% the spacing in memoir is weird, you'll need to use this
\DisemulatePackage{setspace}
%% \usepackage[onehalfspacing]{setspace}
\usepackage[doublespacing]{setspace}

%% List setup; the ``hanglist`` environment will allow you to have
%% nicely-typeset enumerated lists (i.e. with the numbers hanging in
%% the margins).  You need at least version 2.1 of enumitem.sty.  If
%% you don't have enumitem installed at all, hanglist will just be an
%% alias for enumerate.
\IfFileExists{enumitem.sty}{%
\usepackage[loadonly]{enumitem}[2007/06/30]%
\newlist{hanglist}{enumerate}{1}% 
\setlist[hanglist]{label=\arabic*.}%
\setlist[hanglist,1]{leftmargin=0pt}%
}{%
\newenvironment{hanglist}{\begin{enumerate}}{\end{enumerate}}%
}


%% Avoid ugly "Type 3" fonts
\usepackage{lmodern}
\usepackage[LY1]{fontenc}
\usepackage[LY1]{eulervm}
%% Comment out any of these that you don't want
\usepackage{amssymb}
\usepackage{amsmath}
\usepackage{amsthm}
\usepackage{bm}
%\usepackage{theorem}
\usepackage{hyperref}
\usepackage[dvipsnames]{xcolor}
\usepackage{wrapfig}
\usepackage[nonumberlist,xindy,toc]{glossaries}

\usepackage{algorithm}
\usepackage[noend]{algpseudocode}
\usepackage{color}
\usepackage{framed}
\definecolor{shadecolor}{gray}{0.9}

\algnewcommand{\algorithmicreshape}{\textbf{reshape}}%
\algnewcommand{\Reshape}{\algorithmicreshape~}%

\algnewcommand{\algorithmicrecompute}{\textbf{compute}}%
\algnewcommand{\Compute}{\algorithmicrecompute~}%

\algnewcommand{\algorithmicreset}{\textbf{set}}%
\algnewcommand{\Set}{\algorithmicreset~}%

\renewcommand{\algorithmicrequire}{\textbf{Input:}}
\renewcommand{\algorithmicensure}{\textbf{Output:}}

\usepackage{multirow}
\usepackage{array}
\usepackage{booktabs}
\usepackage{comment}
\usepackage{forest}
\usepackage{float}

\newcommand{\tr}{\ensuremath{\mbox{tr}}}
\newcommand{\str}{\ensuremath{\mbox{\small tr}}}
\newcommand{\bvec}[1]{\hbox{\boldmath $#1$}}
\newcolumntype{C}[1]{>{\centering\let\newline\\\arraybackslash\hspace{0pt}}m{#1}}


\definecolor{light-gray}{gray}{.9}
\definecolor{dark-gray}{gray}{0.60}

\newcommand{\reducedstrut}{\vrule width 0pt height .9\ht\strutbox depth .9\dp\strutbox\relax}
\newcommand{\chighlight}[1]{%
  \begingroup
  \setlength{\fboxsep}{.5pt}%  
  \colorbox{light-gray}{\reducedstrut#1\/}%
  \endgroup
}

\usepackage{tikz}
\usetikzlibrary{decorations.pathreplacing,calc}
\newcommand{\tikzmark}[1]{\tikz[overlay,remember picture] \node (#1) {};}

\newcommand*{\AddNote}[4]{%
    \begin{tikzpicture}[overlay, remember picture]
        \draw [decoration={brace,amplitude=0.5em},decorate,ultra thick,red]
            ($(#3)!(#1.north)!($(#3)-(0,1)$)$) --  
            ($(#3)!(#2.south)!($(#3)-(0,1)$)$)
                node [align=center, text width=2.5cm, pos=0.5, anchor=west] {#4};
    \end{tikzpicture}
}%


\IfFileExists{mathpartir.sty}{%
\usepackage{mathpartir}%
}{}

%%%%% LISTINGS package and setup
\IfFileExists{listings.sty}{%
\usepackage{listings}%
}{}



%% Get rid of ugly borders around PDF hyperlinks (e.g. for cross-references, bib entries, or URLs)
\hypersetup{pdfborder = 0 0 0}

%% You want microtype.
\IfFileExists{microtype.sty}{%
\usepackage[protrusion=true,expansion=true]{microtype}%
}{}

%\pagestyle{thesisdraft}

\usepackage{afterpage}

% Surround parts of graphics with box
\usepackage{boxedminipage}

%% booktabs (thx to Nate Rosenblum for bringing this beautiful package
%% to my attention)
\IfFileExists{booktabs.sty}{%
\usepackage{booktabs}%
}{}

% This is now the recommended way for checking for PDFLaTeX:
\usepackage{ifpdf}



%% Substitute your favorite serif and sans fonts here....
%\IfFileExists{tgpagella.sty}{%
% TeX Gyre pagella, like Palatino
%\usepackage{tgpagella}%
%}{}

\usepackage{ifthen}

\providecommand\pdfforprinting{true}

\ifthenelse{\equal{\pdfforprinting}{false}}{%
\newenvironment{myrotatedtable}{%
  \newgeometry{margin=1cm}%
  \landscape%
  \table[p]%
}{%
  \endtable%
  \endlandscape%
  \restoregeometry%
}%
}{%
\newenvironment{myrotatedtable}{%
    \begin{sidewaystable}%
}{%
    \end{sidewaystable}
}%
}%

\usepackage{makeidx}
\makeindex
\makeglossaries

{\theoremstyle{plain}
\newtheorem{thm}{Theorem}[chapter]
\newtheorem{cor}[thm]{Corollary}
\newtheorem{define}[thm]{Definition}
\newtheorem{exmpl}[thm]{Example}
}
{\theoremstyle{remark}
\newtheorem{rmk}[thm]{Remark}
}

\newtheoremstyle{customsty1}
{3pt}%
{3pt}%
{}% --- body font
{}% --- indent amount
{\bfseries}% --- Theorem head font
{:}% --- Punctuation after head
{.5em}% --- space after head
{}% --- theorem head spec (can be left empty, meaning 'normal')

% Define 'newtheorems' that use ``customsty1''
{\theoremstyle{customsty1} 
}

% Some helper commands used in original source documents
\newcommand{\newtext}[1]{#1}
\newcommand{\oldtext}[1]{}
\newcommand{\newmath}[1]{\ensuremath{#1}}
\newcommand{\oldmath}[1]{}
\newcommand{\addedfragile}[1]{#1}
\renewcommand{\changed}[2]{\newtext{#1}\oldtext{#2}}




%%% NB: the ``deposit'' chapter- and page- styles should conform to UW
%%% requirements.  If you are producing a pretty version of your
%%% dissertation for web use later, you will certainly want to make
%%% your own chapter and page styles.

\makechapterstyle{deposit}{%
  \renewcommand{\chapterheadstart}{}
  \renewcommand{\printchaptername}{}
  \renewcommand{\chapternamenum}{}
  \renewcommand{\printchapternum}{\parbox{2em}{\MakeLowercase{\Large\scshape\thechapter{}}} }
  \renewcommand{\afterchapternum}{}
  \renewcommand{\printchaptertitle}[1]{%
  \raggedright\Large\scshape\MakeLowercase{##1}}
  \renewcommand{\afterchaptertitle}{%
  \vskip\onelineskip \hrule\vskip\onelineskip}
}

\makepagestyle{deposit}
 
\makeatletter
 
\renewcommand{\chaptermark}[1]{\markboth{#1}{}}
\renewcommand{\sectionmark}[1]{\markboth{#1}{}}
 
\makeevenfoot{deposit}{}{}{}
\makeoddfoot{deposit}{}{}{}
\makeevenhead{deposit}{\thepage}{}{}
\makeoddhead{deposit}{}{}{\thepage}
\makeatother

%%% set up page numbering for chapter pages to satisfy UW requirements
%%% NB: You will want to delete until the ``SNIP'' mark if you are
%%% making a ``nice'' copy
\copypagestyle{chapter}{plain}
\makeoddfoot{chapter}{}{}{}
\makeevenhead{chapter}{\thepage}{}{}
\makeoddhead{chapter}{}{}{\thepage}
%%% SNIP

%%% bib nonsense
\makeatletter
\newenvironment{wb-bib}[1]{%
  \chapter*{references}
\ifnobibintoc\else 
\phantomsection 
\addcontentsline{toc}{chapter}{References} 
\fi 
\prebibhook
  \begin{bibitemlist}{#1}}{\end{bibitemlist}\postbibhook}

\AtBeginDocument{%
  \@ifpackageloaded{natbib}{% natbib is loaded
    \addtodef{\endthebibliography}{}{\vskip-\lastskip\postbibhook}
    \@ifpackagewith{natbib}{sectionbib}{% with sectionbib option
      \renewcommand{\bibsection}{\@memb@bsec}}%
      {\renewcommand{\bibsection}{\@memb@bchap}}}%
  {}
  \@ifpackagewith{chapterbib}{sectionbib}{%
    \renewcommand{\sectionbib}[2]{}
    \renewcommand{\bibsection}{\@memb@bsec}}{}
}
\makeatother


\renewcommand{\sc}{\scshape}

%%% Local Variables:
%%% mode: latex
%%% TeX-master: "../document"
%%% End:

\input{includes/defs}
\input{includes/thesisdefs}
%\usepackage[noend]{algcompatible}
\usepackage{algorithm}
\usepackage[noend]{algpseudocode}
\usepackage{tikz}
\usetikzlibrary{calc}
\algnewcommand\algorithmicgoto{\textbf{goto: }}
\algnewcommand\Goto{\algorithmicgoto}
\algrenewcommand\algorithmicindent{0.5em}%
\algnewcommand{\LineComment}[1]{\State {\color{NavyBlue} \(\triangleright\) \textbf{#1}}}

\usepackage{caption}% http://ctan.org/pkg/caption
\captionsetup[ruled]{labelsep=colon}
\makeatletter
\@addtoreset{algorithm}{chapter}% algorithm counter resets every chapter
\makeatother
\renewcommand{\thealgorithm}{\thechapter.\arabic{algorithm}}% Algorithm # is <chapter>.<algorithm>


\makeatletter
\newcommand{\StatexIndent}[1][3]{%
  \setlength\@tempdima{\algorithmicindent}%
  \Statex\hskip\dimexpr#1\@tempdima\relax}
\algdef{S}[IF]{IfNoThen}[1]{\algorithmicif\ #1}%
\makeatother

\makeatletter
\tikzset{%
  remember picture with id/.style={%
    remember picture,
    overlay,
    save picture id=#1,
  },
  save picture id/.code={%
    \edef\pgf@temp{#1}%
    \immediate\write\pgfutil@auxout{%
      \noexpand\savepointas{\pgf@temp}{\pgfpictureid}}%
  },
  if picture id/.code args={#1#2#3}{%
    \@ifundefined{save@pt@#1}{%
      \pgfkeysalso{#3}%
    }{
      \pgfkeysalso{#2}%
    }
  }
}

\def\savepointas#1#2{%
  \expandafter\gdef\csname save@pt@#1\endcsname{#2}%
}

\def\tmk@labeldef#1,#2\@nil{%
  \def\tmk@label{#1}%
  \def\tmk@def{#2}%
}

\tikzdeclarecoordinatesystem{pic}{%
  \pgfutil@in@,{#1}%
  \ifpgfutil@in@%
    \tmk@labeldef#1\@nil
  \else
    \tmk@labeldef#1,(0pt,0pt)\@nil
  \fi
  \@ifundefined{save@pt@\tmk@label}{%
    \tikz@scan@one@point\pgfutil@firstofone\tmk@def
  }{%
  \pgfsys@getposition{\csname save@pt@\tmk@label\endcsname}\save@orig@pic%
  \pgfsys@getposition{\pgfpictureid}\save@this@pic%
  \pgf@process{\pgfpointorigin\save@this@pic}%
  \pgf@xa=\pgf@x
  \pgf@ya=\pgf@y
  \pgf@process{\pgfpointorigin\save@orig@pic}%
  \advance\pgf@x by -\pgf@xa
  \advance\pgf@y by -\pgf@ya
  }%
}


% end of Andrew's code

\newlength\AlgIndent
\setlength\AlgIndent{0pt}
% main command to draw the colored background
\newcounter{mymark}
\newcommand\ColorLine{%
  \stepcounter{mymark}%
  \tikz[remember picture with id=mark-\themymark,overlay] {;}%
  \begin{tikzpicture}[remember picture,overlay]%
    \filldraw[YellowGreen]%
   let \p1=(pic cs:mark-\themymark),
   \p2=(pic cs:mark-\themymark) in 
   ([xshift=-\ALG@thistlm,yshift=-0.7ex]0,\y1)  rectangle ++(\linewidth+.1em,\baselineskip);
  \end{tikzpicture}%
}%
\makeatother

% colored loops and declarations
\makeatletter
\algnewcommand\CREQUIRE{\item[\setlength\AlgIndent{1.6em}\ColorLine\algorithmicrequire]}%
\algnewcommand\CENSURE{\item[\setlength\AlgIndent{1.6em}\ColorLine\algorithmicensure]}%
\algnewcommand\CSTATE{\State\ColorLine}%
\algnewcommand\CSTATEx{\Statex\ColorLine}%
\algnewcommand\CCOMMENT{\Comment\ColorLine}%

\algdef{SE}[WHILE]{CWHILE}{CENDWHILE}%
   [2][default]{\ColorLine\algorithmicwhile\ #2\ \algorithmicdo #1}%
   {\ColorLine\algorithmicend\ \algorithmicwhile}%
\algdef{SE}[FOR]{CFOR}{CENDFOR}%
   [2][default]{\ColorLine\algorithmicfor\ #2\ \algorithmicdo #1}%
   {\ColorLine\algorithmicend\ \algorithmicfor}%
\algdef{SE}[FOR]{CFORALL}{CENDFORALL}%
   [2][default]{\ColorLine\algorithmicforall\ #2\ \algorithmicdo #1}%
   {\vspace{-\baselineskip}%\algorithmicend\ \algorithmicfor
}%
\algdef{SE}[FOR]{FORALL}{ENDFORALL}%
   [2][default]{\algorithmicforall\ #2\ \algorithmicdo #1}%
   {\algorithmicend\ \algorithmicfor}%
\algdef{SE}[LOOP]{CLOOP}{CENDLOOP}%
   [1][default]{\ColorLine\algorithmicloop #1}%
   {\ColorLine\algorithmicend\ \algorithmicloop}%
\algdef{SE}[REPEAT]{CREPEAT}{CUNTIL}%
   [1][default]{\ColorLine\algorithmicrepeat #1}%
   [1]{\ColorLine\algorithmicuntil\ #1}%
\algdef{SE}[IF]{CIF}{CENDIF}%
   [2][default]{\ColorLine\algorithmicif\ #2\ \algorithmicthen #1}%
   {\vspace{-\baselineskip}%\algorithmicend\ \algorithmicif
}%
\algdef{C}[IF]{CIF}{CELSIF}%
   [2][default]{\ColorLine\algorithmicelse\ \algorithmicif\ #2\ \algorithmicthen #1}%
\algdef{Ce}[ELSE]{CIF}{CELSE}{CENDIF}%
   [1][default]{\ColorLine\algorithmicelse #1}%



\makeatother

%%% Local Variables: 
%%% mode: latex
%%% TeX-master: "nonmanifold_levelsets"
%%% End: 


\svnidlong{$LastChangedBy$}{$LastChangedRevision$}{$LastChangedDate$}{$HeadURL: http://freevariable.com/dissertation/branches/diss-template/dissertation.tex $} 

\clearpage\pagenumbering{roman}  % This makes the page numbers Roman (i, ii, etc)

\title{Techniques for Single System Integration of Elastic Simulation Features}
\author{Nathan M. Mitchell}
\department{Computer Sciences}

\date{2016}

\begin{document}

%%% Uncomment the following if your .bib contains references that you will not 
%%% explicitly cite, but that should be in the final bibliography:
% \nocite{*}

\ifpdf
\DeclareGraphicsExtensions{.pdf, .jpg, .tif}
\else
\DeclareGraphicsExtensions{.eps, .jpg}
\fi

\maketitle

%% Add \part declarations if you want, but it's not necessary
%\part{Preliminaries}

\svnidlong{$LastChangedBy$}{$LastChangedRevision$}{$LastChangedDate$}{$HeadURL: http://freevariable.com/dissertation/branches/diss-template/frontmatter/frontmatter.tex $}
\vcinfo{}

%%% SOME OF THIS CODE IS ADAPTED FROM THE VENERABLE withesis.cls

% COPYRIGHT PAGE
%  - To include a copyright page use \copyrightpage
\copyrightpage

% DEDICATION
\begin{dedication}
  \emph{For my grandfather,\\ whose lifelong passion for teaching others\\ inspired me to
    pursue my own goals and education.}\\
  \vspace*{3cm}
  \includegraphics[width=4cm]{frontmatter/dahlia.png}
\end{dedication}

%% BEGIN PAGESTYLE

%%% You can pick a pagestyle if you want; see the memoir class
%%% documentation for more info.  The default ``deposit'' option meets
%%% the UW thesis typesetting requirements but is probably
%%% unsatisfactory for making a version of your dissertation that
%%% won't be deposited to the graduate school (e.g. for web or a nice
%%% printed copy)

\chapterstyle{deposit}
\pagestyle{deposit}


% ACKNOWLEDGMENTS
\begin{acks}

I would like to thank all the people who have
supported me and my work over the years. In
particular, I would like to thank my family who
have shown me unwavering support and my friends
who have kept me sane. Also, I would like to thank
Ben Recht, Ronald Fedkiw, Matthew Cong for their
advice and discussions on several of the
contributions made in this
dissertation. Additionally, I wish to extend my
thanks to Dr. Timothy King, the medical residents
of the Plastic \& Reconstructive Surgery program,
and the School of Medicine and Public Health at
the University of Wisconsin-Madison for their
involvement and support with the virtual plastic
surgery platform, as well as acknowledge Aaron
Oliker and BioDigital Systems for supplying the
models used in this project.

This work was generously supported by the following grants
and funding sources: NSF grants IIS-1253598,
IIS-1407282, CCF-1423064, CCF-1533885,
CNS-1218432, CCF-1438992.


%%% Local Variables: 
%%% mode: latex
%%% TeX-master: "../dissertation"
%%% End: 

\end{acks}

% CONTENTS, TABLES, FIGURES
\renewcommand{\printtoctitle}[1]{\chapter*{#1}}
\renewcommand{\printloftitle}[1]{\chapter*{#1}}
\renewcommand{\printlottitle}[1]{\chapter*{#1}}

\renewcommand{\tocmark}{}
\renewcommand{\lofmark}{}
\renewcommand{\lotmark}{}

\renewcommand{\tocheadstart}{}
\renewcommand{\lofheadstart}{}
\renewcommand{\lotheadstart}{}

\renewcommand{\aftertoctitle}{}
\renewcommand{\afterloftitle}{}
\renewcommand{\afterlottitle}{}

\renewcommand{\cftchapterfont}{\normalfont} 
\renewcommand{\cftsectionfont}{\itshape} 
\renewcommand{\cftchapterpagefont}{\normalfont} 
\renewcommand{\cftchapterpresnum}{\bfseries} 
%\renewcommand{\cftchapterleader}{\cftparfillskip} 
%\renewcommand{\cftsectionleader}{\cftparfillskip} 
%\renewcommand{\cftchapterafterpnum}{\cftparfillskip} 
%\renewcommand{\cftsectionafterpnum}{\cftparfillskip} 

% \captionnamefont{\small\sffamily} 
% \captiontitlefont{\small\sffamily} 

\renewcommand{\cftdot}{}

% \renewcommand{\contentsname}{contents}
% \renewcommand{\listfigurename}{list of figures}
% \renewcommand{\listtablename}{list of tables}

\addtocontents{toc}{~\hfill\textbf{Page}\par}
\tableofcontents

\clearpage
\listoftables
\addtocontents{lot}{~\hfill\textbf{Page}\par}

\clearpage
\listoffigures
\addtocontents{lof}{~\hfill\textbf{Page}\par}

\clearpage
\listofalgorithms
\addtocontents{loa}{~\hfill\textbf{Page}\par}
\addcontentsline{toc}{chapter}{List of Algorithms}


\clearpage
% NOMENCLATURE
% \begin{conventions}
% % \begin{description}
% % \item{\makebox[0.75in][l]{term}
% %        \parbox[t]{5in}{definition\\}}
% % \end{description}
% \input{conventions}
% \end{conventions}

%% The UW graduate school no longer wants a UMI abstract page
%% Should you need one for some reason, uncomment the following
%% lines.  Thanks to Matt Fredrikson for reporting this!

% \advisorname{Gottlob Frege}
% \advisortitle{Professor}
% \begin{umiabstract}
%  
Techniques for simulating the behavior of elastic objects has matured
considerably over the last several decades, tackling diverse problems
from non-linear models for incompressibility to accurate
self-collisions. Alongside these contributions, advances in parallel
hardware design and algorithms have made simulation more efficient and
affordable than ever before. However, prior research has commonly made
the choice to sacrifice one aspect of simulation for another,
resulting in a fragmented landscape of solutions. For complex,
real-world tasks, such as virtual surgery, a holistic approach is
desirable, where complex behavior, performance, and ease of modeling
are supported equally. This dissertation aims to provide that in the
form of several interconnected investigations, each of which
contribute a piece of an unified solution. First, it will be
demonstrated how advanced non-linear materials can be combined with
lattice deformers yield simulations with behavioral richness and a
high potential for parallelism. This potential will be exploited to
show how a hybrid solver approach based on large macroblocks can
accelerate the convergence of these deformers. Extending the lattice
concept with non-manifold topology allows for efficient processing of
self-collisions and topology change. Finally, a case study on virtual
plastic surgery will demonstrate a real-world problem space where
these ideas can be combined to build an expressive authoring tool,
allowing surgeons to record procedures digitally for future reference
or education.


%%% Local Variables:
%%% mode: latex
%%% TeX-master: "../dissertation"
%%% End:

% \end{umiabstract}

\begin{abstract}
  
Techniques for simulating the behavior of elastic objects has matured
considerably over the last several decades, tackling diverse problems
from non-linear models for incompressibility to accurate
self-collisions. Alongside these contributions, advances in parallel
hardware design and algorithms have made simulation more efficient and
affordable than ever before. However, prior research has commonly made
the choice to sacrifice one aspect of simulation for another,
resulting in a fragmented landscape of solutions. For complex,
real-world tasks, such as virtual surgery, a holistic approach is
desirable, where complex behavior, performance, and ease of modeling
are supported equally. This dissertation aims to provide that in the
form of several interconnected investigations, each of which
contribute a piece of an unified solution. First, it will be
demonstrated how advanced non-linear materials can be combined with
lattice deformers yield simulations with behavioral richness and a
high potential for parallelism. This potential will be exploited to
show how a hybrid solver approach based on large macroblocks can
accelerate the convergence of these deformers. Extending the lattice
concept with non-manifold topology allows for efficient processing of
self-collisions and topology change. Finally, a case study on virtual
plastic surgery will demonstrate a real-world problem space where
these ideas can be combined to build an expressive authoring tool,
allowing surgeons to record procedures digitally for future reference
or education.


%%% Local Variables:
%%% mode: latex
%%% TeX-master: "../dissertation"
%%% End:

\end{abstract}

\clearpage\pagenumbering{arabic}

%%% END STUFF TAKEN FROM WITHESIS EXAMPLE FILE

%%% Local Variables:
%%% mode: latex
%%% TeX-master: "../dissertation"
%%% End:


\chapter{Introduction}

Mankind has been seeking better methods to capture the human body and
its pathologies for the purpose of healing since the earliest days of
modern surgery. From the anatomical drawings of De Vinci, to more
modern practices of constructing realistic simulacra, doctors, and
their students, have been chasing tools that allow them to practice
their skills before operating on real patients. Existing research
shows the benefits of engaging in these practice
sessions\cite{GallaRCHFMSS:2005}. Practiced surgeons make fewer
mistakes and can use preparation sessions to plan new approaches
safely.

This general philosophy, which can be summed up with the classic
proverb of ``measure twice, cut once'', is practiced by many high risk
professions. From flight school to driving simulators, computer
constructed virtual environments have become an integral part of
training people. The reasoning is three-fold: computer simulations are
relatively low cost and are able to be reset quickly, novel parameters
and situations can be introduced more easily than physical
environments, and a trainee's progress can be easily recorded for
later review. With these advantages over purely physical training
environments, why are surgeons still using aids such as diagrams,
physical mannequins, and cadavers?

The short answer is that they often lack any better
alternatives. Performing surgery is a complex task involving a
combination of dexterous and cognitive, often spatial reasoning,
skills\cite{GallaRCHFMSS:2005}. Tools that support all of these areas
are difficult to get right, and most attempts to build technological
aids have focused on subsets of the skills required. Historically,
these have been the dexterous skills, which many authors have tried to
solve with a variety of haptic simulation
techniques\cite{MendoL:2003,LindbT:2007}. While these surgical
simulation philosophies are useful, and have been used in commercial
products\cite{SUSAC:2002--2014}, they don't really meet the need of
training cognitive skills. This need varies across surgical
specialties - reconstructive plastic surgery, which is the focus of this document,
requires the surgeon to have internalized geometrical intuitions in
order to manipulate tissue in the three dimensional space of the human
body. It is this lack that has kept traditional, less technological
aides as the core of many plastic surgery training programs.

In contrast to internal surgery, plastic surgery suffers from the
practical reality that the results of any operation will be visible to
others. This fact puts added pressure on surgeons to not only complete
their tasks correctly, but also in an aesthetically pleasing way. It
follows then that a simulator for plastic surgery operations must
provide an environment for surgeons to freely practice design, as well
as correctly display the outcomes.

This dissertation aims to support the following statement: The creation
of simulators for craniofacial reconstructive plastic surgery is now
technically feasible. The craniofacial region, consisting of the scalp
and face, has been chosen for since many operations can be performed
by manipulating localized skin flaps. In pursuit of this goal, this document will
describe how existing standard practices for simulating elastic
materials can be combined in a holistic fashion, all with an eye
towards performance and practical usability. In the process,
limitations with the current approaches will be explored and, in some
cases, alternative techniques will be proposed to solve issues
stemming from the unique challenges found in plastic surgery simulators.


\chapter{Motivation}

In order to properly understand this dissertation, it is important to
place it the proper context. While the driving topic of this document
is simulation in the plastic surgery domain, the work presented here
in this dissertation derives from more fundamental motivations. The
goal of this chapter is to present a more complete motivation for the
design decisions and technical contributions presented throughout this
document.  In pursuit of this goal, the remainder of this chapter is
divided into three sections: Simulation for Modern Medicine, Practical
Interactive Systems, and Software Design. Of these, the second section
is the longest, as it provides a philosophical framework for the rest
of the document.

\section{Medical Simulation}

The first motivation sets the scene. Why go through the effort of
building a complex, interactive simulation system? What could warrant
the thousands of man-hours required to build such a piece of software?
An argument that the act of building such a system is its own reward
certainly does exist, and is explored more deeply in the second
section, but is not what needs to be discussed first. Instead, we must
provide a practical purpose which guides the evaluation of the final
product. What is its real-world usefulness? How well does it
accomplish the task it was designed for?  What are the fine grained
aspects of the task? What features need to implemented (or perhaps
more interestingly, not implemented)? Who are its users? All of these
questions require an external guide to answer and are not addressed by
pure software design issues.

For this dissertation and the work described within, the external
guide is that of plastic surgery. In particular, the software systems
presented in this document are designed to support reconstructive
plastic surgeons in describing craniofacial operations in a virtual
environment. In later sections, more information about the technical
details of this domain and what specific tasks this research helps
address will be presented, but currently it is important examine the
question: why. \textbf{Why medicine?} Concerns about improving healthcare are
certainly not new phenomena in human society. From saving lives to
simply improving quality of life, the more effective we can make
doctors through technology benefits us all. \textbf{So why surgery?} Surgery is
a complex discipline that requires a long period of committed
training\cite{Chung:2005}. And since surgery is almost always the last resort of
treatment, when an illness or condition has reached its most severe
state, everyone wants their surgeon to be the best available. The
combination of these two aspects will always create a high demand for
experienced and practiced surgeons. However, research has shown that
many residents complete surgical programs with no or limited first
hand experience in many procedures\cite{BellBTRCBL:2009}. The goal of
simulated aids is to provide other avenues for learning and practice beyond
what is available in books, or through personal experience.\textbf{ So why
reconstructive craniofacial plastic surgery?} In this case, the reasons are both
humanitarian and technical.

Reconstructive plastic surgery, as opposed to general cosmetic plastic
surgery, is commonly performed after other surgeries as a
post-operation corrective action. For example, when an individual is
treated for skin cancer and has had a tumor removed, plastic surgery
is often required to ``fix'' (i.e. close) the hole left over from the
removal. Reconstructive plastic surgery is also used to correct for
congenital deformities, often early in life. Pathologies such as cleft
lip and palate, while now treatable, still require complex surgeries at a
young age. Failure to do so, or performing interventions poorly, can
leave a person with not only physical complications, but social and
emotional issues. Plastic surgery, corrective or otherwise, deals with
appearance and shape. These surgeries are tricky due to a dual set of
requirements for success: not only must they solve the immediate
technical problem (e.g. close a hole), but they must do so in the most
aesthetically pleasing way. It is this aspect of plastic surgery that makes it
very different from other forms of surgical practice, and also makes
it very interesting for computer graphics research. It is also why
there is a humanitarian need for computer technologies to aid
physicians in training and practice: Given the severe psychological
impact a poorly executed plastic surgery can cause, plastic surgeons
should have as much experience as possible before performing any procedure
on a live patient.



\section{Practical Visual Interactive Systems}


At a high level, the work I completed in support of my thesis falls under the description
of a Practical Visual Interactive Systems, or PVIS. I am defining a
PVIS as follows: A Practical Visual Interactive System is a dynamic
virtual environment where by user input produces changes in state for
the purpose of supporting some task. These systems are actually
everywhere in our modern society, though we probably don't often think
of them in these terms. Examples include video games, virtual avatar
systems, training systems, and more. Over the course of this
section, my goal is to define what a PVIS is and is not, talk briefly
about some of the design challenges one might face when building them,
and finally try to justify why their complexity is ultimately worth
the effort in their creation.

\subsection{PVIS Deconstruction}

At its core, a PVIS is a virtual world, or environment. The size and
scope of the environment is not important, instead one should focus on
the fact that the environment \textit{is not real}. When discussing a PVIS, we
are talking about artificially constructed settings, often completely
described by a computer program and displayed via some
device. However, this is not to say that a PVIS is completely
disconnected from reality. As we will see later when we talk about
tasks, a PVIS is often attempting to mimic real life.

So what is a PVIS then? Is any virtual environment a PVIS? No, not
really. Referring to the definition at the beginning of this section,
a PVIS must adhere to several important properties. Some of these are
more important than others however. 

\begin{description}
\item[Dimension] Most of the forms of PVIS that people interact with
  today are displayed as 3D worlds. Full of rich and detailed 3D
  models, this type of environment is often preferred when the goal is
  to recreate some aspect of actual reality. However, I don't
  necessarily restrict PVIS to three dimensions. The representation
  of the environment is only important as far as it serves the purpose
  of the PVIS.

\item[Reactivity] A key aspect of a PVIS is that it holds some type of
  state and this state can be changed by the user in response to their
  interactions. This is vital to the system's ability to be
  interactive as it separates the PVIS from a static diorama.

\item[Dynamism] Often conflated with the aspect of reactivity, dynamism
  brings in a temporal component to the PVIS. This moves the PVIS away
  from a what otherwise might be a simple state machine, whose states
  can be traversed in any order, to a system with a history. Whether
  this aspect of history is permanent, where user input creates
  indelible changes to the virtual environment, or flexible, giving a
  user a timeline of consequences to move through, depends on the
  ultimate purpose of the PVIS.

\item[Interactivity] What separates a PVIS from a movie is the fact
  that user input is a key aspect of using a PVIS. While a movie often
  has simple controls, such as play, pause, rewind, interacting with a
  PVIS requires a richer set of semantics. The closest example of a
  PVIS movie would be a video game. While video games can often be
  cinematic and heavily plot driven, the player's input is ultimately
  responsible for driving the world forward in a fine grained way, as
  opposed to simply pressing play.

\item[Purpose] A PVIS without a purpose is not a PVIS. All of the
  other aspects of a PVIS must work together to support the
  purpose. The purpose of a PVIS helps define what features are
  important and clarifies what metrics should be used to measure its
  ultimate success. However, the reasons one might want a PVIS are
  wide and varied. I've already touched on video games, whose purposes
  might be to entertain or tell a story. But PVIS come in many forms,
  for many purposes. A virtual meeting place, where participants exist
  as avatars is a PVIS, and the ultimate purpose is to facilitate
  communication between users. Professional disciplines also employ
  PVIS for a variety of purposes, including training via simulators
  and prototyping via computer aided design tools.
      
\end{description}

\subsection{PVIS Challenges}

Designing a Practical Visual Interactive System comes with a wide
array of challenges, which is not especially surprising when
considering how many components go into their makeup. In this section,
these challenges will be reviewed, hopefully to give better context
for the design decisions I have made for the rest of my work in this
document.

\subsubsection{No One Size Fits All?}

When developing software, an oft times discussed design goal is
re-useability. Whether this refers to reusing techniques or software
modules varies, but in general the more reusable something can be
made, the more useful it becomes. Designing a PVIS, however, imposes
some interesting roadblocks for the principle of reusability. The
first issue that often comes up is that the requirements for a PVIS,
while they can appear similar on the surface (e.g. display a 3D
environment, respond to user input, etc.), are often implemented
with situtational specific optimizations due to the tight restrictions
placed on such systems (e.g. near real-time performance, extremely
complex environments, etc.). Developers faced with these issues can
easily fall into the trap of \textit{blind optimization}, where they
optimize the implementation, often quite expertly, for some aspect,
but without considering the rest of the system, or how the system
might be reused later\footnote{I think its important to distingiush this from
\textit{premature optimization}, where the developer spends time
optimizing an implementation before knowing if such effort is
required. Blind optimization is justifiable, if perhaps unwise in a
larger context. Premature optimization may end up being wasted effort
at best, deterimental at worst.}.

This is not to say that designing general purpose PVIS platforms is
impossible. Game developers, over a long developmental history, have
created many excellent general platforms for game development,
referred to as game engines. But increasingly, these platforms, and
the developers who design them, are becoming a field in their own
right. While in the past, game developers may have done everything
from writing low level graphics code to higher level game logic,
modern games are often written by developers who know little about the
low level optimizations required to reach the fidelty and performance
expected of modern games. Instead, these skills and the software that
results from their practice is found in game engines - highly tuned,
carefully optimized systems which are not a game per se, but act as a
solid foundation for games written on top of them. They provide
\textit{services}: rendering, resource management, network support,
user interface toolkits, and much more. In essence, this divide is not
dissimilar to that of applications and operating systems. 

The work I have done in this document generally follows this
philosophy. As will be described in later sections, I have attempted
to stick to two general principles when designing the systems
presented here.

\begin{enumerate}
\item \textbf{Avoid Uncalled for Optimization} In short, when
  designing my implementations, it was important to avoid optimizing
  too soon. In many cases, there were open questions (and still are!)
  that premature optimization could have prevented us from
  understanding. Only when it was clear that optimization was needed,
  were further steps taken, and taken in complete understanding of
  their consequences on the rest of the system.

\item \textbf{Enforce Clean Separation} As will be discussed in more
  detail later, I have attempted to separate the domain specific
  motivation, plastic surgery, from the underlying technologies that
  enabled its simulation. This allowed me to design a coupled, but
  seperable, system, where the components needed to build a plastic
  surgery simulation were isolated from the components needed to build
  a high performance finite element simulator.
  
\end{enumerate}


\subsubsection{Realism and Believability}

For the visual and virtual world aspect of a PVIS, one primary
question that arises is one of  believability or realism. These
concepts are often conflated, though they really should be considered
separately. I view realism as a more direct measure of how much a user
of the system sees what is being presented as an accurate simulacra of
a real object or environment. In contrast, believability is the
measure of how much a user trusts, or is willing to accept, the
information being presented. Lets use modern special effects in
movies, for an example. On one hand, special effects can be used to
introduce elements that have direct real world counterparts, which 
are either too expensive or impractical to use. Examples might include
simulating a full ocean, using a green screen to replace environments,
or full simulation of real people. If done well, these techniques
increase realism, by convincing the audience that the elements being
fabricated really exist. On the other hand, special effects can
introduce clearly fantastical elements: mythical creatures, exotic
environments, or magical effects. These are elements that even the
least critical audience member would not hesitate to say are fake. But
the goal here is not to trick them into believing something is real,
but to convince them that its \textit{plausible}. That the fire breathing
dragon, if it were to really exist, would look just like it does on
the screen. This is the believability aspect at work. Of course, there
is no reason that realism and believability can not work
together. Examples include live action heroic characters being
replaced by animated versions in order to have them perform impossible
stunts. Here the result must be have realism (it looks like the real
person), but also be believable (the impossible stunt looks like it
could have been pulled off).

So what does this mean for PVIS designs in general? And for medical
simulation specifically? The primary issue with a PVIS is that of user
input. While a movie can be scripted and refined until realism and
believability is exactly where the actors and artists want it to be, a
PVIS must produce similar levels of fidelity when faced with arbitrary
user actions. Dealing with this challenge often requires domain
specific knowledge in order to limit the potential space of user
interactions. For example, in a surgery simulation, the user can't do
absolutely anything to a section of simulated tissue. They are forced,
by the context of the situation, to interact with it using the
simulated tools provided: scalpels, sutures, etc... This allows a
surgery PVIS to be engineered to properly handle all the potential
outcomes of this restricted interface, to better produce realistic and
believable results.

%\subsection{Application Driven Design}




\subsubsection{Design Conflicts}

A major problem facing the construction of any PVIS is that of design
goal conflicts. By this I am referring to the all too common situation
where supporting one feature or aspect of a system, such as
performance, comes into direct conflict with implementing another
feature. Continuing with the performance example, suppose we wanted to
impose a strict requirement on visual updates to the user. By doing
so, we have restricted our updates with an upper bound on the maximum
amount of work they can complete at any one time due to time
restrictions. This choice may bias further choices towards the use of
other techniques, not because of any technical merit, but of
complexity.

Many of these conflicting goals exist, some well known in general
software design circles. In the context of physical simulation, there
are several conflicts we need to be especially aware of.


\paragraph{Reactivity Vs. Accuracy}
I touched on general reactivity before when talking about what defines
a PVIS. For simulation, we can use a more precise definition where the
reactivity of a system refers to the time between a user applies some
change to a simulated system (a force impulse, a constraint change,
etc...) and when the user sees the result of the action. This cause
and effect timing is the reactivity of the system. The smaller this
time, the more reactive the simulation \textit{feels}. We can see this
when comparing the simulation to real materials. For real objects, the
reactivity is effectively infinite, as the time between cause and
effect is extremely close to zero.

Of course, real materials have an advantage that simulated materials
do not. Because they are composed of individual atoms, real objects
effectively act as a perfect finite element simulation, where the
elements are almost infinitely small and operate completely in
parallel to each other. Computer simulated materials are much coarser
in their resolution and, despite great advances in parallel
processing, do not come close to that naturally available in real
materials. Thus, as we increase a simulated object's resolution in
order to capture more and more detail, or use more complex elements
that capture more interesting macroscopic effects, the overall
reactivity of the simulation decreases as more effort is spent
resolving each user action.

The challenge is to find the appropriate balance between the desired
reactivity of a simulation and the accuracy of the simulation. A major
focus of my work has been to explore how both of these aspects can be
increased simultaneously, both by exploiting underutilized parallelism
opportunities and by looking at novel data structures to extract
additional effective resolution without significantly doing so.
  
\paragraph{Domain Utility Vs. Generality}

Another two aspects that often find themselves in conflict for
physical simulation systems are the concepts of domain utility and
generality. Lets look at domain utility first, as its the more
straightforward of the two. In the simplest terms, domain utility
refers to making design choices in a system that primary serve the
specific task, or domain, that it is currently being built for. This
may refer to choosing or discarding certain features, deciding what
API best suits the current task, or making optimization along critical
paths for the client application. All of these choices can be
reasonable, even correct, as long as you never intend to reuse the
system for any other purpose.

Generality, on the other hand, asks what is the commonality of
different tasks and guides design choices along this route. A general
design should be flexible to different and changing requirements. Such
a system typically eschews APIs built for specific tasks and instead
tries to distill out the fundamental building blocks that any
potential client may need from the system. The difficulties with this
philosophy are two-fold. First, it isn't always obvious what the
fundamental interfaces are, partially because designers by necessity
must look at past applications to define them and are ignorant about
future ones. But secondly, general designs often cannot make
simplifying assumptions that domain specific knowledge provides and
leaves them with overly complex code that tries to optimize for every
use case or simpler code which doesn't optimize anything.

So why would anyone design a general system? On the surface, they seem
harder to build effectively and often don't result in well optimized
solutions, impacting other aspects such as reactivity. The short
answer is flexibility. For a well studied domain, where every last
detail is known and accounted for, a specialized system is probably
the best choice. But in order to answer challenging research
questions, tools and problems often have to change quickly and in
unexpected ways as researchers adapt to new findings and explore new
directions. Medical simulation is very much one of these areas, where
new questions are constantly arising and old preconceptions are
abandoned. As such, I have made considerable effort to building
generalized simulation systems, and attempting to identify which areas
are ready for optimization and which are not.

 
\subsection{Why Develop a PVIS?}

At this point, a concerned reader may be asking why one should go
through all the trouble of developing these types of systems. In the
previous sections, I've described a complex series of requirements for
PVIS construction, each challenging in isolation, let alone
combined. On top of these technical features, I've also laid out
various higher level issues, such as problems with generality, user
acceptance, and correctness. Despite all of these problems, I maintain
that developing PVIS style platforms is not only doable, but
ultimately desirable. In order to understand my position, let me
describe the three avenunes by which a PVIS creates value:
As a Catalyst, By Filling a Need, and Intrinsically.  

\subsubsection{Catalyst for Advances}

As I've covered before, a reasoning about a PVIS is a complex task,
implementing one is more so. In going about this process however, we
have the potential to learn a lot. Anytime we have to adapt a PVIS to
a new domain or integrate new functionality, we will ultimately
generate questions and hopefully new answers. In this way, PVIS
implementations are a generator for new research. Whether it is
answering questions about rendering, human-computer interaction,
systems, software design, optimization, or in the case of this thesis,
physical simulation, a PVIS acts as a fertile soil within which we, as
researchers, are able to experiment in many areas. But more than
simply providing a platform to test isolated ideas, a PVIS is by its
nature integrated. Any change affects and is affected by everything
else, forcing researchers to take in and understand the big picture
around their work and where it fits into the whole.

\subsubsection{Utility Gap}

The second reason that building a PVIS is often worthwhile is to fill
a need. As I stated, a PVIS is, at its core, software with a
purpose. In some cases, such as game development, many implementations
of a PVIS have already been created, making the bar much higher as to
the need for another one. But in other areas, such as medical
simulation, the gaps in functionality coverage are more severe. To use
the PVIS described by this document as an example, there have been
many projects developing systems for laparoscopic organ surgery, for
instance, but few systems for performing simulated plastic surgery,
let alone a fully featured PVIS with a rich set of
capabilities. Filling this gap with the utility provided by a PVIS is
then extremely valuable, as without it practitioners are left behind in
a world where their colleagues are more and more enjoying the benefits
of modern computing technology.

\subsubsection{Intrinsic Value}

At a basic level, a PVIS itself is valuable. Even if we ignore the
value of a PVIS in fulfilling its specified purpose, or the additional
research that can be spawned as a result of its construction, building
the PVIS is beneficial to its developers and the greater
community. For the former, implementing a PVIS requires time,
dedication, and skill - but no one enters and leaves such a project
unchanged. Simply being a developer on a PVIS helps a developer become
a better software engineer, simply through the long hours of practice
they will spend on it. Beyond individual developers, building a PVIS
is important to the community at large for the simple reason that it
demonstrates that such a project can be done. Like all large pieces of
software, sometimes the most important idea they can convey is that such a project is
even feasible at all.

% Let us break this down a little,
% starting with the first and last words, Practical Systems. The first
% important bit to notice is that I am using stating this in
% plural. Often, when discussing a piece of academic work we describe it
% in terms of its concrete contribution, and how this or that piece of
% code acts as a proof-of-concept. While this is certainly valuable as a
% method to convey new knowledge out into the world, I argue its less
% useful overall than a more complex arrangement of parts that solve
% real-world, practical problems. And I would further argue that a
% single domain specific system is often insufficient, and actual
% solutions require multiple interacting systems from a variety of
% domains to be truly successful.

% So what then are Practical \textit{Interactive} Systems? On a surface
% level reading, this could refer to anything that allows user
% interaction. However I would like to define a narrower scope, by
% restricting the discussion to interactive graphical systems. An
% excellent real-world example would be video games. A modern video game
% generally consists of multiple interlocking systems from a variety of
% disciplines: rendering, artificial intelligence, human computer
% interaction, and networking. All of these systems however are working
% towards one general goal: The display of a real-time interactive
% virtual experience to one or multiple users simultaneously. Now of
% course, all of these requirements come with caveats. For example,
% real-time interactivity is generally understood to be not real-time,
% but at a discreet frame-rate above some threshold. The important point
% here is that for these systems to be truly practical to people, they
% must merge the constraints imposed by interactivity and the complexity
% of interacting parts to provide requisite functionality. Tying these
% two pieces together, interactivity and complex systems, ultimately
% returns us back to the first facet: good design.


\section{Software Design}

To start, let us examine the first facet, Software Design. Whole books have
been written about this topic, many of which are full of valuable and
insightful ideas. Its not my goal to claim any particular innovation
or new vision for software design. Instead, I merely wish to lay clear
how software design has served as a guiding focus during my
research. All too commonly, research software is written quickly and,
while not nessessarily shoddily, perhaps without careful attention to
choices made that will later effect reusablity, maintainability, or
debugging. While these issues affect any major software project, my
experience has shown that the stresses associated with research often
exhasbertate them. Strict deadlines and a constant push for new ideas
can easily result in poor design choices. For these reasons, I have
tried to instill good software design ideas into my work.

What exactly is good software design? And more revelently, what is
good design in the context of simulation software? I'll be touching on
these questions more fully when I talk about the specific systems
I implemented over the course of the research described in this
document. In short, however, I think the primary goal of good design
for simulation software should be to support the concept of
Extensiblity Flexibility. What do I mean by that? In plain english, I
mean that simulation software should be designed such that is it easy
to add new features or use the software for new purposes without
requiring large amounts of rewriting. This is especially important for
research code, as required features are often unknown, or even
unknowable when development begins on a project. Being able to
continously reuse existing proven code as much as possible between projects
allows researchers to develop quicker, hopefully with less bugs.


\chapter{Related Work}

Faced with the challenge of providing an interactive virtual
environment for authoring plastic surgery simulations, the field of
computer graphics research has generated many potential solutions and
techniques to solve this problem. Procedural techniques
\cite{JoshiMDGS:2007,WangP:2002,KavanCZO:2008} offer real-time
performance for certain animation tasks but lack the physical accuracy
needed in surgical simulations. Consequently, some research ventures
into surgical simulation turned to elastic deformation models
\cite{TerzoPBF:1987} that responded more realistically to scenarios of
probing and cutting \cite{BroC:1996,MendoL:2003,NienhS:2001}. However,
these early works were limited in their scope and effectiveness due to
computational cost, geometric constraints and oversimplified material
models. In this section we outline prior contributions that help
address these limitations, and review a number of existing surgical
simulation systems.

\section{Simulation of Elastic Materials}

Simulation of elastic deformable models is ubiquitous in computer
graphics and remains a vibrant area of research. Algorithmic
techniques for deformable body simulation, pioneered by Terzopoulos et
al \shortcite{TerzoPBF:1987} have attained a significant level of
maturity, leading to broad adoption in visual effects, games, virtual
environments and biomechanics applications. However, numerous
theoretical and technical challenges remain. Research efforts often
emphasize improved computational performance for cost-conscious
interactive applications. Simulation of complex materials and concerns
about accuracy and fidelity, especially in biomechanics applications,
place additional strain on simulation techniques. Finally, ease of use
and deployment in production environments is an important trait that
scholarly research work needs to be sensitive to. Our paper proposes a
grid-based simulation technique with a number of original components
that enhance performance and parallelism, natively accommodate complex
materials (including skin, flesh and muscles) while offering the
simple and familiar front-end of a lattice deformer for easy
integration into an animation pipeline.

Lattice-based volumetric deformers are popular components in both
physics-based and procedural animation techniques. In the case of
physics-based simulation, one of their key advantages is that they
avoid having to construct a simulation-ready conforming volume mesh,
which is a delicate preprocessing task often requiring supervision and
fine-tuning. Another crucial benefit is that the regularity of such
data structures enables aggressive performance optimizations as
vividly demonstrated by shape matching techniques
\cite{RiverJ:2007}. Cartesian lattices have also been leveraged to
accelerate performance in physics-based approaches, albeit
predominantly for simple models such as linear or corotated elasticity
\cite{MuellTG:2004,GeorgW:2008,McAdaZSETTS:2011}. Prior graphics work,
however, has not demonstrated such aggressive performance gains from
lattice-based discretizations when highly nonlinear, anisotropic or
incompressible materials are involved. In part, this is attributed to
the fact that simulation of complex materials commands an increased
level of attention to issues of robust convergence and reliable
treatment of incompressibility. Mature solutions to these concerns
have predominantly been demonstrated in the context of specific
discretizations (e.g. explicit tetrahedral meshes) where regularity of
data structures, compactness of memory footprint and
parallelization/vectorization potential were not inherently
emphasized. Furthermore, as applications requiring the use of complex
materials are also likely to emphasize geometric accuracy, they often
opt for conforming mesh discretizations due to their superior
performance in capturing intricate boundary features, even if their
computational cost is higher.

The use of non-conforming meshes has valuable benefits for fracture
modeling, but embedded techniques have also been used in their own
right for reasons of simplicity and performance. Mueller et al.\!
\shortcite{MuellTG:2004} employed an embedding scheme to perform
volumetric simulation of objects described by their boundary mesh. The
regularity of lattice embeddings has also been exploited in shape
matching techniques \cite{RiverJ:2007} to achieve significant
performance accelerations. Embedding has been combined with
homogenization \cite{NesmePF:2006,KhareMOD:2009} to resolve
sub-element variation of material parameters, optionally with the use
of non-manifold embedding lattices to support objects with a branching
structure \cite{NesmeKJF:2009}.  Jerabkova et al.\!
\shortcite{JerabBBFA:2010} employed a method similar to our own, using
a finer voxel grid to capture material topology to be embedded in a
coarser, non-manifold voxel grid. Finally, Zhao and Barbi\v{c}
\shortcite{ZhaoB:2013} demonstrated the use of multiple voxel grid
domains to segment a model hierarchically, which they used to simulate
plants at interactive rates. In addition to classical FEM approaches,
some authors have achieved success with more exotic
variations. Extended FEM (XFEM) formulations have also been explored
\cite{JerabK:2009}, where discontinuities are introduced into the
element's shape functions, to model cutting. In a similar vein,
Kaufmann et al.\!  \shortcite{KaufmMBG:2009} used discontinuous
Galerkin FEM formulations. Others have dispensed with mesh based
discretizations completely, preferring meshless methods
\cite{DeB:2000} which were also used for surgical simulation
\cite{DeKLS:2005}.

\section{Anatomical Modeling}
Approaches based on the Finite Element Method (FEM) have been
particularly popular in the medical simulation community
\cite{MarchADC:2008} where the need for biologically accurate
materials is more pronounced. In one of the earliest uses of advanced
materials in computer animation, Chen and Zeltzer
\shortcite{ChenZ:1992} focused on anatomical structures such as
muscles. FEM techniques were further leveraged in the animation
literature for the discretization of linear elasticity for fracture
modeling in a small-strain regime \cite{OBriH:1999}. Highly nonlinear
materials such as active musculature \cite{TeranBHF:2003} exposed
challenges in robustness and numerical stability of FEM
discretizations. Invertible FEM \cite{IrvinTF:2004} improved
simulation robustness in scenarios involving extreme compression,
while modified Newton methods \cite{TeranSIF:2005} reduced the cost of
implicit schemes with large time steps. Several of these algorithms
have been incorporated in open-source modeling and simulation packages
\cite{SinSB:2013}. Solutions have also been proposed for material
behaviors such as incompressibility \cite{IrvinSF:2007} and
viscoelasticity \cite{GokteBO:2004,WojtaT:2008}, both of which can be
found in typical biomaterials.  Recent results in coupled
Lagrangian-Eulerian simulation of solids have also facilitated the
inclusion of intricate contact and collision handling in biomechanical
modeling tasks \cite{SuedaKP:2008,LiSNP:2013,FanLP:2014}.

\section{Topology Change}

A number of techniques have targetted topology change during
simulation, due to cutting or fracture. Early work \cite{TerzoF:1988b}
resorted to breaking connectivity of elements when stress limits were
exceeded. Later methods \cite{NienhS:2001} split tetrahedra near cut
boundaries and then used vertex snapping to more accurately
approximate the cut. Local remeshing was also employed to simulate
cracks in brittle materials \cite{OBriH:1999}. An issue with such
subdivision schemes is the possible creation of poorly conditioned
elements, which prompted a number of authors to pursue embedded
simulation schemes \cite{MolinBF:2004,TeranSBNLF:2005}. These
techniques use non-conforming meshes with elements which are only
partially covered by material, in lieu of conforming
remeshing. Embedded simulation can provide a great degree of
flexibility in cutting and fracture scenarios \cite{SifakDF:2007},
although cutting meshes along arbitrary surfaces requires delicate
book-keeping and careful handling of degeneracies.


\section{Surgical Simulation}
While much of the previously discussed work is geared towards general
elastic body simulation in computer graphics, many relevant results
originated in surgery-specific work. Pieper et al.\!
\shortcite{PiepeLR:1995} demonstrated a very early surgical simulation
platform for facial procedures, using FEM elastic shells.  Many
surgical simulation projects focus on the mechanical manipulation of
organs and other soft internal objects
\cite{NienhS:2001,KimCDS:2007}. Even expensive commercial simulators
like the Lap Mentor and GI Mentor primarily focus on pushing and
cutting simulated internal organs
\cite{SUSAC:2002--2014b,SUSAC:2002--2014}. These types of simulations
are so common that several open source frameworks have been built to
specifically support further development
\cite{AllarCFBPDDGo:2007,CavusGT:2006}. These provide easy access to
common components like haptic feedback and APIs to connect multiple
simulated components. Certain surgical simulation systems are tailored
to specific skills, including interactive simulations of needle
insertion \cite{ChentARCHGSO:2009}.

\section{Performance Optimization}
Improving simulation rates is a common challenge for many interactive
modeling tasks, and even more so for accuracy-conscious applications
such as virtual surgery. Attempts to improve performance have either
relied on new data structures, faster solvers, or aggressive use of
parallelization.  The Boundary Element Method \cite{JamesP:1999} has
been used to achieve interactive deformation rates for objects
manipulated via their surface.  Hermann et al.\!
\shortcite{HermaRF:2009} analyzed data flow in their simulations to
inform a parallel scheduler for multicore systems. To avoid write
hazards during parallel code execution, Kim et al.\!
\shortcite{KimP:2011} proposed a system of computation phases with
coalesced memory writes, which allowed them to parallelize force
computation. Related efforts by Courtecuisse and Allard
\shortcite{CourtA:2009}, developed a parallel version of the
Gauss-Seidel algorithm that can run on GPUs. Regular discretizations
have also been coupled with multigrid solvers to facilitate GPU
accelerations for elastic skinning techniques \cite{McAdaST:2010}.
However, in spite of the efficiency of multigrid schemes, adapting
them to the presence of incisions or other intricate topological
features can be a nontrivial proposition.

The need for efficient, ideally interactive simulation of deformable
bodies has been catered to by several procedural techniques
\cite{JoshiMDGS:2007,KavanCZO:2008,VaillBGCRWGP:2013}, although when
fidelity and realism is the objective, physics-based methods are
typically employed \cite{TerzoPBF:1987}. The Finite Element Method has
been very popular in this aspect, and various authors have
successfully used it to animate a diverse spectrum of behaviors
\cite{OBriH:1999,TeranBHF:2003,IrvinTF:2004}.
 
Grid-based, embedded elastic models
\cite{MuellTG:2004,NesmePF:2006,McAdaZSETTS:2011,PatteMS:2012,MitchCS:2015}
have been very popular due to their inherent potential for performance
optimizations, and can also be used with shape-matching approaches
\cite{RiverJ:2007}. They form the foundation for a class of highly
efficient, multigrid-based numerical solution techniques
\cite{ZhuSTB:2010,GeorgW:2008,DickGW:2011}.

Authors have sought to accelerate simulation performance via a number
of avenues, including the use of optimized direct solvers
\cite{SinSB:2013} and delayed updates to factorization approaches
\cite{HechtLSO:2012}. Others have sought to leverage the Boundary
Element Method \cite{JamesP:1999} to approach real-time deformation
and similar formulations that abstract away interior degrees of
freedom to accelerate collision processing \cite{GaoMS:2014}. Our
method has significant ties to these approaches, as well as the
general class of Schur complement methods \cite{QuartV:1999}. In our
present work, we leverage such a formulation to aggregate local
neighborhoods of simulation elements into composite elements that
interface with the simulation system exclusively via their boundary.

\section{Collision Handling}

Level set methods were first introduced by Osher and
Sethian~\shortcite{OsherS:1988} for tracking moving interfaces in the
context of Hamilton-Jacobi equations.  Subsequently, Adalsteinsson and
Sethian~\shortcite{AdalsS:1994} proposed substantial runtime savings
by restricting computations to a thin band of active voxels near the
interface.  Sethian~\shortcite{Sethi:1998} proposed fast marching
methods for monotonically advancing fronts as well as for redistancing
the level set using values seeded only on the narrow band.  Besides
fast computation, a number of methods have also been proposed for
efficiently storing level sets including octrees~\cite{LosasGF:2004},
RLE representations~\cite{HoustNBNM:2006,IrvinGLF:2006,ChentM:2011},
the VDB data structure~\cite{Muset:2013} which evolved from Dynamic
Tubular Grids~\cite{NielsM:2006} and the DB+Grid data
structure~\cite{Muset:2011}, and the virtual-memory based SPGrid data
structure~\cite{SetalABS:2014}.

Methods have been proposed for computing implicit representations of
non-manifold surfaces~\cite{BloomF:1995,YuanYW:2012}. Similar ideas
were used for simulating bubbles~\cite{ZhengYP:2006} and multiphase
fluids~\cite{LosasSSF:2006}. Our work diverges from these approaches
as we enhance the expressive capability of a \emph{single} level set
by embedding signed distance values on an explicit mesh. Our work is
related to the practice of embedding high-resolution geometry in
regular meshes, a concept that was first leveraged by M\"{u}ller et
al.~\shortcite{MulleTG:2004} for deformable body simulations and
fracture.  In addition to hexahedral embeddings, methods such as the
virtual node algorithm~\cite{MolinBF:2004} have been used to create
non-manifold tetrahedral lattices that correspond to thin topological
features in the embedding geometry. Virtual node concepts are also
similar to XFEM methods which were used for crack
modeling~\cite{MoeesDB:1999} and for cutting and fracturing thin
shells~\cite{KaufmMBGG:2009}. This principle has continued to evolve
with many of the topological limitations in prior approaches being
raised by Sifakis et al.~\shortcite{SifakDF:2007} and has been
successfully used in production tools as well~\cite{HellrSSST:2009}.

Our non-manifold level set approach is inspired by these methods, but
it needs to be made cognizant of further topological limitations that
the signed distance field imposes on our representation (see
Section~\ref{sec:non-manifold-level-sets}). Notably, when dealing with
collisions near thin features, all of the aforementioned approaches
employed detection and response techniques based on surface
meshes~\cite{BridsFA:2002} that rely on the availability of good
surface meshes, are computationally expensive, presume collision-free
history or use impulses which makes implicit integration
challenging. To accelerate collision detection and response while
allowing for implicit integration, methods have been proposed using
implicit surface representations~\cite{McAdaZSETTS:2011} which work
even in near-interactive settings, but require enough level set
resolution to avoid any non-manifold features altogether. Recently,
image-based techniques~\cite{FaureBAF:2008,WangFP:2012} have been
proposed which provide an interesting alternative.  Finally, implicit
surfaces have also been recently used in real-time skinning
applications~\cite{VaillBGCRWGP:2013,VaillGBWC:2014}.

\bibliographystyle{unsrt}
\bibliography{references4}

%% Now the glossary
%\glossarystyle{altlistgroup}
%\glsaddall
%\printglossaries

%% Want an index?  Neither did I.
%\printindex

\end{document}

