
\chapter{Discussion}

The conclusion to any large body of work is difficult, to say the
least. Nevertheless, it is important to reach some measure of
knowledge at the end of day, ``the point'' to use a common
vernacular. In the beginning of this document, a great deal of time
was spent describing a framework for practical visual interactive
systems. Later, these high level concepts were specialized into
systems for authoring plastic surgery operations using physical
simulation. A fair point can be raised that the subsequent chapters
did little to further this topic in a direct and obvious fashion. So
an astute reader is probably asking right now ``What was the point?''

In an attempt to recover the vision of the forest through the
metaphorical trees of technical details that have been presented thus
far, this author would like to propose the following argument: The
goal of creating virtual surgery systems is a valuable endeavor, but
its value lies not only in the destination, but the path taken to get
there. It is true that a plastic surgery \gls{pvis} was built and
demonstrated to actual medical students. But that artifact, and the
public recognition of it, is only part of the rewards from pursuing
this path. Along the way, multiple new technologies and techniques
were investigated, some of them were included in the surgical demo
while others were not. It would be tempting to say that only the final
result mattered, that time and effort were wasted on developments
which were ultimately passed over for inclusion. This would be a
mistake, however, but one born of understandable perspectives.

Approaching a project as large and as complex as virtual surgery
simulation, there is room for many disciplines. Medicine, computer
science, mathematics, engineering, and even business expertise all
have some part to play in the successful development and deployment of
these systems. Unfortunately, it is all too easy to be trapped by one
focus or another. Too much medicine and one can easily be lost in the
almost infinite complexity of reality, too paralyzed by the details
to get anything done. Too much mathematics can lead to equally
unrealistic outcomes, where elegant equations diverge too much from
the messy realities of biology or become too difficult to
implement. And it is possible to have too much computer science, as
well. Which is an odd statement to make in a defense of a computer
science thesis, but important to understand.

One of the dangers we face in simulation is that we can start to
believe anything is possible. It may be hard to do, or require a
complex dance of numerical techniques and new data structures, but
ultimately anything that occurs in reality should be possible to
simulate. This idea is seductive, it plays to the egos of researchers
and while it can lead them to new discoveries and techniques, it often
creates a bottomless pit of impracticality. This idea is related to a
topic brought up in Chapter \ref{chp:motivation} on avoiding premature
optimization. Optimization is generally thought of in simulation as
the process where we work out how to solve a problem faster, or more
correctly. But it can also refer to extreme focus, as in ``it was
optimized to do X, at the cost of excluding Y''. It is this latter
description which can be problematic, its practice responsible for
generating solutions which are excellent, but so narrow and limiting
that they become useless.

This is the problem that we are faced with in building surgical
simulations as computer scientists. We need to understand when to
balance our unending desires for a better solution with the practical
realities faced by other disciplines and each other. It was with this
consideration in mind that the technologies pursued and presented in
this document were chosen. It would have been easier, in some ways, to
focus on one small aspect of surgery simulation, perhaps delivering a
method to perfectly capture physical tissue responses or a better
approach for modeling cutting tissue. And there is little doubt that
these results would have been acceptable from a computer science
research perspective. Instead, the work presented in this document
focuses on a different idea: by providing a framework for fast,
general simulation, by supporting this framework with multiple
implementations catering to different needs, we can act as a stronger
foundation for future research. To use a metaphor, a strong solid base
of concrete rather than an elegant, but spindly tower of glass.

However, this is a computer science dissertation, so it is worth
spending some time discussing some of the technical takeaways from the
techniques presented in previous chapters as we conclude this document.


\section{Nonmanifold Level sets}
\label{sec:discussion}

Our pipeline for non-manifold level sets was been specifically created
for self-collision processing in deformable body simulations. There
are several diverse applications of the standard (manifold) level set
concept such as representing geometry, tracking dynamically evolving
interfaces, as a geometric query structure, etc. However, we believe
that the current methodology may require application-specific
embellishments to cater to broader tasks in modeling and simulation,
beyond what was needed for our collision-oriented proof of concept.
For example, when two dynamically evolving interfaces are brought
together, they may either be allowed to merge (as is typical in fluid
simulations) or overtake one another (e.g.\ the two separate branches
of the lips during self-collision). Thus, application-specific
semantics might be needed to use non-manifold level sets for dynamic
interfaces.  Finally, we showed two examples (Figures
\ref{fig:zplasty-and-net} and \ref{fig:zero-width}) where the
simulated elastic model resulted from conceptual ``cutting''
operations; in both cases we simply generated the non-manifold level
set from the final, cut geometry. If such cuts were progressively
enacted during simulation, our present implementation would sustain a
significant recompuation cost to rebuild the non-manifold implicit
representation. Incremental update following localized topology change
will be a significant direction of future investigation.

Currently, we do not extend the non-manifold level set values into a
narrow band outside the interface because we used the same grid both
for simulation and for storing level set values. If the extent is
predetermined, then we could generate a coarse non-manifold level set
mesh first and refine it as a post-process. Alternatively, one could
also ``grow'' the non-manifold level set by following characteristics,
but as mentioned above, one may or may not wish to merge overlapping
characteristics depending on the context. This issue is also related
to the question of ``evolving'' a non-manifold level set.  In future
work, we wish to address these questions targeting the specific
applications of multiphase flow and crack propagation.  For standard
Cartesian grid-based level sets, there exists an established
theoretical foundation for accurately computing high order gradients
even in the vicinity of singularities. For non-manifold level sets, an
extra level of complexity is introduced because there can be a
topological bifurcation in addition to a singularity.  To obtain valid
values in such complex situations, our current scheme reverts to first
order element-wise trilinear interpolation.  It would be interesting
to explore in future work if there can be a more accurate
representation for the interface in these cases.  In all of our
examples we generated the non-manifold level set at the same
resolution as the underlying lattice-based elasticity
discretization. This was motivated by the fact that the ability of the
elastic model to respond to collision events is restricted by the
resolution of the elastic model, limiting the hazards of interpolation
errors near high-curvature areas.  Although this has been a successful
heuristic for our test cases, there is no guarantee that such
resolution will always be accurate for proper collision detection and
response. A higher resolution can be used for the level set, if
desired; doubling the linear resolution, for example, would incur an
8x increase in level set construction, but no worse than a two-fold
increase of the backtrace cost, in practice (assuming collisions
remain relatively shallow).

Finally, the current implementation used Cartesian grids as template
meshes for generating the non-manifold level set. In future work, an
exploration of different representations such as
octrees~\cite{LosasGF:2004}, RLE
representations~\cite{HoustNBNM:2006,IrvinGLF:2006,ChentM:2011}, the
VDB data structure~\cite{Muset:2013} and SPGrid~\cite{SetalABS:2014}
would be warranted.

\section{Performance Optimization and Macroblocks}

In Chapters \ref{chp:parallelization} and \ref{chp:macroblocks}, we
looked at two different approaches for improving the performance of
computing forces and solving elastic deformation problems. To some
extent, the two approaches are the result of sequential
development. While the blocking organization allowed us to compute
forces for elasticity problems very efficiently, we found that the
addition of large numbers of force constraints and complex materials,
like muscle fibers, negatively impacted our ability to achieve
reasonable convergence with Conjugate Gradients, the iterative solver
we used for the surgery simulation prototype. At the time, this
was not a large problem, as we used the partially converged steps of
the simulation as an approximation of dynamics. Moreover, the
relatively rapid CG iterations allowed us to present more visual
updates to users, giving the platform a more responsive
feel. Nevertheless, we were hoping to improve upon the poor
convergence we were seeing, a quest which ultimately led us to the
macroblock idea. Unfortunately, despite achieving the goal of better
convergence and a better ratio of memory to computational resources,
this technique fell short in several important areas.

The most important limitations of our macroblock formulation are (a)
the restriction of our scheme to Cartesian lattice-based
discretizations of elasticity, and (b) the explicit lack of support
for self collisions or other elastic interactions that would couple
together disjoint parts of the mesh. We consciously limited our
preliminary exploration to applications of macroblocks within a
Newton-Raphson iterative solution scheme. In principle, there would
have been an opportunity to also consider using macroblocks in the
design of a highly efficient box smoother for multigrid, or as a
replacement of the local optimization step in projective dynamics; we
defer exploration of those interesting threads to future work. In
terms of practical performance, the macroblock technique also includes
an expensive factorization step. While this process should be
vectorizable, and thereby mitigating this cost somewhat, it remains a
significant bottleneck for interactive use cases at the moment. It is
this issue, along with the lack of self collision support, which has
prevented it from being included in the current iterations of the
plastic surgery simulation tool.

\section{Surgery Simulation Platform}

We found that our participants were extremely comfortable with aspects
of 3D visual navigation, even with the very rudimentary orientation
that was provided. Most participants found the visual examples of
procedures demonstrated to be very enlightening, with many of them
commenting that this visual illustration was the most informative
exposure they had for procedures they only knew from reference
literature (most of them had not witnessed these procedures in the
operating room). Almost no participant volunteered the lack of
self-collision processing as an observed omission, until the
interviewer explicitly asked them about this aspect (all
demonstrations in our workshop entailed full suturing of wound
closure). On the contrary, several participants identified
inaccuracies in the elastic behavior of the virtual tissues, finding
that our models appeared to be more ``permissive'' to manipulation
than real flesh tissue. An interface feature that was pointed out as
lacking, is the inability to appreciate (simply by looking at the
final sutured result) the deformation patterns that have resulted from
a certain repair; it was suggested that adding a texture (grid lines
or checkerboard patterns) on the skin surface would be much more
useful in evaluating tissue strain and deformation. We also received
requests for biologically inspired aids - in particular a
visualization of anatomical elements such as blood vessels in the
tissue being cut. Several users requested more traditional animation
features, such as timeline scrubbing and history undo, as well as
side-by-side views of different repair approaches for visual
comparison.

As was mention in Chapter \ref{chp:deployment}, the feature set of the
surgical tool was relatively modest. These limitations were a
conscious choice to allow us to focus on the platform itself,
balancing performance and extensibility. From a research standpoint,
our biggest perceived challenge is improving the accuracy of the
constitutive models for the biomaterials involved. Full self-collision
support will be essential for extending our work to procedures that
rely on contact, in addition to sutures, to model properly. We hope
that the non-manifold level set approach from Chapter
\ref{chp:nonmanifold} will prove to be a good solution for this
problem. Finally, a large class of reconstructive procedures cannot be
modeled with all incisions performed at the beginning of time; a
flexible topology change modeling system would be necessary to
incorporate a seamless ability for topology change, concurrent with
deformation.

  





%%% Local Variables:
%%% mode: latex
%%% TeX-master: "../document"
%%% End:
