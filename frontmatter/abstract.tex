
Techniques for simulating the behavior of elastic objects have matured
considerably over the last several decades, tackling diverse problems
from non-linear models for incompressibility to accurate
self-collisions. Alongside these contributions, advances in parallel
hardware design and algorithms have made simulation more efficient and
affordable than ever before. However, prior research often has had to
commit to design choices that compromise certain simulation features
to better optimize others, resulting in a fragmented landscape of
solutions. For complex, real-world tasks, such as virtual surgery, a
holistic approach is desirable, where complex behavior, performance,
and ease of modeling are supported equally. This dissertation caters
to this goal in the form of several interconnected threads of
investigation, each of which contributes a piece of an unified
solution. First, it will be demonstrated how various non-linear
materials can be combined with lattice deformers yield simulations
with behavioral richness and a high potential for parallelism. This
potential will be exploited to show how a hybrid solver approach based
on large macroblocks can accelerate the convergence of these
deformers. Further extensions of the lattice concept with non-manifold
topology will allow for efficient processing of self-collisions and
topology change. Finally, these concepts will be explored in the
context of a case study on virtual plastic surgery, demonstrating a
real-world problem space where these ideas can be combined to build an
expressive authoring tool, allowing surgeons to record procedures
digitally for future reference or education.


%%% Local Variables:
%%% mode: latex
%%% TeX-master: "../dissertation"
%%% End:
