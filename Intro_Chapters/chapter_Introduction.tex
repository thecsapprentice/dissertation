\chapter{Introduction}

This document describes a series of techniques and designs dealing
with the topics of modeling, discretization, and simulation in the
context of deformable solids and data structures related to regular
Cartesian grids. This general design was chosen for its ease of use
and for its potential for performance optimizations. In order to gain
these benefits, this document describes a series of contributions
which enable desirable simulation capabilities within this design
regime.

We use animations of reconstructive plastic surgery as the external
motivation for this work. This domain was used to define the scope of
the research presented in this document and the virtual surgery
systems developed within were designed as testing grounds for the
algorithmic and data structure contributions described in later
chapters.

Finally, solutions were explored for deploying these testing
platforms, demonstrating practical feasibility beyond pure computer
science research.

\section{Thesis}


This dissertation aims to support the following statement: Creating
virtual simulators for craniofacial reconstructive plastic surgery has
reached the point of technical feasibility. This capability is
demonstrated using craniofacial reconstructive surgery as the
benchmark; both for its intrinsic value and the degree of complexity
and challenges it exemplifies.

In pursuit of this goal, this document will describe how current
practices for simulating elastic materials can be combined in a
holistic fashion to optimize for performance and practical
usability. In the process, limitations with the current approaches
will be explored and, in some cases, alternative techniques will be
proposed to solve technical challenges that our benchmark application
exposes.


\section{Motivation}

Physicians have been seeking better methods to capture human anatomy
and function, both normal and pathological, for the purpose of healing
since the earliest days of modern surgical theory. From the anatomical
drawings of De Vinci, to more modern practices of constructing
realistic simulacra, surgeons, and their students, have been pursuing
tools that allow them to practice their skills before operating on
real patients. Existing research shows the benefits of engaging in
these practice sessions in a virtual non-invasive
setting~\citep{GallaRCHFMSS:2005}. Practiced surgeons make fewer
mistakes and can use preparation sessions to plan new approaches
safely.

This general philosophy, which can be summed up with the classic
proverb of ``measure twice, cut once'', is practiced by many high risk
professions. From flight school to driving simulators, computer
constructed virtual environments have become an integral part of
training highly skilled professionals. The reasoning is three-fold:
computer simulations are relatively low cost and are easy to reset and
reconfigure quickly, unforeseen parameters and situations can be
introduced more easily than physical environments, and a trainee's
progress can be easily recorded for later review. With these
advantages over purely physical training environments, why are
surgeons still using aids such as diagrams, physical mannequins, and
cadavers?

The part of the answer is that they often lack any better
alternatives. Performing surgery is a complex task involving a
combination of dexterous and cognitive, often spatial reasoning,
skills~\citep{GallaRCHFMSS:2005}. Tools that support all of these areas
are difficult to get right, and most attempts to build technological
aids have focused on subsets of the skills required. Historically,
these have been the dexterous skills, which many authors have tried to
solve with a variety of haptic simulation
techniques~\citep{MendoL:2003, LindbT:2007}. While these surgical
simulation philosophies are useful, and have been used in commercial
products~\citep{SUSAC:2002--2014}, they don't really meet the need of
training cognitive skills. This need varies across surgical
specialties - reconstructive plastic surgery, which is the focus of this document,
requires the surgeon to have internalized geometrical intuitions in
order to manipulate tissue in the three dimensional space of the human
body. It is this lack that has kept traditional, less technological
aids as the core of many plastic surgery training programs.

In comparison to internal surgery, plastic surgery is challenged by the
practical reality that the results of any operation will be visible to
others. This fact adds an additional constraint onto practitioners;
not only must their work be as technically correct as before, but they
must also be considering the final aesthetics of their procedures. It
follows then that a simulator for plastic surgery operations must
provide an environment for surgeons to freely practice design, as well
as correctly display the outcomes.

\section{Contributions}

\paragraph{General Material Support} We present techniques for
accommodating general classes of materials, including nonlinear
materials and anisotropic materials such as muscles, in the context of
a Cartesian grid-based discretization. This feature is supported
despite the challenge of simultaneously supporting additional
simulation constraints such as parallelism, topology change, and collision handling.

\paragraph{Hybrid Grids for Non-Manifold Embedding} We demonstrate a
augmented data structure that enables the resolution of thin, sub-voxel material features
within an otherwise standard hexahedral grid embedding context. We do
this by combining implicit topology and explicit topology together,
creating a hybrid data structure that has large performance potential
and modeling flexibility. We demonstrate this data structure in the
context of surgical operations with complex, thin incisions created by
user input.

\paragraph{SIMD Parallelization Framework} To relieve the burden of
writing streaming, vectorized numerical kernels, we developed a
framework for writing architecture independent, SIMD structured
kernels while presenting a API resembling scalar-style code. We
demonstrated this framework by constructing kernels for elastic
simulations, showing how even large kernels can be successfully
vectorized without introducing inefficiencies from automatic compiler
vectorization.

\paragraph{Non-Manifold Level Sets} We propose a data structure for
discretizing a level set over a non-manifold domain, allowing the
capture of implicit geometry with zero width incisions and overlapping
regions. Additionally, we provide algorithms for important tasks, such
as locating the nearest surface location from an interior point,
enabling the use of the data structure in self collision scenarios for
elastic simulation.

\paragraph{Macroblock Solver Design} We designed a hybrid
iterative-direct solver for elastic materials defined over hexahedral
grids, which divides the domain into self-contained macroblocks. The
interior of each macroblock is solved in a direct fashion, using a
cache-friendly, hierarchical factorization approach, while the
interfaces between macroblocks are solved iteratively. This technique
provides excellent convergence for non-linear materials, reminiscent
of direct solvers, while remaining fast and tuneable, like iterative
solvers.

\paragraph{Deployment Methodology for Remote Simulation} To support
use cases such as classrooms or remote collaborations, we developed a
prototype surgical simulation system that supports high performance
local and remote deployments. By employing modern web technologies, we
are able to support cross-platform, multi-user shared simulation
environments over the network. This approach provides good scalability
across multiple clients, reduced infrastructure costs, and better long
term maintenance options.

\section{Carteisian Grids as Model Representations}

Capturing the shape of deformable models has been accomplished by a
multitude of methods, including tetrahedral meshes, point clouds,
cages, and grids. This last method, which includes Cartesian grid
representations, has a number of benefits, including a simple and
procedurally defined topology along with great potential for
performance optimizations. Yet, these advantages do not come free:
\begin{itemize}
  \item Grid based representations are only an approximation of the
    object's surface.
  \item We often wish to capture objects with less regularity in shape
    and behavior than the structure of the grid allows, particularly in
    anatomical scenarios.
  \item Changing the topology of the model fundamentally breaks the
    regular topology of the grid data structure.
  \item Current methods for high performance collision handling are
    not designed for grid based representations. Moreover, there
    remains a challenge in turning the potential for performance in
    grid based representations into realizable gains.
  \end{itemize}
  Handling all of these concerns simultaneously adds to the difficulty
  of the task. The work presented in this document describes how these
  issues can be addressed. In particular, techniques for infusing
  additional topological flexibility into grid representations,
  allowing them to effectively capture thin, sub-cell material, while
  not giving up on performance opportunities will be covered in
  Chapters \ref{chp:nonmanifold} and \ref{chp:parallelization}. In
  order to capitalize on the performance potential, the regularity of
  the grid based representation will be used to build efficient
  streaming kernels in Chapter \ref{chp:parallelization}. Taking the
  abstraction higher, in Chapter \ref{chp:macroblocks}, we will show
  how multiple cells in the grid can be ganged together, creating
  larger macroblocks for improved performance and to more
  effectively capture nonlinear behaviors.

\section{Parallelism Concerns with Modern Hardware}

Modern hardware and modern simulation techniques are currently
intersecting with a high level of maturity on both sides. This brings
the possibility of being able to run large simulations on fairly
commodity hardware at near real-time rates, something once considered
impractical. However, simply running existing simulation
implementations on current hardware does not necessarily mean high
performance. Algorithms and techniques must be adapted to the
underlying performance mechanisms in modern computational hardware
architecture - namely thread and vector based parallelism. Each of
these mechanisms carries its own caveats and idiosyncrasies which must
be accounted for in order to gain the most benefit. Unfortunately,
despite the advances in simulation capability and behaviors, less
attention has been paid to these parallelism concerns, often leading
to algorithms which are poorly structured to take advantage of both
forms of parallelism simultaneously. This concern is only complicated
by the fact that current computation ability, expressed in operations per
second, is about two orders of magnitude greater than the amount of
data than can be transferred into main memory. This fact leads to an
awkward situation where even taking advantage of parallelism could
mean the algorithm is starved of data, negating the performance benefits.

The work presented in this document attempts to address these issues
in two major fashions. First, a framework for building vectorized
numerical kernels is presented in Chapter
\ref{chp:parallelization}. This approach delivers a scalar-like API to
the developer, while being able to generate vectorized code for
multiple architectures and architectural widths automatically behind
the scenes. This allows developers to focus on algorithmic
correctness, while not loosing the benefits of vectorization. The
second intervention is the development of a solver for macroblocks
which uses a delicate mirroring pattern to both expose large amounts
of vectorization friendly computation and to keep the memory bounds of
the operation within first level processor caches. This creates a
lower demand for memory bandwidth, at the trade-off of additional
computation, improving upon the memory-computation imbalance found in
modern hardware.

\section{Practical Deployment of Surgical Simulations}

Surgical simulation has been a popular area for deformable solid
researchers due to the inherent challenges in the
space. Unfortunately, most of the systems developed from this research
never find their way into the hands of the expected users. Commercial
simulation tools have fared better, but they are often restricted to
expensive and bulky workstations, capable of serving a limited number
of people simultaneously.

The work in this document attempts to address these issues by
exploring methods for practically deploying surgical simulation
software to a wider audience of users. This problem has been tackled
along two fronts. First, we have attempted to build fast and
interactive simulations using commodity hardware, while not
compromising on simulated features like collisions and nonlinear
materials. This allows for a wider range of platforms to be used and
not restricting the system to exotic, specific requirements. Second,
we have explored a wide range of implementations and approaches for
delivering software to users. These include both local and remote
simulation designs, making use of modern web technologies, and taking
a careful look at third party library implications. This approach was
demonstrated in a live demonstration of the surgical simulation
software for medical students. The details of this development strategy are
covered in Chapter \ref{chp:deployment}.

\section{Outline}

What follows is a short outline of the remaining chapters within this
dissertation.

\begin{description}
\item[Chapter 2] In Chapter \ref{chp:motivation}, the motivation for
  this document is dissected in more detail, exploring both technical
  domain considerations and design philosophies.
\item[Chapter 3] A technical deconstruction of basic deformable solid
  simulation practices is covered in Chapter \ref{chp:engineering},
  along with an introduction to notable technical challenges.
\item[Chapter 4] An examination of related work is covered in Chapter
  \ref{chp:relatedwork}, placing the contributions of this document in
  a broader context.
\item[Chapter 5] Chapter \ref{chp:nonmanifold} demonstrates how the
  regularity of Cartesian grids can be combined with desired amounts
  of topological flexibility, both for capturing model geometry and for
  contact scenarios. 
\item[Chapter 6] Chapter \ref{chp:parallelization} continues with a
discussion around thread and vector based parallelism in the context
of Cartesian grids.
\item[Chapter 7] Chapter \ref{chp:macroblocks} describes a new method
  for solving for elastic deformations which constructs larger
  macroblocks for better convergence.
\item[Chapter 8] Chapter \ref{chp:deployment} describes practical
  deployment concerns and delivers a critique of the various
  implementation options for surgical simulation systems.
\item[Chapter 9] Chapter \ref{chp:discussion} concludes this document
  with a final look back at completed work and considers future work
  along with current shortcomings.
\end{description}

%%% Local Variables:
%%% mode: latex
%%% TeX-master: "../document"
%%% End:
