\chapter{Introduction}

This document describes a set of techniques and design choices
dealing with the processes of modeling, discretization, and simulation
of elastic deformable solids on data structures related to regular
Cartesian grids. The decision to use Cartesian grid representations
for deformable bodies was made due to their ease of use and the
substantial potential for performance optimizations. This document
describes a series of contributions which enable the joint support of
a number of desirable simulation capabilities within this design
regime, and allow the performance potential to materialize in the
context of real-world applications.

We use the task of crafting instructional animations of reconstructive
plastic surgery as the driving application and source of motivation
for this work. This domain was used to define the scope of the
research presented in this document and the virtual surgery systems
developed within were leveraged as testing grounds for the algorithmic
and data structure contributions described in this dissertation.

Finally, this dissertation explores the practical challenges of
deploying these algorithmic contributions in the context of a
practical and usable interactive system, addressing engineering and
deployment issues that extend beyond the details of the constituent
core algorithms.

\section{Thesis}

This dissertation aims to support the following statement: Creating 
virtual simulators for craniofacial reconstructive plastic surgery has
reached the point of technical feasibility. This capability is
demonstrated using craniofacial reconstructive surgery as the
benchmark, both for its intrinsic value and the degree of complexity
and challenges it exemplifies.

In pursuit of this goal, this document will describe how current
practices for simulating elastic materials can be combined in a
holistic fashion to optimize for performance and practical
usability. In the process, limitations with the current approaches
will be explored and, in some cases, alternative techniques will be
proposed to solve technical challenges that our benchmark application
exposes.


\section{Motivation}

Physicians have been seeking better methods to capture human anatomy
and function, both normal and pathological, for the purpose of healing
since the earliest days of modern surgical theory. From the anatomical
drawings of Da Vinci, to more modern practices of constructing
realistic simulacra, surgeons, and their students, have been pursuing
tools that allow them to practice their skills before operating on
real patients. Existing research shows the benefits of engaging in
these practice sessions in a virtual, non-invasive
setting~\citep{GallaRCHFMSS:2005}. Practiced surgeons make fewer
mistakes and can use preparation sessions to plan new approaches
safely.

This general philosophy, which can be summed up with the classic
proverb of ``measure twice, cut once'', is practiced by many high risk
professions. From flight school to driving simulators, computer
constructed virtual environments have become an integral part of
training highly skilled professionals. The reasoning is three-fold:
computer simulations are relatively low cost and are easy to reset and
reconfigure quickly, unforeseen parameters and situations can be
introduced more easily than physical environments, and a trainee's
progress can be easily recorded for later review. With these
advantages over purely physical training environments, why are
surgeons still using aids such as diagrams, physical mannequins, and
cadavers?

Part of the answer involves the reality that better,
technology-assisted alternative educational aids are largely scarce
and mature solutions are narrowly scoped.  Performing surgery is a
complex task involving a combination of dexterous and cognitive, often
spatial reasoning, skills~\citep{GallaRCHFMSS:2005}. Tools that
support all of these areas are difficult to get right, and most
attempts to build technological aids have focused on subsets of the
skills required. Historically, these have been the dexterous skills,
which many authors have tried to solve with a variety of haptic
simulation techniques~\citep{MendoL:2003, LindbT:2007}. While these
surgical simulation philosophies are useful, and have been used in
commercial products~\citep{SUSAC:2002--2014}, they don't really meet
the need of training cognitive skills. This need varies across
surgical specialties - reconstructive plastic surgery, which is
highlighted in this document, requires the surgeon to have
internalized geometrical intuitions in order to manipulate tissue in
the three dimensional space of the human body. The lack of tools
supporting this type of knowledge has kept traditional, less
technological aids as the core of many plastic surgery training
programs.

In comparison to internal surgery, plastic surgery is challenged by the
practical reality that the results of any operation will be visible to
others. This fact adds an additional constraint onto practitioners;
not only must their work be as technically correct as before, but they
must also be considering the final aesthetics of their procedures. It
follows then that a simulator for plastic surgery operations must
provide an environment for surgeons to freely practice design, as well
as correctly display the outcomes.

\section{Contributions}

\paragraph{General Material Support} We present techniques for
accommodating general classes of materials, including nonlinear
mechanical properties, and anisotropic media such as muscles, in the
context of a Cartesian grid-based discretization. This generality is
supported despite the challenge of simultaneously accommodating
additional simulation constraints such as parallelism, topology
change, and collision handling.

\paragraph{Hybrid Grids for Non-Manifold Embedding} We demonstrate an
augmented data structure that enables the resolution of thin,
sub-voxel material features within an otherwise standard hexahedral
grid embedding context. This is accomplished by combining the implicit
topology of a grid and the flexibility of an explicit mesh structure,
creating a hybrid data structure that has large performance potential
and modeling versatility. We demonstrate this data structure in the
context of surgical operations with complex, thin incisions created by
user input.

\paragraph{A Thread- and Vector-conscious Parallelization Framework}
We developed a programming paradigm and an object-oriented code
infrastructure for bridging the performance divide between
hand-optimized numerical kernels and what compiler optimizations were
able to provide in the context of complex simulation tasks. This
framework drastically simplifies the developer's effort, generating
highly optimized SIMD code while presenting a API resembling
scalar-style semantics. We demonstrated this framework by constructing
kernels for elastic simulations, showing how even large kernels can be
successfully vectorized without suffering inefficiencies of automatic
compiler vectorization.

\paragraph{Non-Manifold Level Sets} We propose a data structure for
discretizing a level set over a non-manifold domain, allowing the
capture of implicit geometry with zero width incisions and overlapping
regions. Additionally, we provide algorithms for important tasks, such
as locating the nearest surface location from an interior point,
enabling the use of the data structure in self collision scenarios for
elastic simulation.
 
\paragraph{Macroblock Solver Design} We designed a hybrid
iterative-direct solver for elastic materials defined over hexahedral
grids, which divides the domain into self-contained abstractions of
simulation elements, labeled macroblocks. The interior of each
macroblock is solved in a direct fashion, using a cache-friendly,
hierarchical factorization approach, while the interfaces between
macroblocks are solved iteratively. This technique provides excellent
convergence for non-linear materials, inheriting robustness properties
of direct solvers, while remaining fast and tunable, like iterative
solvers.

\paragraph{Deployment Methodology for Remote Simulation} In order to
facilitate easy and cost-effective deployment and collaboration, we
developed a prototype surgical simulation system that combines
lightweight front-end clients with specialized remote simulation
servers. By employing modern web technologies, we are able to support
cross-platform, multi-user shared simulation environments over the
network. This approach provides good scalability across multiple
clients, reduced infrastructure costs, and better long term
maintenance options.

\section{Cartesian Grids as Model Representations}

Capturing the shape of deformable models in visual computing
applications has been accomplished in conjunction with a variety of
geometry representations, including tetrahedral meshes, point clouds,
cages, and grids. This last method, which includes Cartesian grid
representations, has a number of benefits, including a simple and
procedurally defined topology along with an excellent potential for
performance optimizations. Yet, these advantages come with
important caveats:
\begin{itemize}
\item Grid based representations form only an approximation of the
  object's surface.
\item Although the use of grids provides regularity at the data
  structure level, irregularity can also manifest in other ways, such
  as the heterogeneity of material properties, especially in models
  inspired by anatomy.
\item Many of the ways that model topology might be required to change
  in scenarios of cutting or fracture can jeopardize the regular
  implicit topology of the grid data structure.
\item Several established methods for collision handling are not
  optimized for grid-based representations, and even less so for the
  circumstances that can emerge from topology change.
\item Finally, translating the \emph{potential} for performance in
  grid based representations into \emph{practical gains} in a
  fully-featured interactive system is a nontrivial proposition.
\end{itemize}
Handling all of these concerns simultaneously is the essence of the
challenge at hand.  The work presented in this dissertation describes
methods for addressing these issues. In particular, techniques for
infusing additional topological flexibility into grid representations,
allowing them to effectively capture thin, sub-cell material, while
not giving up on performance opportunities will be covered in Chapters
\ref{chp:nonmanifold} and \ref{chp:parallelization}; these chapters
also address collision processing within the same framework.  In order
to capitalize on the performance potential, the regularity of the grid
based representation will be used to build efficient streaming kernels
in Chapter \ref{chp:parallelization}. Additionally, higher level
abstractions, described in Chapter \ref{chp:macroblocks}, are
possible: grouping together multiple cells and creating larger
macroblocks for improved performance and to more effectively capture
nonlinear behaviors.

\section{Parallelism Concerns with Modern Hardware}

Modern hardware and modern simulation techniques are currently
intersecting with a high level of maturity on both sides. This brings
the possibility of being able to run large simulations on commodity
hardware at near real-time rates, something once considered
impractical. However, simple reuse of existing simulation
implementations on current hardware does not always result in
competitive performance. Algorithms and techniques must be adapted to
the underlying performance mechanisms in modern computational hardware
architecture - namely thread and vector based parallelism. Each of
these mechanisms carries its own caveats and idiosyncrasies which must
be accounted for in order to gain the most benefit. Unfortunately,
despite the advances in simulation capability and behaviors, less
attention has been paid to these parallelism concerns, often leading
to algorithms which are poorly structured to take advantage of both
forms of parallelism simultaneously. Although for some application
domains the performance divide between platform-specific and
platform-agnostic algorithmic development might be limited, in the
case of performance-conscious iteractive applications, such as the
clinical domain we target, the wasted performance opportunities are too
severe to ignore.

This concern is only complicated by the fact that current arithmetic
bandwidth, expressed in operations per second, is about two orders of
magnitude greater than the available memory bandwidth on most
platforms. This impacts a large majority of low-level numerical
algorithms that deformable simulations are built on; as a consequence,
the source of inadequate performance in interactive simulation
scenarios is often traced in our work to imbalances in the
memory-computation mix of the underlying algorithms. Thus, many of our
interventions that prove most effective in restoring interactive
performance focus on the efficiency of low-level throughput-sensitive
kernels.

The work presented in this document attempts to address these issues
in two key aspects. First, a framework for building vectorized and
multithreaded numerical kernels is presented in Chapter
\ref{chp:parallelization}. This approach delivers a scalar-like API to
the developer, while being able to generate efficient parallelized
code for multiple architectures and architectural widths automatically
behind the scenes. This allows developers to focus on algorithmic
correctness, while not loosing the benefits of vectorization. The
second intervention is the development of a solver for local
neighborhoods of grid cells (macroblocks) as discussed in Chapter
\ref{chp:macroblocks}, which uses a delicate mirroring pattern to both
expose large amounts of vectorization friendly computation and to keep
the memory bounds of the operation within first level processor
caches. This creates a lower demand for memory bandwidth, at the
trade-off of additional computation, improving upon the
memory-computation imbalance found in modern hardware.

\section{Practical Deployment of Surgical Simulations}

Surgical simulation has been an active area for deformable solid
research due to its intrinsic value as well as its potential as a
catalyst for algorithmic innovation. Unfortunately, some of the most
promising results from basic research are confronted with complex
challenges upon attempted integration into a comprehensive real-world
system. Commercial surgical simulation tools have fared better, but
they are often restricted to expensive and bulky workstations, only
support a small number of simultaneous users, and provide a limited
feature set.

The work in this document attempts to address these issues by
exploring methods for practically deploying surgical simulation
software to a wider audience of users. This problem has been tackled
along two fronts. First, we have attempted to build fast and
interactive simulations using commodity hardware, while not
compromising on simulated features like collisions and nonlinear
materials. This allows for a wider range of platforms to be used and
not restricting the system to exotic, specific requirements. Second,
we have explored a wide range of implementations and approaches for
delivering software to users. These include both local and remote
simulation designs, making use of modern web technologies, and taking
careful look at third party library implications. This approach was
demonstrated in a live pilot deployment of the surgical simulation
software for medical students. The details of this development
strategy are covered in Chapter \ref{chp:deployment}.

\section{Outline}

What follows is a short outline of the remaining chapters within this
dissertation. In Chapter \ref{chp:motivation}, the motivation for this
document is dissected in more detail, exploring both technical domain
considerations and design philosophies.  A technical deconstruction of
basic deformable solid simulation practices is covered in Chapter
\ref{chp:engineering}, along with an introduction to notable technical
challenges.  An examination of related work is covered in Chapter
\ref{chp:relatedwork}, placing the contributions of this document in a
broader context.  Chapter \ref{chp:nonmanifold} demonstrates how the
regularity of Cartesian grids can be combined with desired amounts of
topological flexibility, both for capturing model geometry and for
contact scenarios.  Chapter \ref{chp:parallelization} continues with a
discussion around thread and vector based parallelism in the context
of Cartesian grids.  Chapter \ref{chp:macroblocks} describes a new
method for solving for elastic deformations which constructs larger
macroblocks for better convergence behavior.  Chapter \ref{chp:deployment}
describes practical deployment concerns and delivers a critique of the
various implementation options for surgical simulation systems.
Chapter \ref{chp:discussion} concludes this document with a final look
back at completed work and considers future work along with current
shortcomings.

%%% Local Variables:
%%% mode: latex
%%% TeX-master: "../document"
%%% End:
