\chapter{Introduction}

Physicians have been seeking better methods to capture human anatomy and
its pathologies for the purpose of healing since the earliest days of
modern surgical theory. From the anatomical drawings of De Vinci, to more
modern practices of constructing realistic simulacra, surgeons, and
their students, have been pursuing tools that allow them to practice
their skills before operating on real patients. Existing research
shows the benefits of engaging in these practice
sessions~\citep{GallaRCHFMSS:2005}. Practiced surgeons make fewer
mistakes and can use preparation sessions to plan new approaches
safely.

This general philosophy, which can be summed up with the classic
proverb of ``measure twice, cut once'', is practiced by many high risk
professions. From flight school to driving simulators, computer
constructed virtual environments have become an integral part of
training highly skilled professionals. The reasoning is three-fold:
computer simulations are relatively low cost and are able to be reset
quickly, unforeseen parameters and situations can be introduced more easily
than physical environments, and a trainee's progress can be easily
recorded for later review. With these advantages over purely physical
training environments, why are surgeons still using aids such as
diagrams, physical mannequins, and cadavers?

The part of the answer is that they often lack any better
alternatives. Performing surgery is a complex task involving a
combination of dexterous and cognitive, often spatial reasoning,
skills~\citep{GallaRCHFMSS:2005}. Tools that support all of these areas
are difficult to get right, and most attempts to build technological
aids have focused on subsets of the skills required. Historically,
these have been the dexterous skills, which many authors have tried to
solve with a variety of haptic simulation
techniques~\citep{MendoL:2003, LindbT:2007}. While these surgical
simulation philosophies are useful, and have been used in commercial
products~\citep{SUSAC:2002--2014}, they don't really meet the need of
training cognitive skills. This need varies across surgical
specialties - reconstructive plastic surgery, which is the focus of this document,
requires the surgeon to have internalized geometrical intuitions in
order to manipulate tissue in the three dimensional space of the human
body. It is this lack that has kept traditional, less technological
aids as the core of many plastic surgery training programs.

In comparison to internal surgery, plastic surgery suffers from the
practical reality that the results of any operation will be visible to
others. This fact adds an additional constraint onto practitioners;
not only must their work be as technically correct as before, but they
must also be considering the final aesthetics of their procedures. It
follows then that a simulator for plastic surgery operations must
provide an environment for surgeons to freely practice design, as well
as correctly display the outcomes.

This dissertation aims to support the following statement: Creating
virtual simulators for craniofacial reconstructive plastic surgery has
reached the point of technical feasibility. This capability is
demonstrated using craniofacial reconstructive surgery as the
benchmark; both for its intrinsic value and the degree of complexity
and challenges it exemplifies. In pursuit of this goal, this document
will describe how current practices for simulating elastic materials
can be combined in a holistic fashion to optimize for performance and
practical usability. In the process, limitations with the current
approaches will be explored and, in some cases, alternative techniques
will be proposed to solve technical challenges that our benchmark
application exposes.


%%% Local Variables:
%%% mode: latex
%%% TeX-master: "../document"
%%% End:
